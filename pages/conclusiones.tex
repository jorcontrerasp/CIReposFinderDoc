La realización de este estudio me ha permitido, en primer lugar, ampliar mis conocimientos sobre las diferentes herramientas de integración continua existentes y conocer cuáles son las predominantes en entornos de alojamiento de proyectos software como GitHub y GitLab.
También he ampliado conocimientos en técnicas de ``research'' mediante la minería de repositorios utilizando las APIs de GitHub y GitLab para implementar un programa en Python, que es uno de los lenguajes que más está creciendo actualmente. 
Con 11.3 millones de usuarios, en el año 2021, Python ha sido, en cuanto a crecimiento, uno de los lenguajes de programación más a tener en cuenta, consiguiendo precisamente ocupar este vacío como favorito en desarrollo DS/ML, pero además siendo una opción para desarrollo de aplicaciones de IoT (internet de las cosas).

En primer lugar se estudiaron diversas herramientas de integración continua, tanto su funcionamiento como la estructura de los ficheros de configuración de trabajos automatizados con el objetivo de implementar un heurístico que fuese capaz de encontrar repositorios que usasen herramientas de este tipo.

A continuación se realizaron varios experimentos aplicando el heurístico, de los que se obtuvieron resultados suficientes como para poder tomar conclusiones al respecto.

Tras la realización de estos experimentos se puede destacar cómo cada vez es más habitual que los programadores que integran sus proyectos de forma continua utilicen herramientas propias de la plataforma en donde están alojando sus proyectos, en este caso GitHub Actions si el repositorio está en GitHub o GitLab CI en el caso de que se almacene en GitLab, probablemente debido a la fácil integración que puedan tener con sus proyectos al estar almacenados en su propia plataforma.

No obstante, también se siguen usando otros medios para poder emplear integración continua como puedan ser Travis o incluso Jenkins que en un principio parecía que iban a ser más habituales en proyectos de código abierto GitHub o GitLab. 

Además, se puede apreciar como en muchos de los repositorios obtenidos en el proceso utilizan más de una herramienta de integración continua, por lo general, una clásica como pueden ser las ya mencionadas Travis y Jenkins, y otra moderna, siendo esta última la propia de la plataforma. Esto se podría deber a que aún hay proyectos tratando de transitar de una herramienta de CI a otras más actuales.

En cuanto a los trabajos futuros, se intentarán ejecutar de algún modo los trabajos definidos en cada fichero YML o YAML de configuración y replicar tests automatizados para comprobar su efectividad. Además, se revisarán en cada repositorio subidas de código o ``commits'' antiguos para comprobar si en los inicios de los repositorios se utilizaban sistemas de integración continua distintos al que tengan actualmente, es decir, comprobar si se produjo algún tipo de migración de sistemas y con qué objetivo. Por último, mencionar que se pondrán los resultados a disposición de investigadores interesados en estudiar otros aspectos sobre integración continua.
