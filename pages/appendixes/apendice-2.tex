\section{Librería PyGilab}

AAA

\section{Generación de token de autenticación GitLab}

Para poder acceder a la API que proporciona la plataforma GitLab y poder obtener información sobre los diferentes proyectos de código abierto ``open source'' es necesario generar un token de autenticación de la siguiente manera:

\begin{enumerate}
    \item Desplegar ajustes de usuario pinchando en el icono situado en la parte superior derecha de la pantalla y acceder a la opción ``Preferences''.

    \begin{figure}[h]
        \centering
        \includegraphics[width=0.4\textwidth,clip=true]{\GitLabTokenA}
        \caption{Generar token GitHub. Paso 1.}
    \end{figure}

    \item En el desplegable situado a la izquierda de la pantalla, acceder a la opción de menú ``Access Tokens''.

    \begin{figure}[h]
        \centering
        \includegraphics[width=0.4\textwidth,clip=true]{\GitLabTokenB}
        \caption{Generar token GitHub. Paso 2.}
    \end{figure}

    \item A continuación aparecerá una pantalla para rellenar los campos necesarios para la generación del token de autenticación: nombre del token, fecha de validez y permisos.
    
    \begin{figure}[h]
        \centering
        \includegraphics[width=1\textwidth,clip=true]{\GitLabTokenC}
        \caption{Generar token GitHub. Paso 3.}
    \end{figure}

    \item Por último, una vez rellenos los campos de la pantalla con la información requerida, pinchar en el botón ``Create personal access token'' para generar el token de autenticación.

    \begin{figure}[h]
        \centering
        \includegraphics[width=0.5\textwidth,clip=true]{\GitLabTokenD}
        \caption{Generar token GitHub. Paso 4.}
    \end{figure}

\end{enumerate}