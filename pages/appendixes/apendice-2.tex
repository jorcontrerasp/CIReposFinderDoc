\section{Librería Python ``Gitlab''}

Minería de repositorios también se puede llevar a cabo sobre la plataforma de almacenamiento de software GitLab. Para ello es necesario utilizar la librería Python que ofrece la propia plataforma GitLab, conocida con el nombre de ``Python-GitLab''.

Una vez instalada la librería mediante el comando de instalación de paquetes desde un terminal (pip install python-gitlab), se puede importar en el proyecto de la siguiente manera:

\begin{lstlisting}[language=Python]
    import gitlab
\end{lstlisting}

Una vez importada la biblioteca, se puede generar un objeto de la clase ``GitLab'' que se utilizará para listar proyectos en base a un conjunto de criterios pasados como parámetros de búsqueda. Este objeto se construye de la siguiente forma:

\begin{lstlisting}[language=Python, caption=Autenticación en API de GitLab]
    token = aux.readFile("tokens/gitlab_token.txt")
    gl = gitlab.Gitlab('http://gitlab.com', private_token=token)
\end{lstlisting}

A partir de este objeto ya generado, se pueden listar proyectos:

\begin{lstlisting}[language=Python]
    projects = gl.projects.list(visibility='public', 
                                last_activity_after='2016-01-01T00:00:00Z', 
                                pagination='keyset', 
                                id_after=idAfter, 
                                page=1, 
                                order_by='id', 
                                sort='asc')
\end{lstlisting}

En este caso se buscarán proyectos de código abierto o públicos cuya última actividad es posterior al año 2016, partiendo desde el identificador de proyecto ``idAfter'' pasado como parámetro y ordenados por identificador de forma ascendente.

\section{Generación de token de autenticación GitLab}

Para poder acceder a la API que proporciona la plataforma GitLab y poder obtener información sobre los diferentes proyectos de código abierto ``open source'' es necesario generar un token de autenticación de la siguiente manera:

\begin{enumerate}
    \item Desplegar ajustes de usuario pinchando en el icono situado en la parte superior derecha de la pantalla y acceder a la opción ``Preferences''.

    \begin{figure}[h]
        \centering
        \includegraphics[width=0.4\textwidth,clip=true]{\GitLabTokenA}
        \caption{Generar token GitHub. Paso 1.}
    \end{figure}

    \newpage

    \item En el desplegable situado a la izquierda de la pantalla, acceder a la opción de menú ``Access Tokens''.

    \begin{figure}[h]
        \centering
        \includegraphics[width=0.4\textwidth,clip=true]{\GitLabTokenB}
        \caption{Generar token GitHub. Paso 2.}
    \end{figure}

    \item A continuación aparecerá una pantalla para rellenar los campos necesarios para la generación del token de autenticación: nombre del token, fecha de validez y permisos.
    
    \begin{figure}[h]
        \centering
        \includegraphics[width=1\textwidth,clip=true]{\GitLabTokenC}
        \caption{Generar token GitHub. Paso 3.}
    \end{figure}

    \newpage

    \item Por último, una vez rellenos los campos de la pantalla con la información requerida, pinchar en el botón ``Create personal access token'' para generar el token de autenticación.

    \begin{figure}[h]
        \centering
        \includegraphics[width=0.5\textwidth,clip=true]{\GitLabTokenD}
        \caption{Generar token GitHub. Paso 4.}
    \end{figure}

\end{enumerate}