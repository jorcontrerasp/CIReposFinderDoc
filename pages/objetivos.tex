En este trabajo se pretende realizar un estudio sobre herramientas de integración continua empleadas en proyectos software alojados en las plataformas GitHub y GitLab con el objetivo de ampliar conocimientos al respecto y, de esta forma, poder responder a la gran variedad de preguntas que surjan sobre el uso de dichos sistemas de CI.

Algunas de estas cuestiones a intentar resolver con este estudio son las siguientes:
\begin{compactitem}
    \item ¿Cuáles son los sistemas de CI más utilizados en cada plataforma?
    \item ¿Se suelen combinar más de un sistema de CI en un mismo repositorio?
    \item ¿Qué lenguajes de programación predominan en repositorios que utilizan sistemas de CI?
    \item ¿Cuántos ``jobs'' se suelen configurar en repositorios de código abierto?
    \item ¿Cuáles son los principales escenarios o ``stages'' en los que se suelen ejecutar trabajos automatizados?
\end{compactitem}

GitHub es una compañía sin fines de lucro que ofrece un servicio de hosting de repositorios almacenados en la nube utilizando el sistema de control de versiones Git. 
Además, cuenta con una API REST disponible para cualquier desarrollador que quiera implementar alguna aplicación relacionada con el servicio que ofrece. 
En este caso, para poder llevar a cabo el estudio, se implementa una aplicación encargada de buscar proyectos que puedan tener sistemas de integración continua utilizando el lenguaje de programación Python y, concretamente, la librería “PyGithub” que permite utilizar la versión 3 de la API ya mencionada.

Gitlab Inc. es una compañía de núcleo abierto y es la principal proveedora del software GitLab, un servicio web de control de versiones, DevOps y desarrollo de software colaborativo basado en Git. Además de un gestor de repositorios, el servicio ofrece también alojamiento de wikis y un sistema de seguimiento de errores, todo ello publicado bajo una licencia de código abierto, principalmente.
Fue escrito por los programadores ucranianos Dmitriy Zaporozhets y Valery Sizov en el lenguaje de programación Ruby con algunas partes reescritas posteriormente en Go, inicialmente como una solución de gestión de código fuente para colaborar con su equipo en el desarrollo de software. Luego evolucionó a una solución integrada que cubre el ciclo de vida del desarrollo de software, y más tarde a todo el ciclo de vida de DevOps. 
La arquitectura tecnológica actual incluye Go, Ruby on Rails y Vue.js.

En cuanto a los principales objetivos a cumplir en la realización de este trabajo, se definen los siguientes:
\begin{enumerate}
    \item Aprender el funcionamiento de la integración continua en proyectos software.
    \item Construir un conjunto de datos con información sobre herramientas de integración continua.
    \item Estudiar la forma en la que se utilizan esas herramientas de integración continua en proyectos ``open source''.
    \item Expandir conocimientos en programación Python.
    \item Afianzar conocimientos sobre técnicas de research en GitHub.
    \item Aprendizaje sobre técnicas de research en la plataforma GitLab.
\end{enumerate}