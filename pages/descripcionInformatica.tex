Para llevar a cabo este estudio se implementa una programa encargado de aplicar el heurístico desarrollado sobre los repositorios alojados tanto en GitHub como en GitLab.

\section{Herramientas utilizadas}
Las herramientas que se han utilizado para elaborar el programa son las siguientes:
\subsection{Python}
Es un lenguaje de programación cuya filosofía hace hincapié en la legibilidad del código. Sus principales características son las siguientes:
\begin{enumerate}
    \item Multiparadigma: ya que más que forzar a los programadores a adoptar un estilo de programación, permite varios estilos, soportando la orientación a objetos, la programación imperativa y la funcional.
    \item Interpretado.
    \item Dinámico: permitiendo que una variable pueda tomar valores de distinto tipo.
    \item Multiplataforma.
\end{enumerate}

\subsection{API de GitHub}
API REST que nos va a permitir utilizar diferentes métodos para obtener información acerca de los repositorios almacenados en GitHub.

\subsection{API de GitLab}
API REST que nos va a permitir utilizar diferentes métodos para obtener información acerca de los repositorios almacenados en GitLab.

\subsection{Visual Studio Code}
Visual Studio Code es un editor de código fuente ligero pero potente que se ejecuta en su escritorio y está disponible para Windows, macOS y Linux. Viene con soporte incorporado para JavaScript, TypeScript y Node.js y tiene un rico ecosistema de extensiones para otros lenguajes  (como C++, Java, Python, PHP o Go) y tiempos de ejecución (como .NET y Unity).
En este trabajo se va a utilizar tanto para la implementación del programa en python que se va a utilizar para obtener información sobre integración continua como para la escritura de la memoria final en LaTeX.

\section{Implementación Python}

\subsection{Ficheros Python}

\subsection{Librerías utilizadas}

\subsection{Ficheros resultantes}