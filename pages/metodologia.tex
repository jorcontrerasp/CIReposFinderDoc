Para la construcción de este conjunto de datos se va a utilizar una técnica de ``research'' conocida con el nombre de minería de repositorios.
Un repositorio software contiene una gran cantidad de información histórica y valiosa sobre el desarrollo general del sistema que trata (estado, progreso y evolución del proyecto) y esta técnica de ``research'' se va a centrar en la extracción y análisis de los datos heterogéneos disponibles en estos repositorios para descubrir información interesante, útil y procesable sobre el sistema.

En primer lugar, para obtener este conjunto de datos que nos permita analizar el funcionamiento de la integración continua en GitHub y GitLab, se van a enumerar diferentes herramientas de integración continua encontradas en repositorios seleccionados. 
Estos repositorios se elegiran en función de su repercusión en las plataformas, es decir, por su elevado número de estrellas o las bifurcaciones ``forks'' que tengan, ya que al tratarse de repositorios muy conocidos serán considerados como prometedores en cuanto a la utilización de herramientas de este tipo, objetivo de este estudio, para diferentes aspectos como por ejemplo la automatización de tests.

Tras seleccionar los repositorios, se analizan manualmente en busca de herramientas de integración continua para ir enumerándolas.

Una vez obtenida la lista de sistemas de integración continua a explorar, tanto en GitHub como en GitLab, se va a estudiar el funcionamiento de cada una de ellas y la forma en la que se construyen los ficheros de configuración que utilizan para realizar la automatización de trabajos. De esta forma, por cada uno de estos sistemas de CI, se establece un criterio único de localización de repositorios que emplean herramientas de integración continua y se conforma el heurístico que se va a utilizar para realizar el estudio.

Con el heurístico de localización de repositorios prometedores ya construido, se realiza un conteo de estas herramientas sobre una búsqueda de 500 repositorios GitHub y 500 repositorios GitLab, haciendo un total de 1000 repositorios en este experimiento inicial.

A continuación, se analizan los resultados manualmente para verificar la efectividad del heurístico, es decir, comprobar que cuando el heurístico haya encontrado un repositorio que utilice una herramienta de integración continua concreta, efectivamente use esa herramienta de integración continua.

Por cada proyecto de Github o GitLab se realiza lo siguiente:
\begin{compactitem}
    \item Localizar el sistema de CI que tiene el proyecto, buscando en su árbol de directorios los ficheros especificados en las carpetas que corresponda (se buscan todos los ficheros en cada repositorio ya que un mismo proyecto podría tener más de un sistema de CI configurado).
    \item Anotar en qué sistemas de CI el proyecto ha dado positivo.
    \item Presentación en una tabla de los resultados: nº de proyectos que usan cada sistema de CI.
\end{compactitem}
	
Con este primer experimento realizado, se establecen cuáles son las herramientas de integración continua más populares con el objetivo de realizar un análisis más exhaustivo sobre ellas.

A continuación se realiza un segundo experimento sobre un número mayor de repositorio, obteniendo exclusivamente más de 500 repositorios en cada plataforma GitHub y  GitLab que hayan dado positivo en algun sistema de CI, descartando por completo aquellos en los que no se ha encontrado nada. Es decir, este conjunto final de datos va a quedar conformado por proyectos que utilizan herramientas de integración continua.

En este segundo experimento, por cada proyecto, se estudian diferentes aspectos que no se tuvieron en cuenta en el experimento inicial:
\begin{compactitem}
    \item Número de trabajos.\cotutor{Igual, mejor hablar de jobs}
    \item Número de tareas.
    \item Número medio de tareas por trabajo.
    \item Momento en el que se ejecutan los trabajos: push, pull request, en ramas, schedule...
    \item Lenguajes de programación predominantes en cada sistema de CI.
    \item Etc.
\end{compactitem}

Finalmente, una vez analizados todos los datos obtenidos, se pueden sacar conclusiones y contestar a todas las preguntas que se formularon inicialmente.
	
Por lo tanto, en cuanto a la metodología del trabajo, los pasos seguidos son los siguientes:
\begin{enumerate}
    \item Identificación de sistemas de CI que se van a estudiar.
    \item Análisis de los sistemas de CI identificados.
    \item Implementación de un heurístico localizador de repositorios GitHub y GitLab que utilicen herramientas de integración continua.
    \item Conteo de repositorios tanto positivos como negativos a los que se les ha aplicado el heurístico.
    \item Análisis manual de estos repositorios para verificar el heurístico.
    \item Selección de las herramientas de integración continua más populares en cada plataforma.
    \item Aplicación del heurístico de tal forma que se obtengan más de 500 repositorios GitHub y más de 500 repositorios GitLab positivos.
    \item Análisis de los resultados.
\end{enumerate}

\section{Identificación de sistemas de CI}

En primer lugar localizamos los sistemas de CI que se van a estudiar. Para ello, se obtienen repositorios que sean conocidos en las plataformas GitHub y GitLab, filtrando por su número de estrellas y bifurcaciones realizadas por otros programadores de la comunidad sobre los mismos. Además, se añaden otras herramientas de integración continua que sean bastante conocidas en el mundo de la informática y la automatización de trabajos en proyectos software con el objetivo de realizar un análisis de herramientas de CI más amplio.
Por cada una de estas herramientas identificadas se realizan las siguientes tareas:
\begin{enumerate}
    \item Se estudia su funcionamiento.
    \item Se estudia la construcción del fichero de configuración utilizado para implementar la automatización de trabajos.
    \item Se buscan ejemplos de uso.
\end{enumerate}

A continuación, se especifican los sistemas de CI identificados y que serán utilizados para conformar el heurístico de búsqueda de repositorios:
\begin{compactitem}
    \item \underline{Jenkins}: búsqueda del fichero ``Jenkinsfile'' ubicado en la raíz del proyecto \cite{jenkins}.
    \item \underline{Travis}: búsqueda de los ficheros ``.travis-ci.yml'' o ``.travis.yml'' ubicados en la raíz del proyecto \cite{travisCI}.
    \item \underline{Circle CI}: búsqueda del fichero ``.circle-ci'' situado en el directorio ``circleci'' \cite{circleCI}.
    \item \underline{GitHub Actions}: búsqueda de ficheros YML o YAML situados en el directorio ``.github/workflows'' \cite{githubActions}.
    \item \underline{Azure Pipelines}: búsqueda del fichero ``azure-pipelines.yml'' en la raíz del proyecto o el directorio ``.azure-pipelines'' \cite{azurePipelines}.
    \item \underline{Bamboo}: búsqueda del fichero YML o YAML ``bamboo'' en el directorio ``bamboo-specs'' \cite{bamboo}.
    \item \underline{Concourse}: búsqueda de la carpeta ``tasks'' o de algún fichero YML en algun directorio ``concourse'' \cite{concourse}.
    \item \underline{GitLab CI}: búsqueda del fichero ``.gitlab-ci.yml'' en la raíz del proyecto \cite{gitlabCI}.
    \item \underline{Codeship}: búsqueda, en la raíz del proyecto, de alguno de estos ficheros ``codeship-services.yml'', ``codeship-steps.yml'' o ``codeship-steps.json'' \cite{codeship}.
    \item \underline{TeamCity}: búsqueda del fichero ``settings.kts'' en el directorio ``.teamcity'' \cite{teamcity}.
    \item \underline{Bazel}: búsqueda de alguno de los ficheros ``presubmit.yml'' o ``build\_bazel\_binaries.yml'' en el directorio ``.bazelci'' o el fichero ``.bazelrc'' en la raíz del proyecto \cite{bazel}.
    \item \underline{Semaphore CI}: búsqueda del fichero ``semaphore.yml'' en los directorios ``.semaphore'' o ``.semaphoreci'' \cite{semaphoreCI}.
    \item \underline{AppVeyor}: búsqueda del fichero ``Appveyor.yml'' en la raíz del proyecto.
\end{compactitem}
	
De esta forma, mediante el lenguaje de programación Python, se implementa un programa encargado de ejecutar el heurístico localizador de repositorios que emplean sistemas de CI.
