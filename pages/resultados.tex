\section{Experimento preliminar}
Se realiza un primer experimento ejecutando el programa implementado sobre 500 repositorios de la plataforma GitHub y 500 repositorios de la plataforma GitLab, todos ellos repositorios de código abierto u ``open source'', sin discriminar entre resultados positivos o negativos en la búsqueda, es decir, entre esos 1000 repositorios se pretende encontrar tanto repositorios que utilizan alguna de las herramientas de integración continua contempladas como repositorios que no las utilizan.

El objetivo de este experimento es conocer cuáles son los sistemas de integración continua más utilizados en repositorios de este tipo para posteriormente analizarlas más en profundidad en un segundo experimento.

Los parámetros con los que se ejecuta el programa son los siguientes:
\begin{compactitem}
    \item Para \textbf{\underline{GitHub}}:
    \begin{compactitem}
        \item created: \textgreater2016-01-01
        \item pushed: \textgreater2021-01-01
        \item stars: \textgreater=9500
        \item forks: \textgreater=800
        \item archived: false
        \item is: public
        \item onlyPositives: false
    \end{compactitem}
    \item Para \textbf{\underline{GitLab}}:
    \begin{compactitem}
        \item visibility: public
        \item last\_activity\_after: 2016-01-01T00:00:00Z
        \item stars: \textgreater25
        \item onlyPositives: false
    \end{compactitem}
\end{compactitem}

Tras lanzar el programa se obtienen los resultados reflejados en la tabla \ref{tab:tabla_p1}.

A simple vista se puede observar que predomina el uso de los sistemas de integración continua propios de cada plataforma, siendo GitHub Actions para repositorios almacenados en GitHub y GitLab CI para repositorios almacenados en GitLab.

A parte de estos dos sistemas de CI ya mencionados se localizan repositorios que utilizan el sistema de integración continua proporcionado por Travis CI, obteniendo 68 casos en repositorios almacenados en GitHub y otros 25 en repositorios almacenados en GitLab, dando un total de 93 repositorios en ambas plataformas, es decir, en 9'3\% del total de repositorios.

También encontramos repositorios que utilizan Jenkins, que en un principio, se consideró como uno de los sistemas que se iban a utilizar en un mayor número de repositorios junto con GitHub Actions y Travis CI.

Otros sistemas a tener en cuenta son Circle CI, Azure Pipelines y Bazel, obteniendo un total de 42, 6 y 8 repositorios que utilizan dichos sistemas; resultados considerados insuficientes como para profundizar en ellos en el experimento final a realizar.

Finalmente, mencionar también los 2 repositorios encontrados que utilizan la herramienta de integración continua Concourse, uno para GitHub y otro para GitLab, considerados como falsos positivos ya que a pesar de haber cumplido el criterio de búsqueda establecido no se utiliza esta herramienta de integración en ninguno de los dos casos. Esto se debe a que el criterio establecido para la búsqueda de este sistema de CI, ``tasks'', directorio propuesto por la documentación de Concourse para almacenar los ficheros de configuración de trabajos, es demasiado genérico como para encontrar repositorios que, teniendo ese directorio en su árbol de directorios, usen Concourse.

\section{Experimento final}
Con las conclusiones obtenidas tras la ejecución del primer experimento de este trabajo, se realiza un segundo experimento estudiando más en profundidad los sistemas de CI ofrecidos por Travis CI, GitHub Actions y GitLab CI, ya que son los sistemas de los que se han obtenido más información.

Para ello se modifica el programa de tal forma que se pueda analizar la forma en la que están construidos los ficheros YML o YAML de configuración de trabajos de las herramientas seleccionadas.

La principal mejora añadida para poder analizar dichos ficheros de configuración es la implementación de un script encargado de transformar cada fichero de configuración CI en objetos Python (ci\_yml\_parser.py) con el objetivo de poder manejar la información obtenida y construir los ficheros excel de resultados fácilmente.

Este script va a permitir añadir las siguientes columnas al excel principal de resultados:
\begin{compactitem}
    \item \textbf{STAGES}: columna en la que se almacenan los escenarios en los que se ejecutan los puntos del fichero de configuración considerados como trabajos.
    \item \textbf{NUM\_JOBS}: columna en la que se almacena el número de trabajos que se ejecutan en el repositorio en cuestión.
    \item \textbf{TOTAL\_TASKS}: columna en la que se almacena el número total de tareas de todos los trabajos automatizados en el repositorio.
    \item \textbf{TASK\_AVERAGE\_PER\_JOB}: columna en la que se almacena la media de tareas por trabajo.
\end{compactitem}

Con esta información y de manera automática, se generan otros ficheros excel de estadísticas relacionadas con estos ficheros de configuración de trabajos. Estas columnas van contener información en formato json y a modo de diccionario con la intención de diferenciar los datos obtenidos en de cada sistema de CI en particular, ya que puede darse el caso de que se encuentren repositorios que utilicen más de una herramienta de integración continua, que, como se puede ver en los resultados obtenidos, es algo más común de lo que en un principio se podía esperar.

Este segundo experimento consiste en ejecutar el programa tres veces para repositorios GitHub y una vez para repositorios GitLab, encontrando en cada una de estas ejecuciones exclusivamente repositorios con un resultado positivo en la búsqueda, es decir, repositorios que utilizan algún sistema de CI. 

Para ello se emplean las siguientes consultas de búsqueda:
\begin{compactitem}
    \item Para \textbf{\underline{GitHub}}:
    \begin{compactitem}
        \item \textbf{Ejecución nº 1}: se buscan repositorios GitHub públicos que sean relativamente recientes.
        \begin{compactitem}
            \item created: \textgreater2016-01-01
            \item pushed: \textgreater2021-01-01
            \item stars: \textgreater=9500
            \item forks: \textgreater=800
            \item archived: false
            \item is: public
            \item onlyPositives: true
        \end{compactitem}
        \item \textbf{Ejecución nº 2}: se buscan repositorios GitHub públicos que hayan sido creados antes de la aparición y auge del sistema de CI GitHub Actions, pero que se hayan seguido modificando recientemente. De esta forma se obtendrán repositorios que podrían haber migrado de algún sistema de CI distinto de las acciones de GitHub a el sistema proporcionado por GitHub, por ejemplo de Jenkins o Travis CI a GitHub Actions.
        \begin{compactitem}
            \item created: 2010-01-01..2016-01-01
            \item pushed: \textgreater2018-01-01
            \item stars: \textgreater=10500
            \item forks: \textgreater=1500
            \item archived: false
            \item is: public
            \item onlyPositives: true
        \end{compactitem}
        \item \textbf{Ejecución nº 3}: se buscan repositorios GitHub públicos que hayan sido creados antes de la aparición y auge del sistema de CI GitHub Actions y que se hayan dejado de modificar recientemente.
        \begin{compactitem}
            \item created: 2010-01-01..2016-01-01
            \item pushed: $<$2018-01-01
            \item stars: \textgreater=500
            \item forks: \textgreater=250
            \item archived: false
            \item is: public
            \item onlyPositives: true
        \end{compactitem}
    \end{compactitem}
    \item Para \textbf{\underline{GitLab}}: se buscan repositorios GitLab de forma genérica.
    \begin{compactitem}
        \item visibility: public
        \item last\_activity\_after: 2016-01-01T00:00:00Z
        \item stars: \textgreater25
        \item onlyPositives: true
    \end{compactitem}
\end{compactitem}

Una vez ejecutado el proceso partiendo de las consultas ya mencionadas, se obtienen los resultados comentados a continuación:

- \textbf{Ejecución nº 1 GitHub y GitLab}: 

A simple vista se puede observar en la tabla \ref{tab:tabla_f1_1}, al igual que en el experimento preliminar, el predominio de sistemas de CI propios de la plataforma en la que se ha ejecutado el programa, es decir, el predominio de GitHub Actions en repositorios almacenados en GitHub y de GitLab CI en repositorios almacenados en GitLab, debido probablemente a la correcta integración y la facilidad de uso de dichos sistemas sobre repositorios almacenados en sus propias plataformas, así como la utilidad de tener tanto el proyecto como la configuración de trabajos automatizados en un mismo sitio.

No obstante, también se encuentran otras herramientas de CI como Travis CI, Circle CI o Azure Pipelines.

Sobre el número de trabajos encontrados y la media de tareas por trabajo calculada en los sistemas de CI GitHub Actions, GitLab CI y Travis CI, datos reflejados en las tablas \ref{tab:tabla_f1_3} y \ref{tab:tabla_f1_7}, se sacan conclusiones similares a las ya obtenidas en el experimento preliminar. Sobre repositorios almacenados en GitHub se obtiene un número alto de trabajos en GitHub Actions con una media de 9'57 tareas por trabajo y sobre repositorios almacenados en GitLab, de forma inversamente proporcional a los trabajos encontrados en repositorios GitHub, se obtiene un número elevado de trabajos en GitLab CI, con una media de 9'03 tareas por trabajo. En cuanto a Travis CI se observa que se mantiene estable el número de trabajos, ya sea ejecutando el proceso sobre repositorios GitHub como sobre repositorios GitLab, con una media de entre 2 y 3 tareas por trabajo (2'75 en GitHub y 3'41 en GitLab).

También se puede observar un comportamiento análogo sobre los datos calculados sobre repositorios de GitHub que utilizan GitHub Actions y repositorios de GitLab que utilizan GitLab CI, ya que se obtienen un número similar de repositorios en cada caso, así como el número de trabajos automatizados junto con su media y mediana.

En cuanto a los lenguajes de programación utilizados para implementar los diferentes repositorios a los que se les ha aplicado el proceso, aparecen varios en común situados en la cúspide de los más utilizados, a tener en cuenta, ordenados de mayor a menos uso, los siguientes: Python, Javascript, Go, C++ y Java \ref{tab:tabla_f1_2a} - \ref{tab:tabla_f1_2b} y \ref{tab:tabla_f1_6a} - \ref{tab:tabla_f1_6b}.

Se observa que Python y JavaScript son los dos lenguajes de programación más utilizados en la implementación de repositorios qur utilizan herramientas de integración continua tanto en GitHub como en GitLab. Estos resultados tendrían explicación en que estos lenguajes de programación son muy utilizados para el desarrollo de aplicaciones web (en el caso de python utilizando el framework Django), que son un tipo de implementación muy propenso a utilizar herramientas de este tipo, sobre todo para la automatización de pruebas. No obstante, el lenguaje de programación Python también se utiliza bastante para el desarrollo del lado del servidor en aplicaciones cliente-servidor. 

Además, en cuanto a estos lenguajes de programación, también se obtienen datos estadísticos sobre los trabajos o ``jobs'' configurados en los ficheros YML de los sistemas de integración continua GitHub Actions, GitLab CI y Travis CI \ref{tab:tabla_f1_4} - \ref{tab:tabla_f1_8}.

Otro aspecto a tener en cuenta sobre los datos obtenidos en el proceso de búsqueda son los escenarios en los que se ejecutan los diferentes trabajos automatizados por estos sistemas de CI. En este caso, tal y como se expone en las tablas \ref{tab:tabla_f1_5} y \ref{tab:tabla_f1_9}, encontramos diferencias notables, ya que en repositorios almacenados en GitHub predominan escenarios como ``push'', ``pull\_request'' o ``schedule'' y en GitLab predominan otros como ``build'', ``test'', ``deploy'' o ``release''.

- \textbf{Ejecución nº 2 GitHub}:

En esta segunda ejecución sobre repositorios de la plataforma GitHub, a pesar de haber intentado localizar repositorios con herramientas de CI distintas de GitHub Actions buscando solamente aquellos que fueron creados antes de la existencia de dicha herramienta, si se consultan los datos de la tabla \ref{tab:tabla_f2_1} se puede ver como sigue predominando el uso de las acciones de la propia plataforma de GitHub, probablemente debido a que en algún momento de tiempo utilizaban otros sistemas de CI y se migraron recientemente para utilizar GitHub Actions. En trabajos futuros se podrían comprobar ``commits'' o subidas de código antiguas de estos repositorios para comprobar si esta suposición es o no correcta.

No obstante, en otros sistemas de integración continua como Travis CI o Circle CI se obtienen más datos con respecto a los obtenidos en la primera ejecución de este experimento.

En cuanto a los datos que se obtienen relativos a los lenguajes de programación empleados en cada repositorio, trabajos automatizados, tareas y escenarios en los que son ejecutados dichos trabajos son muy similares a los obtenidos en la primera ejecución de este experimento final. Se sigue observando el predominio de los lenguajes de programación Python y JavaScript, pero esta vez obteniendo más información sobre JavaScript, posicionado en la cúspide de lenguajes de programación más utilizados con un total de 196 repositorios encontrados \ref{tab:tabla_f2_2a} - \ref{tab:tabla_f2_2b}.

- \textbf{Ejecución nº 3 GitHub}:

Se puede observar en los resultados como no aparecen repositorios que usen GitHub Actions, a causa de que los repositorios que se han intentado buscar en esta ejecución son repositorios cuya fecha de creación es anterior al origen de GitHub Actions, y que además, han sido dejados de modificar tras su aparición, es decir, a diferencia de la segunda ejecución de este experimento final sobre proyectos GitHub, no podría darse el caso de que hubiesen migrado de otra herramienta a GitHub Actions.

En la tabla de contadores \ref{tab:tabla_f3_1} queda reflejada la exclusividad de uso de Travis CI para la implementación de la integración continua en proyectos GitHub de código abierto anteriores a la aparición de GitHub Actions, con un total de 130 trabajos y una media de 1'65 tareas por trabajo.

Como se ve reflejado en la tabla \ref{tab:tabla_f3_4}, Javascript y Java son los dos lenguajes de programación más utilizados. Esto se debe a que en años anteriores, JavaScript y Java eran los dos lenguajes de programación por excelencia y se utilizaban para prácticamente todo.

En cuanto a los escenarios de ejecución se observa que el programa solamente ha encontrado trabajos sin escenarios definidos en etiquetas dedicadas exclusivamente para ello, como por ejemplo las etiquetas ``stage'', ``stages'', la etiqueta ``when'' en el caso de GitLab CI o la etiqueta ``on'' en el caso de GitHub Actions. Al no encontrar escenarios propiamente dichos, la tabla \ref{tab:tabla_f3_5} se ha rellenado con etiquetas del tipo ``script'', ``before\_install'', ``before\_script'' o ``install''.

Además se puede observar como el número total de repositorios y el total de escenarios de cada registro ``stage'' localizado coinciden, dando a entender como, por lo general, la estructura de los ficheros de configuración de Travis CI en años pasados contenían una etiqueta de cada tipo, pudiendo ser estas etiquetas las siguientes: ``before\_install'', ``install'', ``after\_install'', ``before\_script'', ``script'', ``after\_script'', ``before\_deploy'', ``deploy'', ``after\_deploy'', ``before\_cache'', ``cache'', ``after\_cache'', ``after\_success'' y/o ``after\_failure''.

\section{Resumen de resultados}
En esta sección, a modo de resumen del análisis llevado a cabo en los dos experimentos ejecutados, se va a proceder a contestar aquellas preguntas que se expusieron al inicio de este estudio:

- \textbf{¿Cuáles son los sistemas de CI más utilizados en cada plataforma?}

Se observa como en cada plataforma predominan sus propios servicios de integración continua. En el caso de GitHub el sistema GitHub Actions y en el caso de GitLab el sistema GitLab CI.

- \textbf{¿Se suelen combinar más de un sistema de CI en un mismo repositorio?}

Sí, se suelen combinar más de un sistema de CI, para que cada uno de ellos lleve a cabo tareas diferentes o por que se esté migrando de un sistema de CI a otro y estén conviviendo juntos en algún determinado momento.

En el experimento preliminar, en el caso de GitHub encontramos 55 repositorios que utilizan más de un sistemas de CI, siendo un 11 \% de los 500 repositorios totales y un 17'74 \% de los 310 repositorios con al menos un sistema de CI en uso. En el caso de GitLab, encontramos 31 repositorios con más de un sistema de CI, siendo el 6'2 \% de los 500 repositorios totales y el 10'4 \% de los que utilizan sistemas de CI exclusivamente.

En cuanto al experimento final, en la primera ejecución encontramos 92 y 45 repositorios, en GitHub y GitLab respectivamente, que utilizan más de un sistema de CI, siendo el 16'2 \% de los 568 totales en GitLab y el 8'96 \% de los 502 totales en GitLab. En la segunda ejecución, ejecutada exclusivamente para repositorios almacenados en GitHub, encontramos 146 repositorios que utilizan más de un sistema de CI, siendo el 21'66 \% de los 674 repositorios totales. En la tercera y última ejecución, ejecutada también solamente para repositorios alojados en GitHub, se obtienen exclusivamente 79 repositorios que utilizan el servicio proporcionado por Travis CI, dando lugar a que sea el 0 \% los repositorios que utilizan más de un sistema de CI.

- \textbf{¿Con qué lenguajes se suelen utilizar más sistemas de CI?}

Los lenguajes que más aparecen en cuanto al uso de sistemas de integración continua en proyectos software son Python y JavaScript, independientemente de la plataforma en la que estén alojados los repositorios (GitHub o GitLab).

- \textbf{¿Cuántos ``jobs'' suelen haber en repositorios ``open source''?}

El número de trabajos que suele haber en proyectos ``open source'', viendo los resultados obtenidos se observa que depende del sistema de CI que los configura. En el experimento final, primera ejecución, se obtiene ejecutando el proceso para repositorios almacenados en GitHub un número medio de 2'75, 9'57 y 12'25 trabajos en Travis CI, GitHub Actions y GitLab CI respectivamente. Estos datos se corresponden con una media total de 8'19 trabajos en proyectos de código abierto. Para esta misma ejecución pero sobre repositorios almacenados en GitLab, se obtiene una media de 3'41, 4'70 y 9'03 trabajos en Travis CI, GitHub Actions y GitLab CI, siendo la media total de 5'71 trabajos por repositorio.

En la segunda ejecución, que solamente se ejecuta el proceso sobre repositorios almacenados en GitHub, se obtiene una media de trabajos por repositorio de 3'41 en Travis CI, 7'46 en GitHub Actions y 4'89 en GitLab CI, dando lugar a una media total de 5'25 trabajos por repositorios.

Por último, la tercera ejecución, en la que solamente se obtienen repositorios que utilizan Travis CI con una media de trabajos por repositorio de 1'65.

- \textbf{¿Cuáles son los principales escenarios o ``stages'' en los que se suelen ejecutar trabajos automatizados?}

En cuanto a los principales escenarios de ejecución de trabajos en proyectos ``open source'' se podrían destacar ``push'' y ''pull\_request'' en el caso de GitHub Actions, ``build'' y ''test'' en el caso de GitLab CI y la etiqueta ``script'' en el caso de Travis CI.

%En cuanto a los principales escenarios de ejecución de trabajos en proyectos ``open source'' se podrían destacar, en el caso de GitHub Actions, ``push'' y ''pull\_request'' con un total de 3037 y 2119 encontrados en 421 y 370 repositorios diferentes respectivamente en la primera ejecución del experimento final. En el caso de GitLab CI, ``build'' y ''test'' con un total de 696 y 1003 encontrados en 200 y 197 repositorios distintos en la primera ejecución del experimento final. Y por último, en el caso de Travis CI, la etiqueta ``script'' en el caso de Travis CI con un total de 43 encontrados en la tercera ejecución del experimento final.