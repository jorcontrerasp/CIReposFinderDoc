% Clase del documento
\documentclass[12pt,twoside,titlepage]{report}





%%%%%%%%%%%%%%%%%%%%%%% Paquetes %%%%%%%%%%%%%%%%%%%%%%%

\usepackage[a4paper,bindingoffset=3mm,bottom=35mm]{geometry}


% Usad \usepackage[dvips]{graphicx} o \usepackage[pdftex]{graphicx} (no ambos)
%\usepackage[dvips]{graphicx} %%% para LaTeX. Las figuras deben estar en formato eps

\usepackage[colorlinks=true,pdftex]{hyperref}   %%% Opcional. Para incluir marcadores y enlaces en el pdf
\usepackage[pdftex]{graphicx}  %%% para pdflatex. Las figuras pueden estar en pdf, jpg, svg y otros formatos


\usepackage[spanish]{babel}

%\usepackage[latin1]{inputenc} % Usad en WinEdt/MikTex
\usepackage[utf8]{inputenc} % Usad en overleaf

%\usepackage[T1]{fontenc}


% Algunos paquetes útiles

\usepackage{amsmath,amssymb}
\usepackage{hyperref}
\usepackage{xcolor}
\usepackage{afterpage}
\usepackage{paralist}
\usepackage{array}
\usepackage{enumerate}
\usepackage{paralist}
\usepackage{enumitem}
\usepackage{float}
\usepackage{setspace}
\usepackage{listings}
\usepackage{algorithm}
\usepackage{algorithmic}
\usepackage{fancyhdr}
\usepackage{rotating}
\usepackage{multirow}


% Otros paquetes

\usepackage{quotchap}
\usepackage{lipsum}
\usepackage[normalem]{ulem}
% % Default fixed font does not support bold face
% \DeclareFixedFont{\ttb}{T1}{txtt}{bx}{n}{8} % for bold
% \DeclareFixedFont{\ttm}{T1}{txtt}{m}{n}{8}  % for normal

% % Custom colors
% \usepackage{color}
% \definecolor{deepblue}{rgb}{0,0,0.5}
% \definecolor{deepred}{rgb}{0.6,0,0}
% \definecolor{deepgreen}{rgb}{0,0.5,0}

% % Python style for highlighting
% \newcommand\pythonstyle{\lstset{
% language=Python,
% basicstyle=\ttm,
% morekeywords={self},              % Add keywords here
% keywordstyle=\ttb\color{deepblue},
% emph={MyClass,__init__, getContents},          % Custom highlighting
% emphstyle=\ttb\color{deepred},    % Custom highlighting style
% stringstyle=\color{deepgreen},
% frame=tb,                         % Any extra options here
% showstringspaces=false
% }}


% % Python environment
% \lstnewenvironment{python}[1][]
% {
% \pythonstyle
% \lstset{#1}
% }
% {}

% % Python for external files
% \newcommand\pythonexternal[2][]{{
% \pythonstyle
% \lstinputlisting[#1]{#2}}}

% % Python for inline
% \newcommand\pythoninline[1]{{\pythonstyle\lstinline!#1!}}

\definecolor{codegreen}{rgb}{0,0.6,0}
\definecolor{codegray}{rgb}{0.5,0.5,0.5}
\definecolor{codepurple}{rgb}{0.58,0,0.82}
\definecolor{backcolour}{rgb}{0.95,0.95,0.92}
\definecolor{deepblue}{rgb}{0,0,0.5}
\definecolor{deepred}{rgb}{0.6,0,0}
\definecolor{deepgreen}{rgb}{0,0.5,0}

\lstdefinestyle{python}{
    backgroundcolor=\color{backcolour},   
    commentstyle=\color{codegreen},
    keywordstyle=\color{magenta},
    numberstyle=\tiny\color{codegray},
    stringstyle=\color{codepurple},
    basicstyle=\ttfamily\footnotesize,
    breakatwhitespace=false,         
    breaklines=true,                 
    captionpos=b,                    
    keepspaces=true,                 
    numbers=left,                    
    numbersep=5pt,                  
    showspaces=false,                
    showstringspaces=false,
    showtabs=false,                  
    tabsize=2,
    emph={getContents},
    emphstyle=\color{orange},
    morekeywords={self,None}
}

\lstset{style=python}

%%%%%%%%%%%%%%%%%%%%%%%%%%%%%%%%%%%%%%%%%%%%%%%%%%%%%%%%






%%%%%%%%%%%%%%%%%%%%%%% Definiciones básicas %%%%%%%%%%%%%%%%%%%%%%%

\newcommand{\nombreautor}{Jorge Contreras Padilla}
\newcommand{\nombretutor}{Francisco Gortázar Bellas\\Michel Maes Bermejo}
\newcommand{\titulotrabajo}{Estudio de sistemas de integración continua en proyectos GitHub y GitLab}
\newcommand{\escuela}{Escuela Técnica Superior\\de Ingeniería Informática}
\newcommand{\escuelalargo}{Escuela Técnica Superior de Ingeniería Informática}
\newcommand{\universidad}{Universidad Rey Juan Carlos}
\newcommand{\fecha}{01/02/2022}
\newcommand{\grado}{Grado en Ingeniería de Computadores}
\newcommand{\curso}{Curso 2021-2022}
\newcommand{\logoUniversidad}{logoURJC.pdf} % logoURJC.eps

%%%%%%%%%%%%%%%%%%%%%%%%%%%%%%%%%%%%%%%%%%%%%%%%%%%%%%%%%%%%%%%%%%%%

\newcommand{\GitHubTokenA}{imgs/github_token1.PNG}
\newcommand{\GitHubTokenB}{imgs/github_token2.PNG}
\newcommand{\GitHubTokenC}{imgs/github_token3.PNG}
\newcommand{\GitHubTokenD}{imgs/github_token4.PNG}
\newcommand{\GitHubTokenE}{imgs/github_token5.PNG}
\newcommand{\GitLabTokenA}{imgs/gitlab_token1.PNG}
\newcommand{\GitLabTokenB}{imgs/gitlab_token2.PNG}
\newcommand{\GitLabTokenC}{imgs/gitlab_token3.PNG}
\newcommand{\GitLabTokenD}{imgs/gitlab_token4.PNG}
%%% Diagrams %%%
\newcommand{\flujoEjecucion}{imgs/diagrams/process_diagram.PNG}
\newcommand{\flujoGitHub}{imgs/diagrams/github_diagram.PNG}
\newcommand{\flujoGitLab}{imgs/diagrams/gitlab_diagram.PNG}
\newcommand{\ciDiagram}{imgs/diagrams/ci_diagram.JPG}
\newcommand{\githubActionsDiagram}{imgs/diagrams/github_actions_diagram.PNG}
\newcommand{\toolDiagram}{imgs/diagrams/tool_diagram.PNG}
%%% Logos %%%
\newcommand{\logoGitLab}{imgs/logos/gitlab_logo.jpeg}
\newcommand{\logoGitHub}{imgs/logos/github_logo.PNG}
\newcommand{\logoVscode}{imgs/logos/vscode_logo.PNG}
\newcommand{\logoPython}{imgs/logos/python_logo.PNG}
\newcommand{\logoAzurePipelines}{imgs/logos/azure_pipelines_logo.PNG}
\newcommand{\logoBamboo}{imgs/logos/bamboo_logo.PNG}
\newcommand{\logoBazel}{imgs/logos/bazel_logo.PNG}
\newcommand{\logoCircleCI}{imgs/logos/circleci_logo.PNG}
\newcommand{\logoCodeship}{imgs/logos/codeship_logo.PNG}
\newcommand{\logoGitHubActions}{imgs/logos/github_actions_logo.PNG}
\newcommand{\logoGitLabCI}{imgs/logos/gitlabci_logo.PNG}
\newcommand{\logoJenkins}{imgs/logos/jenkins_logo.PNG}
\newcommand{\logoSemaphore}{imgs/logos/semaphore_logo.PNG}
\newcommand{\logoTeamCity}{imgs/logos/teamcity_logo.PNG}
\newcommand{\logoTravis}{imgs/logos/travis_logo.PNG}
%%% Graphs & Histograms %%%
\newcommand{\graphA}{imgs/graphs_histograms/ef1_1a.PNG}
\newcommand{\graphB}{imgs/graphs_histograms/ef1_1b.PNG}
\newcommand{\graphC}{imgs/graphs_histograms/ef2_1.PNG}
\newcommand{\graphD}{imgs/graphs_histograms/ef3_1.PNG}

%%%%%%%%%%%%%%%%%%%%%%%%% Otras definiciones %%%%%%%%%%%%%%%%%%%%%%%%%%

% Definiciones de colores (para hidelinks)
\definecolor{BlueLink}{rgb}{0.165,0.322,0.745}
\definecolor{PinkLink}{rgb}{0.8,0.22,0.5}
\definecolor{gray}{rgb}{0.6,0.6,0.6}


% Enlaces
\hypersetup{hidelinks,pageanchor=true,colorlinks,citecolor=PinkLink,urlcolor=black,linkcolor=BlueLink}


\newcommand\blankpage{%
    \newpage
    \null
    \thispagestyle{empty}%
    %\addtocounter{page}{-1}%
    \newpage}


% Texto referencias
\addto{\captionsspanish}{\renewcommand{\bibname}{Bibliografía}}

% Texto Índice de tablas
\addto\captionsspanish{
\def\tablename{Tabla}
\def\listtablename{\'{I}ndice de tablas}
}


\floatname{algorithm}{Algoritmo}

\newfloat{algorithm}{t}{lop}

%% Etiquetas de comentarios (tutor/alumno)
\newif\ifdraft
\drafttrue
\usepackage{subcaption}
\newcommand{\nb}[2]{
	{
		{\color{black}{
				\small\fbox{\bfseries\sffamily\scriptsize#1}
				{\sffamily\small$\triangleright~${\it\sffamily\small #2}$~\triangleleft$}
	}}}
}

\ifdraft
\newcommand\tutor[1]{\nb{Patxi}{\color{red}#1}}
\newcommand\alumno[1]{\nb{Jorge}{\color{blue}#1}}
\newcommand\cotutor[1]{\nb{Michel}{\color{violet}#1}}
\newcommand{\fixme}[1]{{\textcolor{red}{[FIXME] #1}}\xspace}
\newcommand{\cn}{{\color{violet}[citation required]}}

\else
%\usepackage[disable]{todonotes}
\newcommand\tutor[1]{}
\newcommand\alumno[1]{}
\newcommand\cotutor[1]{}
\newcommand{\fixme}[1]{}
\newcommand{\cn}{}

\fi

\newcommand{\add}[1]{
	\textcolor{teal}{#1}
}
\newcommand{\remove}[1]{
	\textcolor{red}{\sout{#1}}
}





%\newenvironment{pseudocodigo}[1][htb]
%  {\renewcommand{\algorithmcfname}{Pseudocódig}% Update algorithm name
%   \begin{algorithm}[#1]%
%  }{\end{algorithm}}
  
%%%%%%%%%%%%%%%%%%%%%%%%%%%%%%%%%%%%%%%%%%%%%%%%%%%%%%%%%%%%%%%%%%%%





%%%%%%%%%%%%%%%%%%%%%%% Estilo de código (en Python) %%%%%%%%%%%%%%%%%%%%%%%

\definecolor{bg}{rgb}{0.95,0.95,0.95}
\definecolor{mydeepteal}{rgb}{0.16,0.22,0.23}
\definecolor{myteal}{rgb}{0.31,0.44,0.46}
\definecolor{mymediumteal}{rgb}{0.41,0.58,0.60}

\DeclareFixedFont{\ttb}{T1}{txtt}{bx}{n}{12} % for bold
\DeclareFixedFont{\ttm}{T1}{txtt}{m}{n}{12}  % for normal


%\newcommand*{\FormatDigit}[1]{\textcolor{mydeepteal}{#1}}
\newcommand*{\FormatDigit}[1]{\textcolor{black}{#1}}

% Python style for highlighting
\newcommand\mypythonstyle{\lstset{
language=Python,
basicstyle=\ttfamily\small,
%basicstyle=\linespread{1.0}\footnotesize\ttm,
otherkeywords={self},             % Add keywords here
keywordstyle=\bfseries\ttfamily\color{myteal},
%keywordstyle=\ttb\color{myteal},
commentstyle=\itshape\color{myteal},
stringstyle=\color{mydeepteal},
emph={MyClass,__init__},          % Custom highlighting
emphstyle=\ttb\color{mydeepteal},    % Custom highlighting style
% Any extra options here
showstringspaces=false,            %
backgroundcolor=\color{bg},
rulecolor = \color{bg},
%identifierstyle=\color{deepgreen},
breaklines=true,
numbers=left,
numbersep=5pt,
numberstyle=\tiny,
tabsize=4,
xleftmargin=1em,
frame = single,
framesep = 3pt,
framextopmargin=0pt,
framexbottommargin=0pt,
framexleftmargin=0pt,
framexrightmargin=0pt,
fontadjust=true,
basewidth=0.55em, % compactness of code
upquote=true,
}}

% Python environment
\lstnewenvironment{mypython}[1][]
{
\mypythonstyle
\lstset{#1}
}
{}

\newcommand\mypythonstylenormalinline{\lstset{
language=Python,
basicstyle=\ttfamily\normalsize,
%basicstyle=\linespread{1.0}\footnotesize\ttm,
otherkeywords={self},            % Add keywords here
keywordstyle=\bfseries\ttfamily\color{myteal},
%keywordstyle=\ttb\color{myteal},
commentstyle=\itshape\color{mymediumteal},
stringstyle=\color{mydeepteal},
emph={MyClass,__init__},          % Custom highlighting
emphstyle=\ttb\color{mydeepteal},    % Custom highlighting style
% Any extra options here
showstringspaces=false,            %
backgroundcolor=\color{bg},
rulecolor = \color{bg},
%identifierstyle=\color{deepgreen},
breaklines=false,
numbers=left,
numbersep=5pt,
numberstyle=\tiny,
tabsize=4,
xleftmargin=0em,
frame = single,
framesep = 3pt,
framextopmargin=0pt,
framexbottommargin=0pt,
framexleftmargin=0pt,
framexrightmargin=0pt,
fontadjust=true,
%basewidth=0.55em, % compactness of code
upquote=true,
}}

\newcommand\mypythoninline[1]{{\mypythonstylenormalinline\lstinline!#1!}}

%%%%%%%%%%%%%%%%%%%%%%%%%%%%%%%%%%%%%%%%%%%%%%%%%%%%%%%%%%%%%%%%%%%%%%%%%%%%%%




%%%%%%%%%%%%%%%%%%%%%%%%%%%% Comandos definidos por el autor 

\newcommand{\transpuesta}{\mbox{\tiny $\mathsf{T}$}}








%%%%%%%%%%%%%%%%%%%%%%%%%%%%%%%%%%%%%%%%%%%%%%%%%%%%%%%%%%%%%%%%%%%%%%%
%                           Inicio del documento                       
%%%%%%%%%%%%%%%%%%%%%%%%%%%%%%%%%%%%%%%%%%%%%%%%%%%%%%%%%%%%%%%%%%%%%%%


\begin{document}

\pagestyle{plain}




%%%%%%%%%%%%%%%%%%%%%%%%%%%%%%%%%%%% Portada %%%%%%%%%%%%%%%%%%%%%%%%%%%%%%%%%%

%\pagenumbering{gobble}
%\pagenumbering{arabic}

% Universidad, Facultad
\begin{titlepage}
\selectlanguage{spanish}


% logo
\begin{center}
    \includegraphics[scale=0.7]{\logoUniversidad}
\end{center}

\bigskip

\begin{center}
\begin{LARGE}
\escuela \\
\end{LARGE}
\end{center}

\bigskip
\bigskip

% Grado
\begin{center}
\begin{large}
\textbf{\grado}\\
\end{large}
\end{center}

% Curso
\begin{center}
\begin{large}
\textbf{\curso}\\
\end{large}
\end{center}

\bigskip

\textbf{\begin{center}
\begin{large}
\textbf{Trabajo Fin de Grado}
\end{large}
\end{center}}

\bigskip
\bigskip
\bigskip

% Nombre del TFG
\begin{center}
\textbf{\begin{large}
\MakeUppercase{\titulotrabajo}\\
\end{large}}
\end{center}

% Nombre del autor
\vspace{\fill}
\begin{center}
\textbf{Autor: \nombreautor}\\ \smallskip
% Tutor
\textbf{Tutor: \nombretutor}\\
% Añadir segundo tutor si hubiera


\bigskip

% Fecha
%\textbf{\fecha}\\
\end{center}
\end{titlepage}


%%%%%%%%%%%%%%%%%%%%%%%% Opcional %%%%%%%%%%%%%%%%%%%%%%
%\blankpage

%\thispagestyle{empty}
%\begin{center}

% Nombre del trabajo
%\textbf{\begin{large}
%\MakeUppercase{\titulotrabajo}\\*
%\end{large}}
%\vspace*{0.2cm}
%\vspace{5cm}

% Nombre del autor y del tutor
%\large Autor: \nombreautor \\* \medskip
%\large Tutor: \nombretutor \\*

%\vfill

% Escuela, universidad y fecha
%\escuelalargo \\ \smallskip
%\universidad \\
%\vspace{1cm}
%\fecha \\

%\clearpage

%\end{center}
%%%%%%%%%%%%%%%%%%%%%%%%%%%%%%%%%%%%%%%%%%%%%%%%%%%%%%%%

\hypersetup{pageanchor=true}

\normalsize
\afterpage{\blankpage} % Se deben añadir página en blanco para que lo capítulos de la memoria o estas secciones introductorias empiecen en páginas impares

%%%%%%%%%%%%%%%%%%%%%%%%%%%%%%%%%%%%%%%%%%%%%%%%%%%%%%%%%%%%%%%%%%%%%%%%%%%%%%%





% Estilo de párrafo de los capítulos
\setlength{\parskip}{0.75em}
\renewcommand{\baselinestretch}{1.25}
% Interlineado simple
\spacing{1}

\pagenumbering{Roman}
\setcounter{page}{2}


%%%%%%%%%%%%%%%%%%%%%%%%% Agradecimientos o dedicatoria %%%%%%%%%%%%%%%%%%%%%%%%%%%

\chapter*{Agradecimientos}

Dedicado a todas aquellas personas que me han acompañado en este largo recorrido universitario.

\afterpage{\blankpage}

%%%%%%%%%%%%%%%%%%%%%%%%%%%%%%%%%%%%%%%%%%%%%%%%%%%%%%%%%%%%%%%%%%%%%%%%%%%%%%%%%%%






%%%%%%%%%%%%%%%%%%%%%%%%%%%%%%%%%%%% Resumen %%%%%%%%%%%%%%%%%%%%%%%%%%%%%%%%%%%%%%

\chapter*{Resumen}

En este proyecto se aborda la elaboración de un estudio sobre los sistemas de integración continua que podemos encontrar en repositorios de código abierto almacenados en las plataformas de GitHub y GitLab.
El principal objetivo de este estudio es obtener información relativa a estos sistemas de integración continua, como por ejemplo, conocer cuál es la herramienta predominante en cada plataforma, si se suelen combinar más de una herramienta de este tipo en un mismo repositorio, la finalidad de uso de cada herramienta, en qué momento se ejecutan los diferentes trabajos que están programados, etc.
Para ello se ha implementado un programa escrito en el lenguaje de programación Python encargado de devolver la información necesaria para sacar conclusiones acerca de todas estas incógnitas expuestas sobre la integración continua.
En cuanto a los resultados obtenidos, a pesar de la gran cantidad de sistemas de integración continua existentes, cabría destacar que en proyectos de código abierto es cada vez más habitual la utilización de herramientas de integración continua proporcionados por la propia plataforma en la que se encuentran almacenados, ya sean GitHub o GitLab en el caso de este trabajo.
\mbox{} \bigskip

\noindent \textbf{Palabras clave}:
\begin{compactitem}
    \item Python.
    \item Integración continua o CI.
    \item Escenario o ``stage''
    \item GitHub.
    \item GitLab.
    \item Build.
    \item $\ldots$
\end{compactitem}

\afterpage{\blankpage}

%%%%%%%%%%%%%%%%%%%%%%%%%%%%%%%%%%%%%%%%%%%%%%%%%%%%%%%%%%%%%%%%%%%%%%%%%%%%%%%%%%%





%%%%%%%%%%%%%%%%%%%%%%%%%%%%%%%%%%%% Índices %%%%%%%%%%%%%%%%%%%%%%%%%%%%%%%%%%%%

% Estilo de párrafo de los Índices
\setlength{\parskip}{1pt}
\renewcommand{\baselinestretch}{1}
\renewcommand{\contentsname}{Índice de contenidos}


% Índice de contenidos
\tableofcontents
\afterpage{\blankpage}

% Índice de tablas (OPCIONAL)
\listoftables
\afterpage{\blankpage}
\addcontentsline{toc}{chapter}{\noindent \listtablename}

% Índice de figuras (OPCIONAL)
\listoffigures
\afterpage{\blankpage}
\addcontentsline{toc}{chapter}{\listfigurename}

% Índice de códigos/algoritmos (OPCIONAL).   El término "Códigos" se puede cambiar por "Métodos", "Funciones", "Algoritmos", etc.
\renewcommand\lstlistlistingname{Códigos}
\renewcommand\lstlistingname{Código}
\renewcommand\lstlistlistingname{Índice de códigos}

\lstlistoflistings
\afterpage{\blankpage}
\addcontentsline{toc}{chapter}{\lstlistlistingname}


% En este documento (de momento) no se ha considerado incluir un índice de algoritmos/pseudocódigos, como el que aparece en \ref{AdditionalLouvain}

%%%%%%%%%%%%%%%%%%%%%%%%%%%%%%%%%%%%%%%%%%%%%%%%%%%%%%%%%%%%%%%%%%%%%%%%%%%%%%%%%%%





%%%%%%%%%%%%%%%%%%%%%%% Cabeceras y pies de página (Opcional) %%%%%%%%%%%%%%%%%%%%%%%

%\setlength{\headheight}{15.2pt}
\pagestyle{fancy}


\renewcommand{\chaptermark}[1]{\markboth{Capítulo \thechapter.\ #1}{}}

\pagestyle{fancy}
\fancyhf{}
\fancyhead[LO]{\leftmark}
\fancyhead[RO]{}
\fancyhead[RE]{\nouppercase\rightmark}
\fancyhead[LE]{}
\fancyfoot[C]{\thepage}

%%%%%%%%%%%%%%%%%%%%%%%%%%%%%%%%%%%%%%%%%%%%%%%%%%%%%%%%%%%%%%%%%%%%%%%%%%%%%%%%%%%%






%%%%%%%%%%%%%%%%%%%%%%%%%%%%%% Capítulos de la memoria %%%%%%%%%%%%%%%%%%%%%%%%%%%%%



% Capítulo 1
\chapter{Introducción}


%%%%%%%%%%%%%%%%%%%%%%%%%%%%%%%%%%%%%%%%%%%%%%%%%%%%%%%%%%%%%%%%%%%%%%%%%%

% Estilo resto de páginas
\pagestyle{fancy}


% Estilo de párrafo de los capítulos
\setlength{\parskip}{0.75em}
\renewcommand{\baselinestretch}{1.25}
% Interlineado simple
\spacing{1}
% Numeración contenido
\pagenumbering{arabic}
\setcounter{page}{1}

%%%%%%%%%%%%%%%%%%%%%%%%%%%%%%%%%%%%%%%%%%%%%%%%%%%%%%%%%%%%%%%%%%%%%%%%%%

\section{Contexto y alcance}

La integración continua o CI es una práctica de desarrollo de software mediante la cual los desarrolladores combinan los cambios de código en un repositorio central de forma periódica, permitiendo la ejecución de versiones y la realización de pruebas automáticas sobre las mismas. Es decir, CI como proceso significa que cada cambio subido a un sistema de control de versiones ha sido puesto a prueba, validado y aceptado.

Anteriormente, era común que los desarrolladores de un equipo trabajasen aislados durante un largo periodo de tiempo y solo intentasen combinar los cambios en la versión final una vez completado el trabajo. Como consecuencia, la combinación de los cambios en el código resultaba ser una ardua tarea, dando lugar a que fuese más difícil proporcionar las actualizaciones a los clientes con rapidez.

Con la integración continua, los desarrolladores pueden enviar estos cambios de código de forma periódica a un repositorio compartido con un sistema de control de versiones como Git, y antes de cada envío, los desarrolladores pueden elegir ejecutar una serie de pruebas unitarias sobre el código como medida de verificación adicional antes de la integración.

Por lo tanto, los objetivos principales de la integración continua consisten en encontrar y arreglar errores con mayor rapidez, mejorar la calidad del software y reducir el tiempo que se tarda en validar y publicar nuevas actualizaciones del código fuente.

\section{Herramientas de integración continua}
Actualmente existen una gran cantidad de herramientas de integración continua, a comentar, por su grado de relevancia y uso, las siguientes:
\subsection{Jenkins}
Jenkins es un servidor de automatización de código abierto autónomo que se puede utilizar para automatizar todo tipo de tareas relacionadas con la creación, prueba y entrega o implementación de software.\\
Puede instalarse a través de paquetes del sistema nativo, Docker o incluso ejecutarse de forma independiente en cualquier máquina que tenga instalado Java Runtime Environment (JRE).

\begin{figure}[h]
    \centering
    \includegraphics[width=0.5\textwidth,clip=true]{\logoJenkins}
    \caption{Logo de Jenkins.}
\end{figure}

\subsection{Travis CI}
Travis CI, escrito en Ruby, es un servicio de integración continua que se utiliza para crear y probar proyectos de software alojados en GitHub, Bitbucket, GitLab y Assembla. Fue el primer servicio de CI que brindó servicios a proyectos de código abierto de forma gratuita y continúa haciéndolo.

Las principales funciones de Travis CI son:
\begin{compactitem}
    \item Configuración rápida.
    \item Vistas de construcción en vivo.
    \item Servicios de base de datos preinstalados.
    \item Implementaciones automáticas en compilaciones pasadas.
    \item Limpieza de VM para cada compilación.
    \item Compatibilidad con Mac, Linux e iOS.
\end{compactitem}

\begin{figure}[h]
    \centering
    \includegraphics[width=0.5\textwidth,clip=true]{\logoTravis}
    \caption{Logo de Travis.}
\end{figure}

\subsection{Circle CI}
¿Cómo funciona Circle CI?
\begin{compactitem}
    \item Integración VCS: se integra con GitHub y Bitbucket. Se crea una canalización cada vez que se envía código a cualquiera de las plataformas mencionadas.
    \item Pruebas automatizadas: ejecuta automáticamente su canalización en un contenedor limpio o en una máquina virtual, permitiendo probar cada confirmación.
    \item Notificaciones: se recibe una notificación si falla una canalización para poder solucionar errores rápidamente. Se pueden automatizar gracias a la integración con Slack.
    \item Despliegue automatizado.
\end{compactitem}

Algunas de las principales características de Circle CI son las siguientes:
\begin{compactitem}
    \item Flujos de trabajo para la orquestación de tareas.
    \item Soporte con Docker.
    \item Selección de CPU y RAM para adaptar las canalizaciones al equipo.
    \item Soporte independiente del idioma: se admite cualquier idioma que se desarrolle en Linux, Windows o macOS.
    \item Potente almacenamiento en caché.
    \item SSH o ejecución de compilaciones locales para una depuración sencilla.
    \item Seguridad: LDAP para administración de usuarios, aislamiento de máquinas virtuales a nivel completo y más.
    \item Panel de información: seguimiento del estado y optimización de canalizaciones con facilidad.
\end{compactitem}
Cuenta con dos opciones de hospedaje: en la nube o en servidor y con tres opciones de precios: “Free” 0 dólares al mes, “Performance” 30 dólares al mes y “Scale” con un precio a medida.

\begin{figure}[h]
    \centering
    \includegraphics[width=0.5\textwidth,clip=true]{\logoCircleCI}
    \caption{Logo de Circle CI.}
\end{figure}

\subsection{GitHub Actions}
Las acciones de GitHub ayudan a automatizar tareas dentro del ciclo de vida de un desarrollo de software. Están controladas por eventos, lo que significa que pueden ejecutar una serie de comandos después de que se haya producido un evento específico. Por ejemplo, cada vez que alguien crea una solicitud de extracción para un repositorio, puede ejecutar automáticamente un comando que ejecuta un script de prueba de software.

El siguiente diagrama muestra cómo se pueden usar las acciones de GitHub para ejecutar automáticamente scripts de prueba de software: un evento activa automáticamente un “flujo de trabajo”, que contiene un trabajo. Luego, el trabajo usa pasos para controlar el orden en el que se ejecutan las acciones. Estas acciones son los comandos que automatizan las pruebas de software. Además, hay múltiples componentes de GitHub Actions que trabajan juntos para ejecutar trabajos.

-DIAGRAMA-

\begin{figure}[h]
    \centering
    \includegraphics[width=0.5\textwidth,clip=true]{\logoGitHubActions}
    \caption{Logo de GitHub Actions.}
\end{figure}

\subsection{Azure Pipelines}
Azure Pipelines compila y prueba automáticamente proyectos de código para que estén disponibles para otros usuarios. Funciona con prácticamente cualquier tipo de proyecto o lenguaje.

Sus principales características son:
\begin{compactitem}
    \item Cualquier lenguaje, cualquier plataforma: permite compilar, probar e implementar aplicaciones de Node.js, Python, Java, PHP, Ruby, C/C++, .NET, iOS y Android. Además, permite ejecutar archivos en paralelo en Linux, macOS y Windows.
    \item Contenedores y Kubernetes: permite compilar e insertar fácilmente imágenes en registros de contenedor, como Docker Hub y Azure Container Registry. Además, permite implementar contenedores en hosts individuales o en Kubernetes.
    \item Extensible: ya que deja explorar e implementar una gran variedad de tareas de compilación, pruebas e implementación creadas por la comunidad, junto con cientos de extensiones, desde Slack hasta SonarCloud.
    \item Soluciones en cualquier nube: disponible la entrega continua (CD) del software en cualquier nube como Azure, AWS y GCP.
    \item Gratis para código abierto: asegura canalizaciones rápidas de integración y entrega continuas (CI/CD) para proyectos de código abierto.
    \item Características y flujos de trabajo avanzados: compatibilidad con YAML, integración de pruebas, validación de versiones, informes, etc.
\end{compactitem}

\begin{figure}[h]
    \centering
    \includegraphics[width=0.5\textwidth,clip=true]{\logoAzurePipelines}
    \caption{Logo de Azure Pipelines.}
\end{figure}

\subsection{Bamboo}
Atlassian Bamboo es un servidor de integración continua (CI) e implementación que ayuda a los equipos de desarrollo de software proporcionando:
\begin{compactitem}
    \item creación y prueba automatizadas del estado del código fuente del software.
    \item actualizaciones en compilaciones correctas y fallidas.
    \item herramientas de presentación de informes para el análisis estadístico.
\end{compactitem}

\begin{figure}[h]
    \centering
    \includegraphics[width=0.5\textwidth,clip=true]{\logoBamboo}
    \caption{Logo de Bamboo.}
\end{figure}

\subsection{GitLab CI}
GitLab CI/CD es la parte de GitLab que usa para todos los métodos continuos (Integración continua, Entrega e Implementación). Con GitLab CI/CD, puede probar, crear y publicar su software sin necesidad de una aplicación o integración de terceros.

\begin{figure}[h]
    \centering
    \includegraphics[width=0.2\textwidth,clip=true]{\logoGitLabCI}
    \caption{Logo de GitLab CI.}
\end{figure}

\subsection{Codeship}
CloudBees CodeShip es una solución de software como servicio (SaaS) que permite a los equipos de ingeniería implementar y optimizar CI y CD en la nube. Ayuda a los equipos pequeños y en crecimiento a desarrollar todo, desde aplicaciones web simples hasta arquitecturas de microservicios modernas para lograr una entrega de código rápida, segura y frecuente.

Codeship es una plataforma alojada de integración y entrega continua. Se encuentra entre tu repositorio de código fuente (por ejemplo, GitHub, GitLab o Bitbucket) y el entorno de alojamiento (por ejemplo, Amazon Web Services) y prueba e implementa automáticamente cada cambio en tu plataforma. Tu equipo de ingeniería puede enfocarse en desarrollar mejores aplicaciones en lugar de perder tiempo en mantener un servidor de CI engorroso. Codeship escala según tus necesidades, te permite acelerar las suites de prueba y les permite a tus desarrolladores.

\begin{figure}[h]
    \centering
    \includegraphics[width=0.5\textwidth,clip=true]{\logoCodeship}
    \caption{Logo de Codeship.}
\end{figure}

\subsection{TeamCity}
TeamCity es un servidor comercial de CI/CD que también está basado en Java (al igual que Jenkins). Es una herramienta de gestión y automatización de compilación creada por JetBrains.

El lema de TeamCity es “Potente integración continua lista para usar”, y esta herramienta lo justifica. Ofrece casi todas las funciones de Jenkins con algunas adicionales. TeamCity puede integrarse con Docker para crear contenedores automáticamente a través de docker-compose. Tiene soporte de integración para la herramienta Jira para rastrear problemas fácilmente.

TeamCity es compatible con .NET framework y puede integrar fácilmente TeamCity con varios IDEs como Eclipse, Visual Studio, etc. Con la integración para construir un repositorio de artefactos, TeamCity puede almacenar los artefactos en el sistema de archivos del servidor TeamCity o en el almacenamiento externo.

Con la versión gratuita de TeamCity de la licencia de servidor Professional, puede crear 100 compilaciones y 3 agentes de compilación sin costo alguno.

Cuenta con tres planes:
\begin{compactitem}
    \item TeamCity Cloud.
    \item TeamCity Professional.
    \item TeamCity Enterprise.
\end{compactitem}

\begin{figure}[h]
    \centering
    \includegraphics[width=0.5\textwidth,clip=true]{\logoTeamCity}
    \caption{Logo de TeamCity.}
\end{figure}

\subsection{Semaphore CI}
Semaphore es un servicio de automatización basado en la nube para crear, probar e implementar software.

Semaphore está diseñado para la productividad del desarrollador y se guía por tres principios:
\begin{compactitem}
    \item Velocidad: los desarrolladores deben trabajar en un ciclo de retroalimentación rápido, por lo que los flujos de trabajo de CI/CD deben ser rápidos.
    \item Potencia: la herramienta de CI/CD debe poder ejecutar cualquier flujo de trabajo de software automatizado, a cualquier escala.
    \item Facilidad de uso: CI/CD debe ser lo suficientemente fácil de usar para que todos los desarrolladores estén en estrecho contacto con el funcionamiento de su software y su impacto en los usuarios.
\end{compactitem}

Para el uso de Semaphore CI hay que tener en cuenta los siguientes prerrequisitos:
\begin{compactitem}
    \item Conocimiento básico sobre Git.
    \item Conocimiento básico sobre línea de comandos.
    \item Tener cuenta en Semaphore CI.
    \item Tener cuenta en GitHub.
\end{compactitem}

Cuenta con tres planes de pago:
\begin{compactitem}
    \item Free.
    \item Startup.
    \item Enterpirise Cloud.
\end{compactitem}

\begin{figure}[h]
    \centering
    \includegraphics[width=0.5\textwidth,clip=true]{\logoSemaphore}
    \caption{Logo de Semaphore CI.}
\end{figure}

\subsection{Bazel}
Bazel es otra herramienta de integración continua que ofrece las siguientes ventajas:
\begin{compactitem}
    \item Lenguaje de construcción de alto nivel. Bazel utiliza un lenguaje abstracto y legible por humanos para describir las propiedades de construcción de su proyecto en un alto nivel semántico. A diferencia de otras herramientas, Bazel opera con los conceptos de bibliotecas, binarios, scripts y conjuntos de datos, protegiéndolo de la complejidad de escribir llamadas individuales a herramientas como compiladores y enlazadores.
    \item Rápido y fiable. Bazel almacena en caché todo el trabajo realizado anteriormente y realiza un seguimiento de los cambios tanto en el contenido del archivo como en los comandos de compilación. De esta forma, Bazel sabe cuándo es necesario reconstruir algo y solo lo hace. Para acelerar aún más sus compilaciones, puede configurar su proyecto para que se construya de una manera altamente paralela e incremental.
    \item Multiplataforma. Bazel se ejecuta en Linux, macOS y Windows. Bazel puede crear binarios y paquetes implementables para múltiples plataformas, incluidas computadoras de escritorio, servidores y dispositivos móviles, desde el mismo proyecto.
    \item Escala. Bazel mantiene la agilidad mientras maneja compilaciones con más de 100 000 archivos fuente. Funciona con múltiples repositorios y bases de usuarios de decenas de miles.
    \item Extensible. Se admiten muchos idiomas y puede ampliar Bazel para admitir cualquier otro idioma o marco.
\end{compactitem}

\begin{figure}[h]
    \centering
    \includegraphics[width=0.5\textwidth,clip=true]{\logoBazel}
    \caption{Logo de Bazel CI.}
\end{figure}

Como se ha visto, a pesar de esistir una infinidad de herramientas de integración continua, todas ellas nos van a ofrecer en definitiva recursos muy parecidos para poder integrar nuestros proyectos.

Otras herramientas a mencionar que nos ofrecen características similares de integración continua son: GoCD, Shippable, Buildkite, Codefresh, Buddy, Buildbot, Wercker, Integrity, WeaveFlux, NeverCode, AutoRABIT, Bitrise, Drone CI, UrbanCode, Strider CD y FinalBuilder.

\section{Estructura del documento}

La estructura del TFG no es fija. El tutor indicará una estructura adecuada dependiendo del trabajo concreto.\tutor{Comentario del tutor}

Se puede incluir dentro de cada apartado secciones adicionales. La copia en papel de la memoria del TFG será encuadernada en pasta dura de color azul (p.e. encuadernación tipo chanel). La portada, que puede ser una pegatina transparente, seguirá el modelo que se adjunta, que incluye el escudo y nombre de la URJC, la titulación cursada por el alumno, el curso académico, el título del TFG, el autor y el o los directores/tutores.\alumno{Comentario del alumno}


\subsection{Trabajos de grados en informática}

Una posible estructura de la memoria final asociada con cada TFG podría ser la siguiente (leed la normativa de TFG):
\begin{enumerate}
 \item Introducción
 \item Objetivos (incluyendo descripción del problema, estudio de alternativas y metodología empleada)
 \item Descripción informática (puede incluir especificación, diseño, implementación y pruebas).
 \item Experimentos / validación
 \item Conclusiones (incluyendo los logros principales alcanzados y posibles trabajos futuros)
 \item Bibliografía
 \item Apéndices
\end{enumerate}


% \afterpage{\blankpage} % puede generar problema en índice de contenidos
% \newpage

\blankpage


% Capítulo 2
\chapter{Objetivos}

En este trabajo se pretende realizar un estudio sobre este tipo de herramientas sobre proyectos software alojados en diferentes plataformas en aras de tener un conocimiento más amplio sobre las diferencias entre los repositorios que las conforman.

Las plataformas de alojamiento de código que se van a utilizar para realizar este estudio son GitHub y GitLab.

GitHub es una compañía sin fines de lucro que ofrece un servicio de hosting de repositorios almacenados en la nube utilizando el sistema de control de versiones Git. 
Además, cuenta con una API REST disponible para cualquier desarrollador que quiera implementar alguna aplicación relacionada con el servicio que ofrece. 
En este caso, la aplicación programada para buscar proyectos que puedan tener pruebas end-toend utiliza el lenguaje de programación Python y, concretamente, la librería “PyGithub” que permite utilizar la versión 3 de la API ya mencionada.

Gitlab Inc. es una compañía de núcleo abierto y es la principal proveedora del software GitLab, un servicio web de control de versiones, DevOps y desarrollo de software colaborativo basado en Git. Además de un gestor de repositorios, el servicio ofrece también alojamiento de wikis y un sistema de seguimiento de errores, todo ello publicado bajo una licencia de código abierto, principalmente.
Fue escrito por los programadores ucranianos Dmitriy Zaporozhets y Valery Sizov en el lenguaje de programación Ruby con algunas partes reescritas posteriormente en Go, inicialmente como una solución de gestión de código fuente para colaborar con su equipo en el desarrollo de software. Luego evolucionó a una solución integrada que cubre el ciclo de vida del desarrollo de software, y luego a todo el ciclo de vida de DevOps. La arquitectura tecnológica actual incluye Go, Ruby on Rails y Vue.js.

Los principales objetivos del trabajo son los siguientes:
\begin{compactitem}
    \item Contruir un conjunto de datos con información sobre la integración continua.
    \item Aprender el funcionamiento de la integración continua en proyectos software.
    \item Estudiar la forma en la que se utilizan esas herramientas de integración continua en los repositorios.
    \item Expandir conocimientos en programación Python.
    \item Afianzar conocimientos sobre técnicas de research en GitHub.
    \item Aprendizaje en research en la plataforma GitLab.
\end{compactitem}

\section{Metodología}

Para la construcción de este conjunto de datos ya mencionado se va a utilizar una técnica conocida con el nombre de minería de repositorios.
Un repositorio software contiene una gran cantidad de información histórica y valiosa sobre el desarrollo general del sistema software que trata (estado, progreso y evolución del proyecto) y esta técnica de “research” se va a centrar en la extracción y análisis los datos heterogéneos disponibles en estos repositorio de software para descubrir información interesante, útil y procesable sobre el sistema.

En primer lugar, para obtener este conjunto de datos que nos permita analizar el funcionamiento de la integración continua en GitHub y GitLab, se van a enumerar diferentes herramientas de integración continua encontradas en repositorios seleccionados. 
Estos repositorios se elegiran en función de su repercusión en las plataformas, es decir, por su elevado número de estrellas o las bifurcaciones "forks" que tengan, ya que al tratarse de repositorios muy conocidos serán considerados como prometedores y que podrían utilizar herramientas de integración continuas para diferentes aspectos, objetivo de este estudio, como por ejemplo la automatización de tests.

Tras seleccionar los repositorios, se analizan manualmente en busca de herramientas de integración continua para ir enumerándolas.

Una vez rellena la lista de sistemas de integración continua a explorar tanto en GitHub como en GitLab, se va a estudiar el funcionamiento de cada una y la forma en la que se construyen los ficheros de configuración que utilizan para realizar la automatización de trabajos. De esta forma, por cada herramienta, se establece un criterio único de localización de repositorios contenedores de estos sistemas de automatización.

Con el heurístico de localización de repositorios prometedores ya construido, se realiza un conteo de estas herramientas sobre una búsqueda de 500 repositorios GitHub y 500 repositorios GitLab, haciendo un total de 1000 repositorios en este experimiento inicial.

A continuación, se analizan los resultados manualmente para verificar la efectividad del heurístico, es decir, comprobar que cuando el heurístico haya encontrado un repositorio que utilice una herramienta de integración continua concreta, efectivamente use esa herramienta de integración continua.

Con cada proyecto Github y GitLab se realiza lo siguiente:
\begin{compactitem}
    \item Buscar el sistema de CI que tiene el proyecto, buscando en su árbol de directorios los ficheros especificados en las carpetas que corresponda.
    \item Se buscan todos los ficheros en cada repositorio ya que un mismo proyecto podría tener más de un sistema de CI configurado.
    \item Anotar en qué sistemas de CI el proyecto ha dado positivo.
    \item Presentación en una tabla de los resultados: nº de proyectos que usan cada sistema de CI.
\end{compactitem}
	
Con este primer experimento realizado, se establecen cuáles son las herramientas de integración continua más populares con el objetivo de realizar un análisis más exhaustivo sobre dichas herramientas.

A continuación se realiza un segundo experimento sobre un número mayor de repositorio, obteniendo única y exclusivamente hasta 500 repositorios GitHub y 500 repositorios GitLab que hayan dado positivo en algun sistema de CI, descartando por completo aquellos en los que no se ha encontrado nada.

Sobre este segundo experimento, por cada proyecto se estudian diferentes aspectos:
\begin{compactitem}
    \item Número de trabajos.
    \item Número de tareas.
    \item Número medio de tareas por trabajo.
    \item Momento en el que se ejecutan los trabajos: push en master, pull request, en ramas, schedule...
\end{compactitem}
	
Por lo tanto, los pasos seguidos son los siguientes:
\begin{compactitem}
    \item Identificación de sistemas de CI que se van a estudiar.
    \item Análisis de los sistemas de CI identificados.
    \item Implementación de un heurístico localizador de repositorios GitHub y GitLab contenedores de herramientas de CI.
    \item Conteo de repositorios tanto positivos como negativos a los que se les ha aplicado el heurístico.
    \item Análisis manual de estos repositorios para verificar el heurístico.
    \item Selección de las herramientas de integración continua más populares en la plataforma.
    \item Aplicación del heurístico de tal forma que se obtengan 500 repositorios GitHub y 500 repositorios GitLab positivos.
    \item Análisis de los resultados.
\end{compactitem}

\section{Identificación de sistemas de CI}
En primer lugar localizamos los sistemas de CI que se van a estudiar. Para ello, se obtienen repositorios que sean conocidos en las plataformas GitHub y GitLab, filtrando por su número de estrellas y bifurcaciones realizadas por otros programadores de la comunidad sobre los mismos. Además, se añaden otras herramientas de integración continua que sean bastante conocidas en el mundo de la informática y la automatización de trabajos en proyectos software con el objetivo de realizar un análisis de herramientas de CI más amplio.
Por cada una de estas herramientas identificadas se realizan las siguientes tareas:
\begin{compactitem}
    \item Se estudia su funcionamiento.
    \item Se estudia la construcción del fichero de configuración utilizado para implementar la automatización.
    \item Se buscan ejemplos de uso.
\end{compactitem}

Los sistemas de CI identificados son los siguientes:
\begin{compactitem}
    \item Jenkins: búsqueda del fichero 'Jenkinsfile' ubicado en la raíz del proyecto.
    \item Travis: búsqueda de los ficheros '.travis-ci.yml' o '.travis.yml' ubicados en la raíz del proyecto.
    \item Circle CI: búsqueda del fichero '.circle-ci' situado en el directorio 'circleci'.
    \item GitHub Actions: búsqueda de ficheros YML o YAML situados en el directorio '.github/workflows'.
    \item Azure Pipelines: búsqueda del fichero 'azure-pipelines.yml' en la raíz del proyecto o el directorio '.azure-pipelines'.
    \item Bamboo: búsqueda del fichero YML o YAML 'bamboo' en el directorio 'bamboo-specs'.
    \item Concourse: búsqueda de ficheros 'ci/pipeline.yml' en algun directorio 'concourse'.
    \item GitLab CI: búsqueda del fichero '.gitlab-ci.yml' en la raíz del proyecto.
    \item Codeship: búsqueda, en la raíz del proyecto, de alguno de estos ficheros 'codeship-services.yml', 'codeship-steps.yml' o 'codeship-steps.json'.
    \item Team City: búsqueda del fichero 'settings.kts' en el directorio '.teamcity'
    \item Bazel: búsqueda de alguno de los ficheros 'presubmit.yml' 'XXXXX-binaries.yml' en el directorio '.bazelci' o el fichero '.bazelrc' en la raíz del proyecto.
    \item Semaphore CI: búsqueda del fichero 'semaphore.yml' en los directorios '.semaphore' o '.semaphoreci'.
    \item AppVeyor: búsqueda del fichero 'Appveyor.yml' en la raíz del proyecto.
\end{compactitem}

A continuación se enumeran algunos ejemplos de repositorios en los que se han ido encontrando estos sistemas de integración continua identificados:
-TABLA CON EJEMPLOS-
	
De esta forma, mediante un el lenguaje de programación Python, se implementa un programa encargado de ejecutar el heurístico localizador de repositorios con sistemas de CI.



\blankpage

% Capítulo 3
\chapter{Metodología}

Para la construcción de este conjunto de datos se va a utilizar una técnica de ``research'' conocida con el nombre de minería de repositorios.
Un repositorio software contiene una gran cantidad de información histórica y valiosa sobre el desarrollo general del sistema que trata (estado, progreso y evolución del proyecto) y esta técnica de ``research'' se va a centrar en la extracción y análisis de los datos heterogéneos disponibles en estos repositorios para descubrir información interesante, útil y procesable sobre el sistema.

En primer lugar, para obtener este conjunto de datos que nos permita analizar el funcionamiento de la integración continua en GitHub y GitLab, se van a enumerar diferentes herramientas de integración continua encontradas en repositorios seleccionados. 
Estos repositorios se elegirán en función de su repercusión en las plataformas, es decir, por su elevado número de estrellas o las bifurcaciones ``forks'' que tengan, ya que al tratarse de repositorios muy conocidos serán considerados como prometedores en cuanto a la utilización de herramientas de este tipo, objetivo de este estudio, para diferentes aspectos como por ejemplo la automatización de tests.

Tras seleccionar los repositorios, se analizan manualmente en busca de herramientas de integración continua para ir enumerándolas.

Una vez obtenida la lista de sistemas de integración continua a explorar, tanto en GitHub como en GitLab, se va a estudiar el funcionamiento de cada una de ellas y la forma en la que se construyen los ficheros de configuración que utilizan para realizar la automatización de trabajos. De esta forma, por cada uno de estos sistemas de CI, se establece un criterio único de localización de repositorios que emplean herramientas de integración continua y se conforma el heurístico que se va a utilizar para realizar el estudio.

Con el heurístico de localización de repositorios prometedores ya construido, se realiza un conteo de estas herramientas sobre una búsqueda de 500 repositorios GitHub y 500 repositorios GitLab, haciendo un total de 1000 repositorios en este experimento inicial.

A continuación, se analizan los resultados manualmente para verificar la efectividad del heurístico, es decir, comprobar que cuando el heurístico haya encontrado un repositorio que utilice una herramienta de integración continua concreta, efectivamente use esa herramienta de integración continua.

Por cada proyecto de Github o GitLab se realiza lo siguiente:
\begin{compactitem}
    \item Localizar el sistema de CI que tiene el proyecto, buscando en su árbol de directorios los ficheros especificados en las carpetas que corresponda (se buscan todos los ficheros en cada repositorio ya que un mismo proyecto podría tener más de un sistema de CI configurado).
    \item Anotar en qué sistemas de CI el proyecto ha dado positivo.
    \item Presentación en una tabla de los resultados: nº de proyectos que usan cada sistema de CI.
\end{compactitem}
	
Con este primer experimento realizado, se establecen cuáles son las herramientas de integración continua más populares con el objetivo de realizar un análisis más exhaustivo sobre ellas.

A continuación se realiza un segundo experimento sobre un número mayor de repositorio, obteniendo exclusivamente más de 500 repositorios en cada plataforma GitHub y  GitLab que hayan dado positivo en algún sistema de CI, descartando por completo aquellos en los que no se ha encontrado nada. Es decir, este conjunto final de datos va a quedar conformado por proyectos que utilizan herramientas de integración continua.

En este segundo experimento, por cada proyecto, se estudian diferentes aspectos que no se tuvieron en cuenta en el experimento inicial:
\begin{compactitem}
    \item Número de trabajos.\cotutor{Igual, mejor hablar de jobs}
    \item Número de tareas.
    \item Número medio de tareas por trabajo.
    \item Momento en el que se ejecutan los trabajos: push, pull request, en ramas, schedule...
    \item Lenguajes de programación predominantes en cada sistema de CI.
    \item Etc.
\end{compactitem}

Finalmente, una vez analizados todos los datos obtenidos, se pueden sacar conclusiones y contestar a todas las preguntas que se formularon inicialmente.
	
Por lo tanto, en cuanto a la metodología del trabajo, los pasos seguidos son los siguientes:
\begin{enumerate}
    \item Identificación de sistemas de CI que se van a estudiar.
    \item Análisis de los sistemas de CI identificados.
    \item Implementación de un heurístico localizador de repositorios GitHub y GitLab que utilicen herramientas de integración continua.
    \item Conteo de repositorios tanto positivos como negativos a los que se les ha aplicado el heurístico.
    \item Análisis manual de estos repositorios para verificar el heurístico.
    \item Selección de las herramientas de integración continua más populares en cada plataforma.
    \item Aplicación del heurístico de tal forma que se obtengan más de 500 repositorios GitHub y más de 500 repositorios GitLab positivos.
    \item Análisis de los resultados.
\end{enumerate}

\section{Identificación de sistemas de CI}

En primer lugar localizamos los sistemas de CI que se van a estudiar. Para ello, se obtienen repositorios que sean conocidos en las plataformas GitHub y GitLab, filtrando por su número de estrellas y bifurcaciones realizadas por otros programadores de la comunidad sobre los mismos. Además, se añaden otras herramientas de integración continua que sean bastante conocidas en el mundo de la informática y la automatización de trabajos en proyectos software con el objetivo de realizar un análisis de herramientas de CI más amplio.
Por cada una de estas herramientas identificadas se realizan las siguientes tareas:
\begin{enumerate}
    \item Se estudia su funcionamiento.
    \item Se estudia la construcción del fichero de configuración utilizado para implementar la automatización de trabajos.
    \item Se buscan ejemplos de uso.
\end{enumerate}

A continuación, se especifican los sistemas de CI identificados y que serán utilizados para conformar el heurístico de búsqueda de repositorios:
\begin{compactitem}
    \item \underline{Jenkins}: búsqueda del fichero ``Jenkinsfile'' ubicado en la raíz del proyecto \cite{jenkins}.
    \item \underline{Travis}: búsqueda de los ficheros ``.travis-ci.yml'' o ``.travis.yml'' ubicados en la raíz del proyecto \cite{travisCI}.
    \item \underline{Circle CI}: búsqueda del fichero ``.circle-ci'' situado en el directorio ``circleci'' \cite{circleCI}.
    \item \underline{GitHub Actions}: búsqueda de ficheros YML o YAML situados en el directorio ``.github/workflows'' \cite{githubActions}.
    \item \underline{Azure Pipelines}: búsqueda del fichero ``azure-pipelines.yml'' en la raíz del proyecto o el directorio ``.azure-pipelines'' \cite{azurePipelines}.
    \item \underline{Bamboo}: búsqueda del fichero YML o YAML ``bamboo'' en el directorio ``bamboo-specs'' \cite{bamboo}.
    \item \underline{Concourse}: búsqueda de la carpeta ``tasks'' o de algún fichero YML en algún directorio ``concourse'' \cite{concourse}.
    \item \underline{GitLab CI}: búsqueda del fichero ``.gitlab-ci.yml'' en la raíz del proyecto \cite{gitlabCI}.
    \item \underline{Codeship}: búsqueda, en la raíz del proyecto, de alguno de estos ficheros ``codeship-services.yml'', ``codeship-steps.yml'' o ``codeship-steps.json'' \cite{codeship}.
    \item \underline{TeamCity}: búsqueda del fichero ``settings.kts'' en el directorio ``.teamcity'' \cite{teamcity}.
    \item \underline{Bazel}: búsqueda de alguno de los ficheros ``presubmit.yml'' o ``build\_bazel\_binaries.yml'' en el directorio ``.bazelci'' o el fichero ``.bazelrc'' en la raíz del proyecto \cite{bazel}.
    \item \underline{Semaphore CI}: búsqueda del fichero ``semaphore.yml'' en los directorios ``.semaphore'' o ``.semaphoreci'' \cite{semaphoreCI}.
    \item \underline{AppVeyor}: búsqueda del fichero ``Appveyor.yml'' en la raíz del proyecto.
\end{compactitem}
	
De esta forma, mediante el lenguaje de programación Python, se implementa un programa encargado de ejecutar el heurístico localizador de repositorios que emplean sistemas de CI.


\blankpage

% Capítulo 4
\chapter{Descripción informática}

Para llevar a cabo este estudio se implementa una programa, mediante el lenguaje de programación Python, encargado de aplicar el heurístico desarrollado sobre los repositorios alojados tanto en GitHub como en GitLab.
Dicho programa se puede encontrar junto con su licencia Apache 2.0 en la siguiente url:\\
\url{https://github.com/jorcontrerasp/CIReposFinder}

\section{Herramientas utilizadas}
Las herramientas que se han utilizado para elaborar el programa son las siguientes:
\subsection{Python}
Es un lenguaje de programación cuya filosofía hace hincapié en la legibilidad del código. Sus principales características son las siguientes:
\begin{compactitem}
    \item Multiparadigma: ya que más que forzar a los programadores a adoptar un estilo de programación, permite varios estilos, soportando la orientación a objetos, la programación imperativa y la funcional.
    \item Interpretado.
    \item Dinámico: permitiendo que una variable pueda tomar valores de distinto tipo.
    \item Multiplataforma.
\end{compactitem}

\begin{figure}[h]
    \centering
    \includegraphics[width=0.3\textwidth,clip=true]{\logoPython}
    \caption{Logo de Python.}
\end{figure}

\subsection{API de GitHub}
API REST que nos va a permitir utilizar diferentes métodos para obtener información acerca de los repositorios almacenados en GitHub.

\begin{figure}[h]
    \centering
    \includegraphics[width=0.2\textwidth,clip=true]{\logoGitHub}
    \caption{Logo de GitHub.}
\end{figure}

\subsection{API de GitLab}
API REST que nos va a permitir utilizar diferentes métodos para obtener información acerca de los repositorios almacenados en GitLab.

\begin{figure}[h]
    \centering
    \includegraphics[width=0.5\textwidth,clip=true]{\logoGitLab}
    \caption{Logo de GitLab.}
\end{figure}

\subsection{Visual Studio Code}
Visual Studio Code es un editor de código fuente ligero disponible para Windows, macOS y Linux. Viene con soporte incorporado para JavaScript, TypeScript y Node.js y tiene un rico ecosistema de extensiones para otros lenguajes  (como C++, Java, Python, PHP o Go) y tiempos de ejecución (como .NET y Unity).
En este trabajo se va a utilizar tanto para la implementación del programa en Python encargado de obtener información sobre integración continua como para la escritura de la memoria final en LaTeX.

\begin{figure}[h]
    \centering
    \includegraphics[width=0.3\textwidth,clip=true]{\logoVscode}
    \caption{Logo de Visual Studio Code.}
\end{figure}

\section{Implementación Python}
\cotutor{Yo llamaría a la sección ``Implementación de la herramienta""}

\cotutor{Más que enumerar los ficheros que se han utilizado para la implementación de la herramienta, sería más útil y compacto una figura o diagrama que muestre como se ha implementado. Sobre esta figura ya puedes contar como funciona/se utiliza tu herramienta}

\subsection{Ficheros Python}
El programa localizador de proyectos que utilizan CI cuenta con los ficheros descritos a continuación:
\begin{compactitem}
    \item \textbf{aux\_functions.py}: script de python que contiene funciones auxiliares en las que se apoyará el programa principal.
    \item \textbf{ci\_tools.py}: script en el que se definen todas las herramientas de integración continua que se van a buscar en el proceso junto con la forma en la que se van a encontrar dichas herramientas, es decir, contiene las instrucciones que conforman el heurístico de búsqueda.
    \item \textbf{ci\_yml\_parser.py}: script que transforma las etiquetas de ficheros YML a objetos Python con el objetivo de que sean tratados  de forma trivial al devolver la información que contengan.
    \item \textbf{config.yml}: fichero en el que se definen todas las variables de configuración del proceso de búsqueda.
    \item \textbf{dataF\_functions.py}: script contenedor de todas las funciones relacionadas con la gestión de la estructura de datos utilizada ``DataFrame'' para generar la información de retorno.
    \item \textbf{github\_queryMaker.py}: script que permite construir consultas con el formato aceptado por la API de GitHub.
    \item \textbf{github\_search.py}: script en el que se realiza la búsqueda de repositorios GitHub.
    \item \textbf{github\_tests.py}: script utilizado para aplicar el heurístico sobre un repositorio GitHub concreto, es decir, permite realizar pruebas unitarias sobre repositorios GitHub.
    \item \textbf{gitlab\_search.py}: script en el que se realiza la búsqueda de repositorios GitLab.
    \item \textbf{gitlab\_tests.py}: script utilizado para aplicar el heurístico sobre un repositorio GitLab concreto, es decir, permite realizar pruebas unitarias sobre repositorios GitLab.
    \item \textbf{main.py}: script que se llaman a todos las funciones necesarias para realizar el proceso de búqueda de repositorios. Conforma el programa principal.
    \item \textbf{project\_cleaner.py}: script que va a permitir limpiar el proyectos de ficheros que ya no sirvan o temporales (ficheros excel de resultados, ficheros pickle y ficheros de log).
    \item \textbf{convertExcel2LaTeX.py}: script que va a permitir convertir las tablas de los ficheros excel de resultados en código fuente LaTeX con el objetivo de facilitar la escritura de las mismas en la memoria del trabajo.
\end{compactitem}

El proceso que lleva a cabo este programa Python se ha programado de tal forma que sea configurable mediante el fichero ``config.yml''. Este fichero de configuración cuenta con tres partes bien diferenciadas:

\begin{enumerate}
    \item etiqueta \textbf{process}, con variables genéricas relacionadas con la configuración del proceso.
    \begin{compactitem}
        \item \textbf{execute}: indica si se ejecuta o no el proceso en su totalidad.
        \item \textbf{doGithubSearch}: indica si se ejecuta o no la búsqueda sobre repositorios GitHub.
        \item \textbf{doGitlabSearch}: indica si se ejecuta o no la búsqueda sobre proyectos GitLab.
        \item \textbf{usePickleFile}: indica si se utiliza o no un fichero binario de Python con extensión ``.pickle'' para obtener los repositorios, que fueron recuperados previamente en una búsqueda anterior, a los que se les aplicará el heurístico.
        \item \textbf{useResultsExcelFile}: indica si se continúa rellenando un fichero de resultados excel ya existente.
        \item \textbf{tmpDirectory}: variable que almacena el directorio de ficheros temporales.
        \item \textbf{tmpFile}: variable que almacena el nombre raíz de los ficheros temporales que se vayan utilizando en el proceso.
    \end{compactitem}
    \item etiqueta \textbf{github}, con variables relacionadas con la configuración de la búsqueda GitHub.
    \begin{compactitem}
        \item \textbf{queryFile}: indica qué consulta es la que se va a utilizar en la búsqueda sobre repositorios GitHub.
        \item \textbf{filterCommits}: indica si se filtra o no por el número de ``commits'' que tienen los repositorios.
        \item \textbf{MAX\_COMMITS}: en el caso de filtrar por ``commits'', indica el número máximo que tendrán los repositorios.
        \item \textbf{MIN\_COMMITS}: en el caso de filtrar por ``commits'', indica el número mínimo que tendrán los repositorios.
        \item \textbf{randomizeRepos}: indica si se obtienen ``n'' repositorios aleatorios de la lista inicial obtenida para aplicar el heurístico.
        \item \textbf{N\_RANDOM}: indica, en el caso de que se marque la opción ``randomizeRepos'', el número de repositorios aleatorios que se obtendrán.
        \item \textbf{onlyPositives}: indica si se devuelven única y exclusivamente repositorios positivos en los ficheros de resultados.
    \end{compactitem}
    \item etiquerta \textbf{gitlab}, con variables relacionadas con la configuración de la búsqueda GitLab.
    \begin{compactitem}
        \item \textbf{search1By1}: indica si se aplicará el heurístico a cada proyecto según se vayan encontrando o si, por el contrario, se obtienen todos los proyectos y a continuación, una vez obtenidos, se recorren para aplicarles el heurístico. Si buscamos solo positivos, ``search1By1'' será verdadero en cualquier caso (aspectopor programa), sin tener en cuenta el valor que venga en el fichero de configuración.
        \item \textbf{N\_ERROR\_PAGE\_ATTEMPTS}: indica el número máximo de intentos a ejecutar en caso de que la llamada a la API de GitLab falle.
        \item \textbf{LANGUAGE}: indica el lenguaje en el que tendrán que estar implementados los proyectos GitLab analizados.
        \item \textbf{N\_MAX\_SEARCHES}: indica el número máximo de búsquedas que se van a realizar.
        \item \textbf{N\_MIN\_STARS}: indica el número mínimo de estrellas que tendrán que tener los proyectos GitLab analizados.
        \item \textbf{onlyPositives}: indica si se devuelven única y exclusivamente proyectos positivos en los ficheros de resultados.
        \item \textbf{N\_MAX\_PROJECTS}: si solo buscamos positivos, N\_MAX\_PROJECTS será el nº máximo de positivos a encontrar. Si buscamos tanto positivos como negativos, N\_MAX\_PROJECTS será el número máximo de projectos que tratará (sean o no positivos en CI).
    \end{compactitem}
\end{enumerate}

\subsection{Librerías utilizadas}
Las librerías utilizadas para la implementación del programa son las siguientes:
\begin{compactitem}
    \item \textbf{PyGithub}: librería que facilita el uso de la API de GitHub v3. Permite la gestión de diferentes recursos de GitHub (repositorios, perfiles de usuario, organizaciones, etc.) desde scripts Python.
    \item \textbf{Python-Gitlab}: librería que facilita el uso de la API v4 de GitLab y proporciona una herramienta CLI (Command Line Interface). Permite la gestión de diferentes recursos sobre proyectos almacenados en GitLab desde scripts Python.
    \item \textbf{Pandas}: iniciada en 2008, es una librería que pretende ser el bloque de construcción fundamental de alto nivel para realizar análisis de datos prácticos del mundo real en Python. Además, tiene el objetivo más amplio de convertirse en la herramienta de análisis/manipulación de datos de código abierto más potente y flexible disponible en cualquier idioma. Nos va a permitir manejar DataFrames y convertir la información que queramos tanto en formato excel como en formato csv para su posterior estudio.
    \item \textbf{Pickle}: importando esta librería en el proyecto vamos a poder almacenar la información que queramos en un fichero binario de Python. En este caso, una vez obtenidos los repositorios a analizar se almacenarán en un fichero de este tipo para poder reutilizar esos repositorios en posteriores ejecuciones.
    \item \textbf{Base64}: este módulo proporciona funciones para codificar datos binarios en caracteres ASCII imprimibles y decodificar dichas codificaciones en datos binarios.
    \item \textbf{PyYaml}: PyYAML es un marco YAML con todas las funciones para el lenguaje de programación Python. Nos va a permitir transformar ficheros con extensión YAML o YML en diccionarios Python en formato json.
    \item \textbf{Shutil}: librería que, mediante la instrucción ``rmtree'' va a permitir borrar carpetas de ficheros temporales utilizados para analizar cada repositorio GitHub/GitLab.
    \item \textbf{Json}: librería que va a permitir transformar textos en formato json en un objeto diccionario de Python.
\end{compactitem}

\subsection{Ficheros resultantes}
Tras cada ejecución del proceso, se van a generar varios ficheros en local con información relevante sobre estos sistemas de integración continua para su posterior análisis. Estos ficheros son:
\begin{compactitem}
    \item \textbf{github/github\_results.xlsx}: Fichero excel que va a componer una matriz repositorio de GitHub/herramienta de CI indicando en su intersección mediante una ``***'' si es o no positivo.
    \item \textbf{github/github\_languages.xlsx}: Fichero excel que va a componer una matriz lenguaje (de los repositorios GitHub encontrados)/herramienta de CI indicando el número de repositorios positivos en el lenguaje X y la herramienta de CI Y.
    \item \textbf{github/github\_ci\_statistics.xlsx}: Fichero excel en el que se van a recoger, por cada sistema de CI, datos estadísticos como el mínimo, el máximo, la media y la mediana relativos a la búsqueda que se haya realizado sobre repositorios GitHub.
    \item \textbf{github/github\_language\_statistics.xlsx}: Fichero excel en el que se van a recoger, por cada lenguaje de programación encontrado, datos estadísticos como el mínimo, el máximo, la media y la mediana relativos a la búsqueda que se haya realizado sobre repositorios GitHub.
    \item \textbf{github/github\_stage\_statistics.xlsx}: Fichero excel en el que se van a ir almacenando los contadores de los escenarios de ejecución o ``stages'' encontrados en el proceso de búsqueda GitHub.
    \item \textbf{repos\_github.pickle}: Fichero binario de Python en el que se van a almacenar los repositorios GitHub utilizados a la hora de aplicar el heurístico.
    \item \textbf{gitlab/gitlab\_results.xlsx}: Fichero excel que va a componer una matriz repositorio de GitLab/herramienta de CI indicando en su intersección mediante una ``***'' si es o no positivo.
    \item \textbf{gitlab/gitlab\_languages.xlsx}: Fichero excel que va a componer una matriz lenguaje (de los repositorios GitLab encontrados)/herramienta de CI indicando el número de repositorios positivos en el lenguaje X y la herramienta de CI Y.
    \item \textbf{gitlab/gitlab\_ci\_statistics.xlsx}: Fichero excel en el que se van a recoger, por cada sistema de CI, datos estadísticos como el mínimo, el máximo, la media y la mediana relativos a la búsqueda que se haya realizado sobre repositorios GitLab.
    \item \textbf{gitlab/gitlab\_language\_statistics.xlsx}: Fichero excel en el que se van a recoger, por cada lenguaje de programación encontrado, datos estadísticos como el mínimo, el máximo, la media y la mediana relativos a la búsqueda que se haya realizado sobre repositorios GitLab.
    \item \textbf{gitlab/gitlab\_stage\_statistics.xlsx}: Fichero excel en el que se van a ir almacenando los contadores de los escenarios de ejecución o ``stages'' encontrados en el proceso de búsqueda GitLab.
    \item \textbf{repos\_gitlab.pickle}: Fichero binario de Python en el que se van a almacenar los repositorios GitLab utilizados a la hora de aplicar el heurístico.
    \item \textbf{counting.xlsx}: Fichero excel con un conteo de los excel de resultados tanto en GitHub como en GitLab a modo de resumen.
\end{compactitem}

\section{Proceso de ejecución}
La implementación del programa está dividida en dos partes bien diferenciadas: una encargada de ejecutar el proceso \add{de minado} sobre repositorios GitHub y la otra sobre repositorios GitLab. En cada parte se utilizará para su cometido la API correspondiente a los repositorios a los que se les está aplicando el proceso de búsqueda.

En primer lugar se comprueba mediante la variable ``execute'' si se desea ejecutar o no el proceso en su totalidad. En el caso de que dicha variable sea afirmativa se iniciará el proceso generando las estructuras de datos necesarias que se irán completando con datos obtenidos por el proceso de búsqueda. 

Estas estructuras de datos son los ``DataFrame'' de resultados (matriz de proyectos y sistemas de CI), de lenguajes (matriz de lenguajes y sistemas de CI), de contadores y de estadísticas de datos obtenidos a partir de los ficheros de configuración de estos sistemas (estadísticas de sistemas de CI, lenguajes y escenarios de ejecución de trabajos), las cuales pueden ser generadas de cero o recuperadas a partir de ficheros excel ya generados con anterioridad en función de la variable de configuración ``useResultsExcelFile''.

A continuación, el programa prosigue ejecutando el proceso de búsqueda sobre GitHub y acto seguido sobre GitLab, siempre y cuando las variables ``doGitHubSearch'' y ``doGitLabSearch'' respectivamente sean positivas.

En cuanto a la parte de ejecución sobre repositorios de la plataforma GitHub, en primer lugar se carga la lista de repositorios que va a ser utilizada para aplicar el heurístico. Esta se puede recuperar mediante la carga desde un fichero binario de Python con extensión ``.pickle'' o se puede generar desde cero mediante una llamada a la API de GitHub, en función del valor de la variable de configuración ``usePickleFile''.

Una vez obtenida la lista de repositorios se aplica el heurístico a cada uno de ellos y se irán rellenando las estructuras de datos ya mencionadas con la información que se vaya obteniendo a medida que se ejecuta el proceso. Estas estructuras de datos se devuelven en formato excel en aras de poder ser interpretadas de forma sencilla.

En cuanto a la parte correspondiente a la ejecución sobre proyectos GitLab, al igual que en la parte GitHub ya mencionada, se pueden obtener los repositorios cargados desde un fichero binario de Python con extensión ``.pickle'' o se pueden recuperar mediante una llamada a la API de GitLab. En el caso de que se tenga que llamar a la API de GitLab para generar los repositorios, se diferencian dos formas de actuar en función de la variable de configuración ``search1By1'': obteniendo los repositorios y aplicar el heurístico a la lista que se genere, o ir aplicando el heurístico uno a uno según se vaya cargando la lista de repositorios.

Finalmente, se transforman las estructuras de datos de tipo ``DataFrame'' proporcionadas por la librería ``Pandas'' en formato Excel o Csv.

A modo de resumen, se expone a continuación el flujo de ejecución del programa:

\begin{figure}[h]
    \centering
    \includegraphics[width=1\textwidth,clip=true]{\flujoGitHub}
    \caption{Flujo de ejecución GitHub.}
\end{figure}

\begin{figure}[h]
    \centering
    \includegraphics[width=1.25\textwidth,clip=true]{\flujoGitLab}
    \caption{Flujo de ejecución GitLab.}
\end{figure}

\section{Dificultades y problemas encontrados}
En la realización de este trabajo se han ido encontrando diferentes errores que han supuesto un impedimento a la hora de continuar. 

Los más comunes han sido los siguientes:

- Resulta que la API de GitHub nos permite crear llamadas para obtener información sobre los diferentes repositorios que contiene, y de esta manera poder integrar nuestra aplicación con GitHub, pero con una limitación de 60 peticiones a la hora, número insuficiente para realizar una búsqueda masiva que devuelva una cantidad aceptable de repositorios con la que poder llevar a cabo el estudio de manera objetiva. En el momento en el que se utilizan más de los que esta API puede tratar se lanza la siguiente excepción: 

\alumno{No se ve bien la excepción. Revisar.}
\textit{github.GithubException.RateLimitExceededException: 403 \{``message'': ``API rate limit exceeded for user ID 77851630.'', ``documentation\_url'': ``https://docs.github.com/rest/overview/resources-in-the-rest-api\#rate-limiting''\}}

Generando un token de autenticación podemos incrementar el número de peticiones por hora de 60 a 5000 [AÑADIR ANEXO DE GENERACIÓN DE TOKEN GITHUB]. Sin embargo, este número incrementado de peticiones sigue siendo insuficiente para poder ejecutar el proceso de forma masiva sobre un número muy grande de repositorios.

Como solución, mediante la función ``get-rate-limit()'' \ref{lst:control_github_api} de la API de GitHub, antes de gastar una consulta a esta API, se obtiene el número de peticiones restantes y al llegar al límite se pausa el proceso tanto tiempo como sea necesario para recuperar dichas peticiones. Una vez recuperadas las peticiones se prosigue con la ejecución desde donde se dejó en espera.

\begin{lstlisting}[language=Python, caption=Control del nº de peticiones a la API GitHub, label={lst:control_github_api}]
def doApiRateLimitControl():
    try:
        g = authenticate()
        rl = g.get_rate_limit()
        rl_core = rl.core
        core_remaining = rl_core.remaining
        rl_search = rl.search
        search_remaining = rl_search.remaining
        if core_remaining <= 0:
            reset_timestamp = calendar.timegm(rl_core.reset.timetuple())
            sleep_time = reset_timestamp - calendar.timegm(time.gmtime()) + 5
            print("API rate limit exceded: " + str(sleep_time) + " sleep_time. Waiting...")
            time.sleep(sleep_time)
            g = authenticate()
    except:
        aux.printLog("Error al aplicar el control del API rate limit exceded...", logging.ERROR)
\end{lstlisting}

- La API de GitLab, al igual que la API de GitHub, cuenta con limitaciones a la hora de realizar llamadas sobre ella. En este caso no se limitan el número de peticiones sobre la API, sino que se limitan el número de proyectos que se pueden obtener.

Cada consulta a la API de GitLab cuenta con un número de páginas y en cada página un número X de proyectos, siendo X de 20 a 100 proyectos en función de lo que se configure en la variable ``per\_page'' de la consulta. 

Para tratar los repositorios devueltos por una consulta se pueden ir recorriendo las páginas que la conforman y por cada página se pueden ir tratando los repositorios contenidos. Sin embargo, al alcanzar al repositorio número 20.001 del recorrido se lanza la excepción correspondiente a la limitación de GitLab.

\alumno{No se ve bien la excepción. Revisar.}
\textit{gitlab.exceptions.GitlabHttpError: 405: Offset pagination has a maximum allowed offset of 50000 for requests that return objects of type Project. Remaining records can be retrieved using keyset pagination.}

La solución empleada para corregir este problema es la siguiente: en lugar de realizar una única consulta e ir recorriendo las páginas y repositorios de esa única consulta, se generan varias consultas en serie, siempre recorriendo el número máximo de repositorios devueltos en la primera página. Una vez ya han sido los repositorios correspondientes a la primera página de la consulta lanzada, se lanza otra consulta distinta partiendo del identificador del último repositorio tratado en la consulta anterior. De esta forma se puede esquivar la limitación de 20.000 repositorios por consulta impuesta por la API de GitLab e ir obteniendo infinitos proyectos.

- La librería PyYaml permite transformar un fichero con formato YML en un diccionario Python en formato json. Tras ejecutar el proceso de análisis de repositorios de forma masiva se han encontrado casos en los que el fichero YML de configuración del sistema de CI empleado para automatizar trabajos no estaba bien construido, provocando que esta librería no pudiese realizar la conversión de forma satisfactoria, lanzando excepciones similares a la que se muestra a continuación:

\alumno{No se ve bien la excepción. Revisar.}
\textit{while scanning a quoted scalar in ``tmp/ftmp\_0.yml'', line 21, column 27 found unexpected end of stream in ``tmp/ftmp\_0.yml'', line 22, column 1}

- Otro problema relacionados con la API de GitLab es la escasez de filtros que se pueden aplicar sobre cada consulta. A diferencia de la API de GitHub, la de GitLab no permite filtrar por lenguaje o número de estrellas dando lugar a que se tengan que ir comprobando uno a uno mediante los atributos ``language'' y ``stars'' de los objetos ``project''. Esto ha supuesto que el proceso de ejecución sobre proyectos GitLab tarde más en ejecutarse que la ejecución sobre repositorios GitHub.

\blankpage

% Nuevo capítulo
% \chapter{Contenidos principales}
% \label{chap:contenidos}

% \section{Primera sección}

% Citar una referencia
Esto es una referencia bibliográfica \cite{bibex}. Se recomienda leer ``The Not So Short Introduction to \LaTeX'' \cite{Oetiker2007} (existen versiones más modernas).


\subsection{Ecuaciones y fórmulas}

Gracias a la ecuación de Euler ($e^{ \pm i\theta } = \cos \theta \pm i\sin \theta$) podemos ver la relación entre varias de las constantes matemáticas más importantes:
\[
    e^{i\pi} + 1 = 0.
\]


% Fórmula numerada
Si una ecuación se va a referenciar es necesario numerarla:
\begin{eqnarray}
\label{eq:schemeP}
 \Phi (k)=\dfrac{2}{|R(k)|(|R(k)|-1)} \underset{i,j \in R(k)}{\sum} a_{ij}.
\end{eqnarray}
Posteriormente se hace referencia a la ecuación a través de su etiqueta (label). Por ejemplo, la anterior ecuación \eqref{eq:schemeP}.


Ejemplo de referencia a tabla: \ref{tab:una_tabla}


Problema de optimización:
\begin{equation}\label{eq:LP1}
\begin{array}{cl}
  \displaystyle \begin{array}{c}\mathrm{minimizar} \\ \mathbf{t} \in \mathbb{R}^{n}, \  \mathbf{p} \in \mathbb{R}^{m} \end{array} & \hspace{-0.2cm} \begin{array}{c} \mathbf{1}^{\transpuesta}\mathbf{t} \\ \mbox{} \end{array}  \\
  & \vspace{-0.4cm} \\ % línea (fila) en blanco, pero la hacemos estrecha con el comando vspace
  \mbox{sujeto a} & -\mathbf{t} \preceq  \mathbf{V}\mathbf{p} - \mathbf{x}  \preceq  \mathbf{t},\\
 \end{array}
\end{equation}




\subsection{Tablas y figuras}

% Insertar una tabla
\begin{table}
  \centering
  \caption{Título de la tabla.}
  \label{tab:una_tabla}

\begin{footnotesize}
\renewcommand{\arraystretch}{1.5} % Para cambiar la separación entre filas (1 por defecto)
\begin{tabular}{ccccccccccc}
  \hline
   & Subs. & Students & A & PE & WA & RE & CTE & IF & TLE & All\\
  \hline
Ex. 1 & 104 & 44 & 1.27    &   0       &   0.55    &   0.23    &   0.20    &   0.11    &   0     & 2.36  \\
Ex. 2 & 118 & 37 & 0.92    &   0       &   0.92    &   0.27    &   0.49    &   0.59    &   0     & 3.19  \\
Ex. 3 & 100 & 28 & 1.21    &   0.39    &   1.18    &   0.54    &   0.14    &   0.07    &   0.04  & 3.57  \\
Ex. 4 & 78  & 25 & 1.08    &   0.84    &   0.52    &   0.40    &   0.24    &   0.04    &   0     & 3.12  \\
Ex. 5 & 116 & 31 & 1.48    &   0.10    &   0.77    &   0.32    &   0.42    &   0.19    &   0.45  & 3.74  \\
Ex. 6 & 213 & 32 & 1.06    &   0.34    &   3.81    &   0.56    &   0.69    &   0.06    &   0.13  & 6.66  \\
Ex. 7 & 116 & 34 & 1.35    &   0.38    &   0.38    &   0.68    &   0.62    &   0       &   0     & 3.41  \\
  \hline
Average & 120.7 & 33 & 1.20 &  0.26 &  1.14 &  0.42 &  0.40 &  0.16 &  0.08 & 3.66 \\
  \hline
 \end{tabular}
\end{footnotesize}

\end{table}









\begin{sidewaystable}
  \centering
  \caption{Tabla rotada. Factor groupings for the Mooshak questionnaire.}\label{tab:factor_analysis}

\renewcommand{\arraystretch}{1.1}
\begin{scriptsize}
 \begin{tabular}{clcc}
   \hline
   Factor & \textbf{Interpretation} / Items$^{*}$ (loadings)  & Median & Mode \\
   \hline
   \hline
    1 & \multicolumn{3}{l}{\textbf{Students' perception of Mooshak towards its helpfulness in learning} } \\
   \hline
    (21.17\%) & m10. Mooshak has forced me to implement programs more carefully $(0.849)$ & 4 & 4 \\
    $\alpha$ = 0.922 & m6.  Mooshak has helped me improve as a programmer $(0.819)$ & 3 & 4 \\
     & m5.  Mooshak has made me more aware of the need to write correct code $(0.781)$ & 3 & 3\\
     & m1. Mooshak has forced me to program more responsibly $(0.713)$ & 3 & 3 \\
     & m15. The specifications regarding the exercises used with Mooshak are adequate $(0.687)$ & 3 & 3 \\
     & m18. Mooshak helps to measure my current programming skills $(0.680)$ & 2.5 & 3 \\
   \hline
%   \multicolumn{4}{c}{} \vspace{-0.2cm}\\
%   \hline
    2 & \multicolumn{3}{l}{\textbf{Disposition towards using Mooshak} } \\
   \hline
    (17.93\%) & m24. I would be willing to participate in a programming contest using Mooshak, with similar exercises to the ones & 2 & 1 \\
    $\alpha$ = 0.897 & seen throughout the course $(0.807)$ & & \\
    & m13. Using Moohak in the final exams is a good idea $(0.748)$ & 2 & 1 \\
    & m14. I would like to use Mooshak or a similar tool in the future $(0.734)$ & 3 & 1 \\
    & m17. Knowing Mooshak can motivate me to take part in a programming contest $(0.655)$ & 2 & 1\\
    & m9. It would have been useful to use Mooshak from the first programming course $(0.527)$ & 2.5 & 1\\
     & m16. Using Mooshak in the course has been interesting $(0.522)$ & 3 & 4 \\
   \hline
%   \multicolumn{4}{c}{} \vspace{-0.2cm}\\
%   \hline
    3 & \multicolumn{3}{l}{\textbf{Effect of Mooshak's feedback in the tool's usefulness} } \\
   \hline
    (14.84\%) & m12. Mooshak's feedback is adequate $(0.832)$ & 2 & 1\\
    $\alpha$ = 0.836 & m3. Using Mooshak has increased my workload considerably $(0.693)$ & 4 & 4 \\
     & m7.  If Mooshak does not accept my code I feel motivated to find and fix the errors $(0.691)$ & 2 & 3 \\
     & m8.  In general, using Mooshak has been a good idea $(0.666)$ & 3 & 4 \\
   \hline
%   \multicolumn{4}{c}{} \vspace{-0.2cm}\\
%   \hline
    4 & \multicolumn{3}{l}{\textbf{Mooshak's effect on persistence} } \\
   \hline
    (11.20\%) & m23. When Mooshak does not accept my code I get discouraged and I abandon the exercise $(0.848)$ & 3 & 3 \\
    $\alpha$ = 0.705 & m22. Mooshak has been a waste of time $(0.597)$ & 2 & 2 \\
    & m25. Once a program has passed Mooshak's tests, I rewrite it in order to enhance it $(0.559)$ & 2 & 2 \\
   \hline
%   \multicolumn{4}{c}{} \vspace{-0.2cm}\\
%   \hline
   5 & \multicolumn{3}{l}{\textbf{Students' perception of Mooshak's features} } \\
   \hline
    (10.87\%) & m20. Even if it is not related to the grade, I feel satisfied if I am one of the first students to complete an exercise $(0.729)$ & 2 & 2\\
   $\alpha$ = 0.742  & m19. I value the fact that a tool like Mooshak returns feedback in real time about the correction of my programs $(0.650)$ & 3.5 & 4 \\
   \hline
%   \multicolumn{4}{c}{} \vspace{-0.2cm}\\
%   \hline
\multicolumn{4}{l}{\scriptsize $^{*}$Measured on a 5-point Likert scale (1: strongly disagree; 2: disagree; 3: neutral; 4: agree; 5: strongly agree).}
  \end{tabular}
\end{scriptsize}
\end{sidewaystable}



\begin{table}
  \centering

\begin{small}
\begin{tabular}{|l|l|l|l|}\hline
  \multirow{10}{*}{numeric literals} & \multirow{5}{*}{integers} & in decimal & \verb|8743| \\ \cline{3-4}
  & & \multirow{2}{*}{in octal} & \verb|0o7464| \\ \cline{4-4}
  & & & \verb|0O103| \\ \cline{3-4}
  & & \multirow{2}{*}{in hexadecimal} & \verb|0x5A0FF| \\ \cline{4-4}
  & & & \verb|0xE0F2| \\ \cline{2-4}
  & \multirow{5}{*}{fractionals} & \multirow{5}{*}{in decimal} & \verb|140.58| \\ \cline{4-4}
  & & & \verb|8.04e7| \\ \cline{4-4}
  & & & \verb|0.347E+12| \\ \cline{4-4}
  & & & \verb|5.47E-12| \\ \cline{4-4}
  & & & \verb|47e22| \\ \cline{1-4}
  \multicolumn{3}{|l|}{\multirow{3}{*}{char literals}} & \verb|'H'| \\ \cline{4-4}
  \multicolumn{3}{|l|}{} & \verb|'\n'| \\ \cline{4-4}          %% here
  \multicolumn{3}{|l|}{} & \verb|'\x65'| \\ \cline{1-4}        %% here
  \multicolumn{3}{|l|}{\multirow{2}{*}{string literals}} & \verb|"bom dia"| \\ \cline{4-4}
  \multicolumn{3}{|l|}{} & \verb|"ouro preto\nmg"| \\ \cline{1-4}          %% here
\end{tabular}
\end{small}

  \caption{Tabla con ``multicolumnas'' y ``multifilas''.}\label{tab:tablacompleja}
\end{table}





% Insertar una figura
\begin{figure}
  \centering
  \includegraphics[width=0.75\textwidth,clip=true]{\logoUniversidad}
  \caption{Logo de la Universidad.}
  \label{fig:logo_universidad}
\end{figure}

% Referenciar una etiqueta (label)
Las tablas y figuras deben presentarse en el texto, referenciadas y numeradas. La descripción de una figura debe ir posicionada debajo de la misma. Las descripciones de tablas pueden aparecer encima o debajo de las mismas (pero de forma consistente en todo el documento).

En las tablas se recomienda evitar líneas verticales y usar pocas horizontales. 

La figura~\ref{fig:logo_universidad} se utiliza en la portada. \LaTeX ubica automáticamente las tablas y figuras. Para ello emplea reglas basadas en la experiencia de profesionales de la edición de textos. Podemos forzar su ubicación, pero en general es recomendable usar la ubicación sugerida por el sistema \LaTeX. Usad gráficos vectoriales siempre que podáis.





\begin{figure}
   \centering

  \begin{minipage}{0.45\textwidth}
   \centering

     \includegraphics[clip=true,width=\textwidth]{triangulo_grande_bb.pdf}\\

    \footnotesize (a)
  \end{minipage}
  \hfill
  \begin{minipage}{0.45\textwidth}
   \centering
     \includegraphics[clip=true,width=\textwidth]{triangulos_separados_bb.pdf}\\

   \footnotesize (b)
  \end{minipage}

    \bigskip

    \includegraphics[clip=true,width=0.5\textwidth]{triangulos_unidos_bb.pdf}\\
    \footnotesize (c)

  \caption{Ejemplo con varias figuras. Demostración visual del teorema de Pitágoras. En (a) tenemos un triángulo rectángulo con hipotenusa $c$ y catetos $a$ y $b$. En (b) se muestra tres copias escaladas del mismo triángulo. El verde se ha escalado por $a$, el rojo/rosa por $b$, y el azul por $c$. En (c) se juntan los triángulos de (b) para formar un rectángulo cuya base es $c^{2}$, pero también $a^{2} + b^{2}$. Por tanto, $a^{2} + b^{2} = c^{2}$.}\label{fig:teoremapitagoras}
\end{figure}





\section{Segunda sección}

% Nueva página
Normalmente no tendremos que insertar saltos de página, salvo para forzar que los capítulos empiecen en páginas impares, con \begin{verbatim}\blankpage\end{verbatim} En cualquier caso, podemos introducir un salto de página con el comando \begin{verbatim}\newpage\end{verbatim}.

\newpage
% También con \pagebreak



\subsection{Código}


\begin{mypython}[float={!t},caption={Titulo del algoritmo/código.},label={alg:etiqueta}]
def sum_list_limits_1(a, lower, upper):
    if lower > upper:
        return 0
    else:
        return a[upper] + sum_list_limits_1(a, lower, upper - 1)
\end{mypython}
El código~\ref{alg:etiqueta} es un ejemplo en Python.



\begin{algorithm}
\begin{algorithmic}[1]
\STATE $\forall i \in V$, \ let $i$ be an isolated community
\STATE $o=permutation(V)$
\FOR{$k \ \in \ o$}
\STATE search in $A$ all the neighbours of $k$, $j$
\STATE $\forall j$, calculate $\Delta Q_k(j)$ in matrix $\mathcal{M}$
\STATE $j^*=\{ \ j \ | \ \Delta Q_k(j^*)=\max_j\{Q_k(j)\} \ \}$
\IF{$\Delta Q_k(j^*)>0$}
\STATE{Move node $k$ to $j^*$ 's community}
\ELSE
\STATE{$k$ remains in its community}
\ENDIF
\ENDFOR
\end{algorithmic}\caption{\textit{Additional Louvain} \textbf{input}=$\left(A, \ \mathcal{M}\right)$ \textbf{output}=$P$}
\label{alg:AdditionalLouvain}
\end{algorithm}
En el algoritmo~\ref{alg:AdditionalLouvain} aparece un ejemplo en pseudocódigo.


% Capítulo 5
\chapter{Validación experimental y resultados}
\label{sec:resulObtenidos}

\section{Experimento preliminar}
Se realiza un primer experimento ejecutando el programa implementado sobre 500 repositorios de la plataforma GitHub y 500 repositorios de la plataforma GitLab, todos ellos repositorios de código abierto u ``open source'', sin discriminar entre positivos o negativos, es decir, entre esos 1000 repositorios se pretende encontrar tanto repositorios que utilizan alguna de las herramientas de integración continua contempladas como repositorios que no las utilizan.

El objetivo de este experimento es conocer cuáles son los sistemas de integración continua más utilizados en repositorios de este tipo para posteriormente analizarlas más en profundidad en un segundo experimento.

Los parámetros con los que se ejecuta el programa son los siguientes:
\begin{compactitem}
    \item Para \textbf{\underline{GitHub}}:
    \begin{compactitem}
        \item created: \textgreater2016-01-01
        \item pushed: \textgreater2021-01-01
        \item stars: \textgreater=9500
        \item forks: \textgreater=800
        \item archived: false
        \item is: public
        \item onlyPositives: false
    \end{compactitem}
    \item Para \textbf{\underline{GitLab}}:
    \begin{compactitem}
        \item visibility: public
        \item last\_activity\_after: 2016-01-01T00:00:00Z
        \item stars: \textgreater25
        \item onlyPositives: false
    \end{compactitem}
\end{compactitem}

Tras lanzar el programa se obtienen los resultados reflejados en la tabla \ref{tab:tabla_p1}.

A simple vista se puede observar que predomina el uso de los sistemas de integración continua propios de cada plataforma, siendo GitHub Actions para repositorios almacenados en GitHub y GitLab CI para repositorios almacenados en GitLab.

A parte de estos dos sistemas de CI ya mencionados se localizan repositorios que utilizan el sistema de integración continua proporcionado por Travis CI, obteniendo 68 casos en repositorios almacenados en GitHub y otros 25 en repositorios almacenados en GitLab, dando un total de 93 repositorios en ambas plataformas, es decir, en 9'3\% del total de repositorios.

También encontramos repositorios positivos en herramientas como Jenkins, que, a priori, se consideró como uno de los sistemas que iban a tener un mayor número de positivos junto con GitHub Actions.

Otros sistemas a tener en cuenta son Circle CI, Azure Pipelines y Bazel, obteniendo un total de 42, 6 y 8 repositorios positivos, resultados considerados insuficientes como para profundizar en ellos en el experimento final a realizar.

Finalmente, mencionar también los 2 repositorios positivos encontrados para la herramienta de integración continua Concourse, uno para GitHub y otro para GitLab, considerados como falsos positivos ya que a pesar de haber cumplido el criterio de búsqueda establecido no se utiliza esta herramienta de integración en ninguno de los dos casos. Esto se debe a que el criterio establecido para la búsqueda de este sistema de CI, ``tasks'', directorio propuesto por la documentación de Concourse para almacenar los ficheros de configuración de trabajos, es demasiado genérico como para encontrar repositorios que, teniendo ese directorio en su árbol de directorios, usen Concourse.

\section{Experimento final}
Con las conclusiones obtenidas tras la ejecución del primer experimento de este trabajo, se realiza un segundo experimento estudiando más en profundidad los sistemas de CI ofrecidos por Travis CI, GitHub Actions y GitLab CI, ya que son los sistemas de los que se han obtenido más información.

Para ello se modifica el programa de tal forma que se pueda analizar la forma en la que están construidos los ficheros YML o YAML de configuración de trabajos de las herramientas seleccionadas.

La principal mejora añadida para poder analizar dichos ficheros de configuración es la implementación de un script encargado de transformar cada fichero de configuración CI en objetos Python (ci\_yml\_parser.py) con el objetivo de poder manejar la información obtenida y construir los ficheros excel de resultados fácilmente.

Este script va a permitir añadir las siguientes columnas al excel principal de resultados:
\begin{compactitem}
    \item \textbf{STAGES}: columna en la que se almacenan los escenarios en los que se ejecutan los puntos del fichero de configuración considerados como trabajos.
    \item \textbf{NUM\_JOBS}: columna en la que se almacena el número de trabajos que se ejecutan en el repositorio en cuestión.
    \item \textbf{TOTAL\_TASKS}: columna en la que se almacena el número total de tareas de todos los trabajos automatizados en el repositorio.
    \item \textbf{TASK\_AVERAGE\_PER\_JOB}: columna en la que se almacena la media de tareas por trabajo.
\end{compactitem}

Con las que, posteriormente y de forma automática, se generan otros ficheros excel de estadísticas relacionadas con estos ficheros de configuración de trabajos. Estas columnas van contener información en formato json y a modo de diccionario con la intención de diferenciar los datos obtenidos en de cada sistema de CI en particular, ya que puede darse el caso de que algún repositorio en concreto haya dado positivo en más de una herramienta de integración continua, que, como se puede ver en los resultados obtenidos, es algo más común de lo que en un principio se podía esperar.

Este segundo experimento consiste en ejecutar el programa tres veces para repositorios GitHub y una vez para repositorios GitLab, encontrando en cada una de estas ejecuciones exclusivamente repositorios positivos en algún sistema de CI. 

Para ello se emplean las siguientes consultas de búsqueda:
\begin{compactitem}
    \item Para \textbf{\underline{GitHub}}:
    \begin{compactitem}
        \item \textbf{Ejecución nº 1}: se buscan repositorios GitHub públicos que sean relativamente recientes.
        \begin{compactitem}
            \item created: \textgreater2016-01-01
            \item pushed: \textgreater2021-01-01
            \item stars: \textgreater=9500
            \item forks: \textgreater=800
            \item archived: false
            \item is: public
            \item onlyPositives: true
        \end{compactitem}
        \item \textbf{Ejecución nº 2}: se buscan repositorios GitHub públicos que hayan sido creados antes de la aparición y auge del sistema de CI GitHub Actions, pero que se hayan seguido modificando recientemente. De esta forma se obtendrán repositorios que podrían haber migrado de algún sistema de CI distinto de las acciones de GitHub a el sistema proporcionado por GitHub, por ejemplo de Jenkins o Travis CI a GitHub Actions.
        \begin{compactitem}
            \item created: 2010-01-01..2016-01-01
            \item pushed: \textgreater2018-01-01
            \item stars: \textgreater=10500
            \item forks: \textgreater=1500
            \item archived: false
            \item is: public
            \item onlyPositives: true
        \end{compactitem}
        \item \textbf{Ejecución nº 3}: se buscan repositorios GitHub públicos que hayan sido creados antes de la aparición y auge del sistema de CI GitHub Actions y que se hayan dejado de modificar recientemente.
        \begin{compactitem}
            \item created: 2010-01-01..2016-01-01
            \item pushed: $<$2018-01-01
            \item stars: \textgreater=500
            \item forks: \textgreater=250
            \item archived: false
            \item is: public
            \item onlyPositives: true
        \end{compactitem}
    \end{compactitem}
    \item Para \textbf{\underline{GitLab}}: se buscan repositorios GitLab de forma genérica.
    \begin{compactitem}
        \item visibility: public
        \item last\_activity\_after: 2016-01-01T00:00:00Z
        \item stars: \textgreater25
        \item onlyPositives: true
    \end{compactitem}
\end{compactitem}

Una vez ejecutado el proceso partiendo de las consultas ya mencionadas, se obtienen los resultados comentados a continuación:

- \textbf{Ejecución nº 1 GitHub y GitLab}: 

A simple vista se puede observar en la tabla \ref{tab:tabla_f1_1}, al igual que en el experimento preliminar, el predominio de sistemas de CI propios de la plataforma en la que se ha ejecutado el programa, es decir, el predominio de GitHub Actions en repositorios almacenados en GitHub y de GitLab CI en repositorios almacenados en GitLab, debido probablemente por la correcta integración y la facilidad de uso de dichos sistemas sobre repositorios almacenados en sus propias plataformas, así como la utilidad de tener tanto el proyecto como la configuración de trabajos automatizados en un mismo sitio.

No obstante, también se encuentran otras herramientas de CI como Travis CI, Circle CI, Azure Pipelines, Bazel, etc.

Sobre el número de trabajos encontrados y la media de tareas por trabajo calculada en los sistemas de CI GitHub Actions, GitLab CI y Travis CI, datos reflejados en las tablas \ref{tab:tabla_f1_3} y \ref{tab:tabla_f1_7}, se sacan conclusiones similares a las ya obtenidas en el experimento preliminar. Sobre repositorios almacenados en GitHub se obtiene un número alto de trabajos en GitHub Actions con una media de 9'36 tareas por trabajo y sobre repositorios almacenados en GitLab, de forma inversamente proporcional a los trabajos encontrados en repositorios GitHub, se obtiene un número elevado de trabajos en GitLab CI, con una media de 9'15 tareas por trabajo. En cuanto a Travis CI se observa que se mantiene estable el número de trabajos, ya sea ejecutando el proceso sobre repositorios GitHub como sobre repositorios GitLab, con una media de entre 2 y 3 tareas por trabajo.

En cuanto a los lenguajes de programación utilizados para implementar los diferentes repositorios a los que se les ha aplicado el proceso, aparecen varios en común situados en la cúspide de los más utilizados, a tener en cuenta javascript, python, c++, go o java, ordenados de mayor a menor uso. Estos datos se pueden consultar en la tablas \ref{tab:tabla_f1_4} y \ref{tab:tabla_f1_8}.

Otro aspecto a tener en cuenta sobre los datos obtenidos en el proceso de búsqueda son los escenarios en los que se ejecutan los diferentes trabajos automatizados por estos sistemas de CI. En este caso, tal y como se expone en las tablas \ref{tab:tabla_f1_5} y \ref{tab:tabla_f1_9}, encontramos diferencias notables, ya que en repositorios almacenados en GitHub predominan escenarios como push, pull\_request o schedule y en GitLab predominan otros como build, test, deploy o release.

- \textbf{Ejecución nº 2 GitHub}:

En esta segunda ejecución sobre repositorios de la plataforma GitHub, a pesar de haber intentado localizar repositorios con herramientas de CI distintas de GitHub Actions buscando solamente aquellos que fueron creados antes de la existencia de dicha herramienta, si se consultan los datos de la tabla \ref{tab:tabla_f2_1} se puede ver como sigue predominando el uso de las acciones de la propia plataforma de GitHub, probablemente debido a que en algún momento de tiempo utilizaban otros sistemas de CI y se migraron recientemente para utilizar GitHub Actions. En trabajos futuros se podrían comprobar commits antiguos de estos repositorios para comprobar si esta suposición es o no correcta.

En cuanto a los datos que se obtienen relativos a los lenguajes de programación empleados en cada repositorio, trabajos automatizados, tareas y escenarios en los que son ejecutados dichos trabajos son muy similares a los obtenidos en la primera ejecución de este experimento final.

- \textbf{Ejecución nº 3 GitHub}:

Se puede observar en los resultados como no aparecen repositorios que usen GitHub Actions, a causa de que los repositorios que se han intentado buscar en esta ejecución son repositorios cuya fecha de creación es anterior al origen de GitHub Actions, y que además, han sido dejados de modificar tras su aparición, es decir, a diferencia de la segunda ejecución de este experimento final sobre proyectos GitHub, no podría darse el caso de que hubiesen migrado de otra herramienta a GitHub Actions.

En la tabla de contadores \ref{tab:tabla_f3_1} queda reflejada la exclusividad de uso de Travis CI para la implementación de la integración continua en proyectos GitHub de código abierto anteriores a la aparición de GitHub Actions, con un total de 110 trabajos y una media de 1'34 tareas por trabajo.

Como se ve reflejado en la tabla \ref{tab:tabla_f3_4}, javascript y java son los dos lenguajes más utilizados. En cuanto a los escenarios de ejecución se observa que el programa solamente ha encontrado trabajos sin escenarios definidos en etiquetas dedicadas exclusivamente para ello, como por ejemplo las etiquetas ``stage'', ``stages'' o la etiqueta ``on'' en el caso de GitHub Actions. Al no encontrar escenarios propiamente dichos, la tabla \ref{tab:tabla_f3_5} se ha rellenado con etiquetas del tipo ``script'', ``before\_install'', ``before\_script'' e ``install''.

\blankpage

% Capítulo 6

\chapter{Conclusiones y trabajos futuros}

La realización de este estudio me ha permitido, en primer lugar, ampliar mis conocimientos sobre las diferentes herramientas de integración continua existentes y conocer cuáles son las predominantes en entornos de alojamiento de proyectos software como GitHub y GitLab.
También he ampliado conocimientos en técnicas de ``research'' mediante la minería de repositorios utilizando las APIs de GitHub y GitLab para implementar un programa en Python, que es uno de los lenguajes que más está creciendo actualmente. 
Con 11.3 millones de usuarios, en el año 2021, Python ha sido, en cuanto a crecimiento, uno de los lenguajes de programación más a tener en cuenta, consiguiendo precisamente ocupar este vacío como favorito en desarrollo DS/ML, pero además siendo una opción para desarrollo de aplicaciones de IoT (internet de las cosas).

En primer lugar se estudiaron diversas herramientas de integración continua, tanto su funcionamiento como la estructura de los ficheros de configuración de trabajos automatizados con el objetivo de implementar un heurístico que fuese capaz de encontrar repositorios que usasen herramientas de este tipo.

A continuación se realizaron varios experimentos aplicando el heurístico, de los que se obtuvieron resultados suficientes como para poder tomar conclusiones al respecto.

Tras la realización de estos experimentos se puede destacar cómo cada vez es más habitual que los programadores que integran sus proyectos de forma continua utilicen herramientas propias de la plataforma en donde están alojando sus proyectos, en este caso GitHub Actions si el repositorio está en GitHub o GitLab CI en el caso de que se almacene en GitLab, probablemente debido a la facil integración que puedan tener con sus proyectos al estar almacenados en su propia plataforma.

No obstante, también se siguen usando otros medios para poder emplear integración continua como puedan ser Travis o incluso Jenkins que en un principio parecía que iban a ser más habituales en proyectos de código abierto GitHub o GitLab.\cotutor{Yo remarcaría el hecho de que hay aún proyectos "en transición" que tienen más de un sistema de CI (normalmente uno clásico y otro moderno, siendo este último el propio de la plataforma)}

En cuanto a los trabajos futuros, se intentarán ejecutar de algún modo los trabajos definidos en cada fichero YML o YAML de configuración y replicar tests automatizados para comprobar su efectividad. Además, se revisarán en cada repositorio subidas de código o ``commits'' antiguos para comprobar si en los inicios de los repositorios se utilizaban sistemas de integraciónn continua distintos al que tengan actualmente, es decir, comprobar si se produjo algún tipo de migración de sistemas y con qué objetivo. Por último, mencionar que se pondrán los resultados a disposición de investigadores interesados en estudiar otros aspectos sobre integración continua.


\blankpage


%%%%%%%%%%%%%%%%%%%%%%%%%%%%%%% Bibliografía %%%%%%%%%%%%%%%%%%%%%%%%%%%%%%%

\phantomsection
\addcontentsline{toc}{chapter}{Bibliografía}

\footnotesize{
%\bibliographystyle{hispa}
\bibliographystyle{IEEEtran}
\bibliography{bibliografia}
}



% No expandir elementos para llenar toda la página
\raggedbottom
\afterpage{\blankpage}

\newpage




%%%%%%%%%%%%%%%%%%%%%%%%%%%%%%% Apéndices %%%%%%%%%%%%%%%%%%%%%%%%%%%%%%%

\appendix

\phantomsection
\addcontentsline{toc}{chapter}{Apéndices}

\mbox{}
\vfill
\begin{center}
\begin{Huge}
\textbf{Apéndices}
\end{Huge}
\end{center}
\vfill
\mbox{}
\thispagestyle{empty}

\newpage
\mbox{}
\thispagestyle{empty}
\newpage


% Primer apéndice
\chapter{Minería de repositorios GitHub}
\label{sec:apendice1}

\section{Librería Python ``PyGithub''}

Para poder llevar a cabo sobre la plataforma GitHub la técnica de research conocida como minería de repositorios, MSR por sus siglas en inglés (Mining Software Repositories), se utiliza la biblioteca ``PyGithub'', la cual nos va a permitir manejar diferentes recursos de GitHub como repositorios, perfiles de usuario, organizaciones, etc. desde cualquier script Python.

Para instalar la biblioteca bastaría con ejecutar el comando pip install pygithub o clonarlo directamente desde el propio GitHub.

Una vez instalado ya se podría importar desde cualquier script Python mediante la siguiente instrucción:

\begin{lstlisting}[language=Python]
    from github import Github
\end{lstlisting}

Con la biblioteca ya importada en el script podemos generar un objeto ``GitHub'' y, partiendo de ese objeto, realizar consultas sencillas utilizando una serie de parámetros definidos por la librería.

\begin{lstlisting}[language=Python]
    usuario = "<usuario>" 
    token = "<token>"
    g = Github(usuario, token)
\end{lstlisting}

Al ejecutar una consulta, no se realiza ninguna consulta o búsqueda como tal, sino que se obtiene un objeto generador y se comienza a ejecutar la consulta al iterar sobre dicho objeto generador.

\begin{lstlisting}[language=Python]
    query= "<query>" 
    generator=g.search_repositories(query=query)
\end{lstlisting}



Listando el objeto generador ``generator'' devuelto por la función ``search\_repositories'' de la API de GitHub, se obtienen todos los repositorios que satisfacen la query pasada como parámetro. Esta lista de repositorios va a ser una lista de objetos “Repository” de los que se van a poder obtener una infinidad de información relativa a cada uno de ellos.

\section{Generación de token de autenticación GitHub}

Generar un token de autenticación GitHub nos permite aumentar el número de peticiones por hora que disponemos, pasando de tener 60 a 1500.
Este token de autenticación lo podemos generar accediendo a los ajustes de desarrollador de GitHub siguiendo los siguientes pasos:

\begin{enumerate}
    \item Desplegar ajustes de usuario pinchando en el icono redondo situado en la parte superior derecha de la pantalla y acceder a la opción ``Settings''.

    \begin{figure}[h]
        \centering
        \includegraphics[width=0.4\textwidth,clip=true]{\GitHubTokenA}
        \caption{Generar token GitHub. Paso 1.}
    \end{figure}

    \item Una vez dentro de “Settings”, acceder a la opción de menú ``Developer settings''.

    \begin{figure}[h]
        \centering
        \includegraphics[width=0.4\textwidth,clip=true]{\GitHubTokenB}
        \caption{Generar token GitHub. Paso 2.}
    \end{figure}

    \item Accediendo a la opción ``Personal Access tokens'' y pinchando el botón ``Generate new token'' nos permitirá rellenar los datos relacionados con el token que utilizaremos posteriormente para autenticarnos utilizando la API de GitHub.
    
    \begin{figure}[h]
        \centering
        \includegraphics[width=1\textwidth,clip=true]{\GitHubTokenC}
        \caption{Generar token GitHub. Paso 3.}
    \end{figure}

    \item Rellenar un nombre y seleccionar los permisos del token a generar. Para este caso seleccionar la opción ``repo'' sería más que suficiente.
    
    \begin{figure}[h]
        \centering
        \includegraphics[width=1\textwidth,clip=true]{\GitHubTokenD}
        \caption{Generar token GitHub. Paso 4.}
    \end{figure}

    \item Por último, pinchar en el botón de generar token.

    \begin{figure}[h]
        \centering
        \includegraphics[width=0.5\textwidth,clip=true]{\GitHubTokenE}
        \caption{Generar token GitHub. Paso 5.}
    \end{figure}

\end{enumerate}

% Segundo apéndice
\chapter{Minería de repositorios GitLab}
\label{sec:apendice2}

\section{Librería PyGilab}

AAA

\section{Generación de token de autenticación GitLab}

Para poder acceder a la API que proporciona la plataforma GitLab y poder obtener información sobre los diferentes proyectos de código abierto ``open source'' es necesario generar un token de autenticación de la siguiente manera:

\begin{enumerate}
    \item Desplegar ajustes de usuario pinchando en el icono situado en la parte superior derecha de la pantalla y acceder a la opción ``Preferences''.

    \begin{figure}[h]
        \centering
        \includegraphics[width=0.4\textwidth,clip=true]{\GitLabTokenA}
        \caption{Generar token GitHub. Paso 1.}
    \end{figure}

    \item En el desplegable situado a la izquierda de la pantalla, acceder a la opción de menú ``Access Tokens''.

    \begin{figure}[h]
        \centering
        \includegraphics[width=0.4\textwidth,clip=true]{\GitLabTokenB}
        \caption{Generar token GitHub. Paso 2.}
    \end{figure}

    \item A continuación aparecerá una pantalla para rellenar los campos necesarios para la generación del token de autenticación: nombre del token, fecha de validez y permisos.
    
    \begin{figure}[h]
        \centering
        \includegraphics[width=1\textwidth,clip=true]{\GitLabTokenC}
        \caption{Generar token GitHub. Paso 3.}
    \end{figure}

    \item Por último, una vez rellenos los campos de la pantalla con la información requerida, pinchar en el botón ``Create personal access token'' para generar el token de autenticación.

    \begin{figure}[h]
        \centering
        \includegraphics[width=0.5\textwidth,clip=true]{\GitLabTokenD}
        \caption{Generar token GitHub. Paso 4.}
    \end{figure}

\end{enumerate}

% Tercer apéndice
\chapter{Diagramas de flujo de ejecución de la herramienta}
\label{sec:apendice3}

\section{Flujo de ejecución GitHub}

\begin{figure}[h!]
    \centering
    \includegraphics[width=1\textwidth,clip=true]{\flujoGitHub}
    \caption{Flujo de ejecución GitHub.}
\end{figure}

\newpage

\section{Flujo de ejecución GitLab}

\begin{figure}[h!]
    \centering
    \includegraphics[width=1.15\textwidth,clip=true]{\flujoGitLab}
    \caption{Flujo de ejecución GitLab.}
\end{figure}

% Cuarto apéndice
\chapter{Tablas de resultados}
\label{sec:apendice4}

\section{Experimento preliminar}

%TABLA
\begin{table}
  \centering
  \caption{Exp. preliminar. Contadores.}
  \label{tab:tabla_p1}

\begin{footnotesize}
\renewcommand{\arraystretch}{1.5} % Para cambiar la separación entre filas (1 por defecto)
\begin{tabular}{ccccccccccc}
  \hline
  {} &  Encontrados GitHub &  Encontrados GitLab \\
  \hline
  jenkins         &                   5 &                   0 \\
  travis          &                  68 &                  25 \\
  circle ci       &                  40 &                   2 \\
  github actions  &                 242 &                  11 \\
  azure pipelines &                   5 &                   1 \\
  bamboo          &                   0 &                   0 \\
  concourse       &                   1 &                   1 \\
  gitlab ci       &                   0 &                 287 \\
  codeship        &                   0 &                   0 \\
  teamcity        &                   1 &                   0 \\
  bazel           &                   7 &                   1 \\
  semaphore ci    &                   0 &                   0 \\
  appveyor        &                   0 &                   0 \\
  \hline
  Totales         &                 310 &                 296 \\
 \end{tabular}
\end{footnotesize}

\end{table}

\newpage

\section{Experimento final}

\subsection{Ejecución nº 1 (GitHub y GitLab)}

%TABLA
\begin{table}
  \centering
  \caption{Exp. final, ejecución 1. Contadores.}
  \label{tab:tabla_f1_1}

\begin{footnotesize}
\renewcommand{\arraystretch}{1.5} % Para cambiar la separación entre filas (1 por defecto)
\begin{tabular}{ccccccccccc}
  \hline
  {} &  Encontrados GitHub &  Encontrados GitLab \\
  \hline
  jenkins         &                   7 &                   1 \\
  travis          &                 129 &                  37 \\
  circle ci       &                  62 &                   4 \\
  github actions  &                 450 &                  19 \\
  azure pipelines &                  10 &                   1 \\
  bamboo          &                   1 &                   0 \\
  gitlab ci       &                   4 &                 488 \\
  codeship        &                   0 &                   0 \\
  teamcity        &                   1 &                   0 \\
  bazel           &                  11 &                   1 \\
  semaphore ci    &                   0 &                   0 \\
  appveyor        &                   0 &                   0 \\
  \hline
  Totales         &                 567 &                 501 \\
 \end{tabular}
\end{footnotesize}

\end{table}

\begin{figure}
  \centering
  \includegraphics[width=0.95\textwidth,clip=true]{\graphA}
  \caption{Exp. final, ejecución 1. Contadores.}
\end{figure}

\begin{figure}
  \centering
  \includegraphics[width=0.65\textwidth,clip=true]{\graphB}
  \caption{Exp. final, ejecución 1. Contadores.}
\end{figure}

%TABLA
\begin{table}
  \centering
  \caption{Exp. final, ejecución 1. Lenguajes.}
  \label{tab:tabla_f1_2a}

\begin{footnotesize}
\renewcommand{\arraystretch}{1.5} % Para cambiar la separación entre filas (1 por defecto)
\begin{tabular}{ccccccccccc}
  \hline
  {} &  Jenkins &  Travis &  Circle CI &  GitHub Actions &  Azure Pipelines &  Bamboo \\
  \hline
  javascript       &        1 &      32 &          9 &              75 &                3 &       0 \\
  typescript       &        1 &       7 &         17 &              74 &                1 &       0 \\
  python           &        3 &      19 &         10 &              65 &                0 &       0 \\
  go               &        0 &      14 &          4 &              53 &                0 &       1 \\
  java             &        0 &      19 &          4 &              31 &                0 &       0 \\
  c++              &        1 &       6 &          4 &              27 &                5 &       0 \\
  ...              &      ... &     ... &        ... &             ... &              ... &     ... \\
  \hline
  Totales          &        7 &     129 &         62 &             450 &               10 &       1 \\
 \end{tabular}
\end{footnotesize}

\end{table}

%TABLA
\begin{table}
  \centering
  \caption{Exp. final, ejecución 1. Lenguajes (continuación).}
  \label{tab:tabla_f1_2b}

\begin{footnotesize}
\renewcommand{\arraystretch}{1.5} % Para cambiar la separación entre filas (1 por defecto)
\begin{tabular}{ccccccccccc}
  \hline
  {} &  GitLab CI &  Codeship &  TeamCity &  Bazel &  Semaphore CI &  AppVeyor &  TOTALES \\
  \hline
  javascript       &        0 &         0 &         0 &      0 &             0 &         0 &    120 \\
  typescript       &        0 &         0 &         1 &      2 &             0 &         0 &    103 \\
  python           &        1 &         0 &         0 &      2 &             0 &         0 &    100 \\
  go               &        0 &         0 &         0 &      2 &             0 &         0 &     74 \\
  java             &        1 &         0 &         0 &      0 &             0 &         0 &     55 \\
  c++              &        2 &         0 &         0 &      5 &             0 &         0 &     50 \\
  ...              &      ... &       ... &       ... &    ... &           ... &       ... &    ... \\
  \hline
  Totales          &        4 &         0 &         1 &     11 &             0 &         0 &      - \\
 \end{tabular}
\end{footnotesize}

\end{table}

%TABLA
\begin{table}
  \centering
  \caption{Exp. final, ejecución 1. Estadísticas de CI GitHub.}
  \label{tab:tabla_f1_3}

\begin{footnotesize}
\renewcommand{\arraystretch}{1.5} % Para cambiar la separación entre filas (1 por defecto)
\begin{tabular}{ccccccccccc}
  \hline
  {} &  Total repositorios &  Total trabajos &  Min &  Max &  Media &  Mediana \\
  \hline
  travis         &        129 &         281 &    1 &   14 &   2.18 &      2.0 \\
  github actions &        450 &        4213 &    1 &  569 &   9.36 &      4.0 \\
  gitlab ci      &          4 &          47 &    1 &   44 &  11.75 &      1.5 \\
 \end{tabular}
\end{footnotesize}

\end{table}

%TABLA
\begin{table}
  \centering
  \caption{Exp. final, ejecución 1. Estadísticas por lenguaje GitHub.}
  \label{tab:tabla_f1_4}

\begin{footnotesize}
\renewcommand{\arraystretch}{1.5} % Para cambiar la separación entre filas (1 por defecto)
\begin{tabular}{ccccccccccc}
  \hline
  {} &  Total repositorios &  Total trabajos &  Min &  Max &  Media &  Mediana \\
  \hline
  javascript       &        106 &         428 &    1 &   40 &   4.04 &      2.0 \\
  typescript       &         84 &         608 &    1 &   52 &   7.24 &      4.5 \\
  python           &         83 &         626 &    1 &   74 &   7.54 &      3.0 \\
  go               &         61 &         459 &    1 &   54 &   7.52 &      5.0 \\
  java             &         43 &         209 &    1 &   26 &   4.86 &      3.0 \\
  c++              &         36 &        1205 &    1 &  569 &  33.47 &      4.5 \\
  %rust             &         18 &         298 &    1 &   82 &  16.56 &     12.0 \\
  %vue              &         18 &         127 &    1 &   43 &   7.06 &      3.0 \\
  %none             &         15 &          35 &    1 &    7 &   2.33 &      2.0 \\
  %jupyter notebook &         15 &          29 &    1 &    5 &   1.93 &      2.0 \\
  %c\#               &         11 &          81 &    1 &   41 &   7.36 &      3.0 \\
  %shell            &          9 &          31 &    1 &   11 &   3.44 &      3.0 \\
  %swift            &          9 &          37 &    1 &    8 &   4.11 &      4.0 \\
  %html             &          9 &          12 &    1 &    3 &   1.33 &      1.0 \\
  %c                &          8 &          32 &    1 &   12 &   4.00 &      3.0 \\
  %kotlin           &          6 &          21 &    1 &   13 &   3.50 &      1.5 \\
  %ruby             &          6 &          60 &    1 &   19 &  10.00 &     10.0 \\
  %php              &          5 &          61 &    1 &   30 &  12.20 &      8.0 \\
  %dart             &          5 &          34 &    1 &   17 &   6.80 &      4.0 \\
  %scss             &          3 &           3 &    1 &    2 &   1.00 &      1.0 \\
  %objective-c      &          3 &          64 &    4 &   38 &  21.33 &     22.0 \\
  %css              &          2 &           0 &    0 &    0 &   0.00 &      0.0 \\
  %vim script       &          2 &           9 &    2 &    7 &   4.50 &      4.5 \\
  %v                &          1 &          53 &   53 &   53 &  53.00 &     53.0 \\
  %haskell          &          1 &           3 &    3 &    3 &   3.00 &      3.0 \\
  %tex              &          1 &           2 &    2 &    2 &   2.00 &      2.0 \\
  %jinja            &          1 &           3 &    3 &    3 &   3.00 &      3.0 \\
  %standard ml      &          1 &           2 &    2 &    2 &   2.00 &      2.0 \\
  %markdown         &          1 &           1 &    1 &    1 &   1.00 &      1.0 \\
  %scala            &          1 &           0 &    0 &    0 &   0.00 &      0.0 \\
  %cmake            &          1 &           2 &    2 &    2 &   2.00 &      2.0 \\
  %blade            &          1 &           1 &    1 &    1 &   1.00 &      1.0 \\
  %jsonnet          &          1 &           1 &    1 &    1 &   1.00 &      1.0 \\
  %dockerfile       &          1 &           4 &    4 &    4 &   4.00 &      4.0 \\
 \end{tabular}
\end{footnotesize}

\end{table}

%TABLA
\begin{table}
  \centering
  \caption{Exp. final, ejecución 1. Estadísticas de escenarios CI GitHub.}
  \label{tab:tabla_f1_5}

\begin{footnotesize}
\renewcommand{\arraystretch}{1.5} % Para cambiar la separación entre filas (1 por defecto)
\begin{tabular}{ccccccccccc}
  \hline
  {} &  Total proyectos &  Total escenarios \\
  \hline
  push                        &                 413 &          2999 \\
  pull\_request                &                 360 &          2070 \\
  schedule                    &                 189 &           532 \\
  workflow\_dispatch           &                 141 &          1191 \\
  script                      &                 111 &           117 \\
  install                     &                  63 &            69 \\
  release                     &                  56 &           168 \\
  %issues                      &                  55 &            91 \\
  %pull\_request\_target         &                  55 &            86 \\
  %issue\_comment               &                  42 &            69 \\
  %before\_install              &                  36 &            38 \\
  %before\_script               &                  32 &            32 \\
  %workflow\_run                &                  24 &            59 \\
  %repository\_dispatch         &                  15 &            82 \\
  %pull\_request\_review         &                   6 &             6 \\
  %workflow\_call               &                   5 &            28 \\
  %branch\_protection\_rule      &                   5 &             5 \\
  %status                      &                   4 &             4 \\
  %label                       &                   3 &             6 \\
  %after\_script                &                   3 &             3 \\
  %test                        &                   2 &            36 \\
  %tests                       &                   2 &             6 \\
  %deploy                      &                   2 &             3 \\
  %check\_suite                 &                   2 &             2 \\
  %page\_build                  &                   2 &             2 \\
  %discussion                  &                   2 &             2 \\
  %watch                       &                   2 &             2 \\
  %build                       &                   1 &            15 \\
  %build examples              &                   1 &             8 \\
  %create                      &                   1 &             7 \\
  %deployment\_status           &                   1 &             2 \\
  %export                      &                   1 &             2 \\
  %format                      &                   1 &             1 \\
  %lint                        &                   1 &             1 \\
  %pull\_request\_review\_comment &                   1 &             1 \\
  %pod lint                    &                   1 &             1 \\
  %danger                      &                   1 &             1 \\
  %collect\_artifacts           &                   1 &             1 \\
  %conditional\_build           &                   1 &             1 \\
  %delete                      &                   1 &             1 \\
 \end{tabular}
\end{footnotesize}

\end{table}

%TABLA
\begin{table}
  \centering
  \caption{Exp. final, ejecución 1. Lenguajes.}
  \label{tab:tabla_f1_6a}

\begin{footnotesize}
\renewcommand{\arraystretch}{1.5} % Para cambiar la separación entre filas (1 por defecto)
\begin{tabular}{ccccccccccc}
  \hline
  {} &  Jenkins &  Travis &  Circle CI &  GitHub Actions &  Azure Pipelines &  Bamboo \\
  \hline
  python           &        0 &       6 &          0 &               1 &                0 &       0 \\
  c                &        0 &       7 &          0 &               7 &                0 &       0 \\
  javascript       &        0 &       5 &          0 &               0 &                0 &       0 \\
  c++              &        0 &       3 &          1 &               5 &                1 &       0 \\
  ruby             &        0 &       2 &          0 &               0 &                0 &       0 \\
  rust             &        0 &       3 &          0 &               1 &                0 &       0 \\
  go               &        0 &       0 &          0 &               0 &                0 &       0 \\
  java             &        0 &       1 &          0 &               0 &                0 &       0 \\
  ...              &      ... &     ... &        ... &             ... &              ... &     ... \\
  \hline
  Totales          &        1 &      37 &          4 &              19 &                1 &       0 \\
 \end{tabular}
\end{footnotesize}

\end{table}

%TABLA
\begin{table}
  \centering
  \caption{Exp. final, ejecución 1. Lenguajes (continuación).}
  \label{tab:tabla_f1_6b}

\begin{footnotesize}
\renewcommand{\arraystretch}{1.5} % Para cambiar la separación entre filas (1 por defecto)
\begin{tabular}{ccccccccccc}
  \hline
  {} &  GitLab CI &  Codeship &  TeamCity &  Bazel &  Semaphore CI &  AppVeyor &  TOTALES \\
  \hline
  python           &        83 &         0 &         0 &      0 &             0 &         0 &     90.0 \\
  c                &        49 &         0 &         0 &      0 &             0 &         0 &     63.0 \\
  javascript       &        50 &         0 &         0 &      0 &             0 &         0 &     55.0 \\
  c++              &        41 &         0 &         0 &      0 &             0 &         0 &     51.0 \\
  ruby             &        33 &         0 &         0 &      0 &             0 &         0 &     35.0 \\
  rust             &        28 &         0 &         0 &      0 &             0 &         0 &     32.0 \\
  go               &        29 &         0 &         0 &      0 &             0 &         0 &     29.0 \\
  java             &        25 &         0 &         0 &      0 &             0 &         0 &     26.0 \\
  ...              &       ... &       ... &       ... &    ... &           ... &       ... &      ... \\
  \hline
  Totales          &       488 &         0 &         0 &      1 &             0 &         0 &        - \\
 \end{tabular}
\end{footnotesize}

\end{table}

%TABLA
\begin{table}
  \centering
  \caption{Exp. final, ejecución 1. Estadísticas de CI GitLab.}
  \label{tab:tabla_f1_7}

\begin{footnotesize}
\renewcommand{\arraystretch}{1.5} % Para cambiar la separación entre filas (1 por defecto)
\begin{tabular}{ccccccccccc}
  \hline
  {} &  Total repositorios &  Total trabajos &  Min &  Max &  Media &  Mediana \\
  \hline
  travis         &         37 &         103 &    1 &   10 &   2.78 &        2 \\
  github actions &         19 &          92 &    1 &   42 &   4.84 &        1 \\
  gitlab ci      &        488 &        4463 &    1 &  846 &   9.15 &        4 \\
 \end{tabular}
\end{footnotesize}

\end{table}

%TABLA
\begin{table}
  \centering
  \caption{Exp. final, ejecución 1. Estadísticas por lenguaje GitLab.}
  \label{tab:tabla_f1_8}

\begin{footnotesize}
\renewcommand{\arraystretch}{1.5} % Para cambiar la separación entre filas (1 por defecto)
\begin{tabular}{ccccccccccc}
  \hline
  {} &  Total repositorios &  Total trabajos &  Min &  Max &  Media &  Mediana \\
  \hline
  python           &         85 &        1418 &    1 &  846 &  16.68 &      5.0 \\
  javascript       &         52 &         214 &    1 &   18 &   4.12 &      3.0 \\
  c                &         51 &         506 &    1 &   62 &   9.92 &      6.0 \\
  c++              &         44 &         670 &    1 &  121 &  15.23 &      7.0 \\
  ruby             &         34 &         346 &    1 &   96 &  10.18 &      3.5 \\
  go               &         29 &         209 &    1 &   33 &   7.21 &      4.0 \\
  rust             &         28 &         212 &    1 &   39 &   7.57 &      5.0 \\
  java             &         25 &         117 &    1 &   11 &   4.68 &      4.0 \\
  %shell            &         25 &          88 &    1 &   10 &   3.52 &      3.0 \\
  %php              &         19 &         173 &    1 &   36 &   9.11 &      6.0 \\
  %typescript       &         15 &         150 &    2 &   26 &  10.00 &      9.0 \\
  %none             &         13 &          31 &    1 &    6 &   2.38 &      2.0 \\
  %kotlin           &          9 &          66 &    4 &   15 &   7.33 &      5.0 \\
  %emacs lisp       &          7 &          23 &    1 &   11 &   3.29 &      2.0 \\
  %c\#               &          6 &          71 &    1 &   28 &  11.83 &     10.0 \\
  %scheme           &          6 &          35 &    1 &   15 &   5.83 &      5.0 \\
  %makefile         &          4 &          20 &    2 &    8 &   5.00 &      5.0 \\
  %dockerfile       &          4 &           5 &    1 &    2 &   1.25 &      1.0 \\
  %lua              &          3 &           5 &    2 &    3 &   1.67 &      2.0 \\
  %haskell          &          3 &          24 &    2 &   19 &   8.00 &      3.0 \\
  %vue              &          3 &           8 &    2 &    4 &   2.67 &      2.0 \\
  %elixir           &          3 &           8 &    2 &    4 &   2.67 &      2.0 \\
  %perl             &          2 &          19 &    8 &   11 &   9.50 &      9.5 \\
  %tex              &          2 &           3 &    1 &    2 &   1.50 &      1.5 \\
  %fortran          &          2 &          36 &    8 &   28 &  18.00 &     18.0 \\
  %ocaml            &          2 &           8 &    1 &    7 &   4.00 &      4.0 \\
  %qml              &          1 &           2 &    2 &    2 &   2.00 &      2.0 \\
  %stylus           &          1 &           2 &    2 &    2 &   2.00 &      2.0 \\
  %erlang           &          1 &           2 &    2 &    2 &   2.00 &      2.0 \\
  %jsonnet          &          1 &          35 &   35 &   35 &  35.00 &     35.0 \\
  %scala            &          1 &           2 &    2 &    2 &   2.00 &      2.0 \\
  %matlab           &          1 &           1 &    1 &    1 &   1.00 &      1.0 \\
  %glsl             &          1 &          32 &   32 &   32 &  32.00 &     32.0 \\
  %common lisp      &          1 &           6 &    6 &    6 &   6.00 &      6.0 \\
  %swift            &          1 &           2 &    2 &    2 &   2.00 &      2.0 \\
  %qt script        &          1 &           3 &    3 &    3 &   3.00 &      3.0 \\
  %assembly         &          1 &           3 &    3 &    3 &   3.00 &      3.0 \\
  %racket           &          1 &           3 &    3 &    3 &   3.00 &      3.0 \\
  %objective-c      &          1 &           1 &    1 &    1 &   1.00 &      1.0 \\
  %vala             &          1 &           7 &    7 &    7 &   7.00 &      7.0 \\
  %vim script       &          1 &           3 &    3 &    3 &   3.00 &      3.0 \\
  %pug              &          1 &           3 &    3 &    3 &   3.00 &      3.0 \\
  %nix              &          1 &           2 &    2 &    2 &   2.00 &      2.0 \\
  %plpgsql          &          1 &          27 &   27 &   27 &  27.00 &     27.0 \\
  %gdscript         &          1 &           8 &    8 &    8 &   8.00 &      8.0 \\
  %powershell       &          1 &           3 &    3 &    3 &   3.00 &      3.0 \\
  %handlebars       &          1 &           1 &    1 &    1 &   1.00 &      1.0 \\
  %crystal          &          1 &           6 &    6 &    6 &   6.00 &      6.0 \\
  %bluespec         &          1 &           4 &    4 &    4 &   4.00 &      4.0 \\
  %openscad         &          1 &           9 &    9 &    9 &   9.00 &      9.0 \\
  %jupyter notebook &          1 &          22 &   22 &   22 &  22.00 &     22.0 \\
  %hcl              &          1 &           4 &    4 &    4 &   4.00 &      4.0 \\
 \end{tabular}
\end{footnotesize}

\end{table}

%TABLA
\begin{table}
  \centering
  \caption{Exp. final, ejecución 1. Estadísticas de escenarios CI GitLab.}
  \label{tab:tabla_f1_9}

\begin{footnotesize}
\renewcommand{\arraystretch}{1.5} % Para cambiar la separación entre filas (1 por defecto)
\begin{tabular}{ccccccccccc}
  \hline
  {} &  Total proyectos &  Total escenarios \\
  \hline
  build                      &                 202 &           715 \\
  test                       &                 200 &          1075 \\
  script                     &                 180 &           558 \\
  before\_script              &                 151 &           156 \\
  deploy                     &                 136 &           354 \\
  release                    &                  34 &            64 \\
  %workflow                   &                  30 &            30 \\
  %before\_install             &                  23 &            31 \\
  %install                    &                  23 &            26 \\
  %prepare                    &                  15 &            42 \\
  %lint                       &                  15 &            39 \\
  %package                    &                  15 &            38 \\
  %push                       &                  13 &            81 \\
  %publish                    &                  12 &            28 \\
  %pull\_request               &                   9 &            70 \\
  %after\_script               &                   9 &            14 \\
  %triage                     &                   7 &            12 \\
  %review                     &                   7 &            12 \\
  %docs                       &                   6 &             9 \\
  %qa                         &                   5 &            11 \\
  %pages                      &                   5 &             5 \\
  %doc                        &                   5 &             5 \\
  %build\_and\_test             &                   4 &            21 \\
  %check                      &                   4 &             8 \\
  %.pre                       &                   4 &             6 \\
  %pull\_request\_target        &                   4 &             4 \\
  %production                 &                   4 &             4 \\
  %benchmark                  &                   3 &            49 \\
  %compile                    &                   3 &            18 \\
  %build-image                &                   3 &            13 \\
  %analysis                   &                   3 &             7 \\
  %docker                     &                   3 &             7 \\
  %format                     &                   3 &             6 \\
  %dependencies               &                   3 &             5 \\
  %cleanup                    &                   3 &             4 \\
  %trigger                    &                   3 &             4 \\
  %post                       &                   3 &             4 \\
  %images                     &                   3 &             3 \\
  %staging                    &                   3 &             3 \\
  %build\_image                &                   3 &             3 \\
  %.post                      &                   3 &             3 \\
  %coverage                   &                   3 &             3 \\
  %integration                &                   3 &             3 \\
  %schedule                   &                   3 &             3 \\
  %unit\_test                  &                   2 &           116 \\
  %extras                     &                   2 &            32 \\
  %static                     &                   2 &             7 \\
  %tests                      &                   2 &             6 \\
  %unit-tests                 &                   2 &             6 \\
  %installer                  &                   2 &             5 \\
  %tarball                    &                   2 &             4 \\
  %sanitychecks               &                   2 &             4 \\
  %workflow\_dispatch          &                   2 &             4 \\
  %notify                     &                   2 &             3 \\
  %distribution               &                   2 &             3 \\
  %analyse                    &                   2 &             3 \\
  %deploy:staging             &                   2 &             3 \\
  %deploy:production          &                   2 &             3 \\
  %bootstrap                  &                   2 &             2 \\
  %issues                     &                   2 &             2 \\
  %build-docker               &                   2 &             2 \\
  %image                      &                   2 &             2 \\
  %dist                       &                   2 &             2 \\
  %bench                      &                   2 &             2 \\
  %version                    &                   2 &             2 \\
  %environment                &                   2 &             2 \\
  %internal                   &                   2 &             2 \\
  %alpha                      &                   2 &             2 \\
  %beta                       &                   2 &             2 \\
  %deploy:canary              &                   2 &             2 \\
  %pre\_build                  &                   2 &             2 \\
  %scan                       &                   1 &           225 \\
  %cudnn                      &                   1 &           132 \\
  %cuda                       &                   1 &           129 \\
  %tiki-check                 &                   1 &            21 \\
  %testing                    &                   1 &            20 \\
  %prebuild                   &                   1 &            19 \\
  %stage1-testing             &                   1 &            19 \\
  %advanced\_build             &                   1 &            18 \\
  %deb                        &                   1 &            15 \\
  %snowflake               &                   1 &             9 \\
  %eval                       &                   1 &             9 \\
  %python                   &                   1 &             8 \\
  %manually\_triggered\_tests   &                   1 &             7 \\
  %compatibility              &                   1 &             7 \\
  %content-generation         &                   1 &             6 \\
  %confidence-check           &                   1 &             6 \\
  %linux build                &                   1 &             6 \\
  %checks                     &                   1 &             6 \\
  %docker\_images              &                   1 &             6 \\
  %deploy\_production          &                   1 &             6 \\
  %test-plugins               &                   1 &             5 \\
  %finalization               &                   1 &             5 \\
  %finish-publish             &                   1 &             5 \\
  %installation               &                   1 &             5 \\
  %quality-assurance          &                   1 &             4 \\
  %externalbuild              &                   1 &             4 \\
  %stage\_test                 &                   1 &             4 \\
  %generate\_test\_data         &                   1 &             4 \\
  %build-backend              &                   1 &             4 \\
  %fstest                     &                   1 &             4 \\
  %build\_unit\_test            &                   1 &             4 \\
  %static\_analysis            &                   1 &             4 \\
  %test and build             &                   1 &             4 \\
  %release-note               &                   1 &             4 \\
  %testall                    &                   1 &             3 \\
  %container                  &                   1 &             3 \\
  %builds                     &                   1 &             3 \\
  %stage\_build\_docker         &                   1 &             3 \\
  %build\_test                 &                   1 &             3 \\
  %deploy-frontend            &                   1 &             3 \\
  %darwin build               &                   1 &             3 \\
  %push artifacts             &                   1 &             3 \\
  %post-test                  &                   1 &             3 \\
  %update                     &                   1 &             3 \\
  %vm                         &                   1 &             3 \\
  %deps                       &                   1 &             3 \\
  %prod\_deploy                &                   1 &             3 \\
  %packaging                  &                   1 &             3 \\
  %exe                        &                   1 &             3 \\
  %zip                        &                   1 &             3 \\
  %ftp                        &                   1 &             3 \\
  %repository                 &                   1 &             3 \\
  %build:prod                 &                   1 &             3 \\
  %deploy:prod                &                   1 &             3 \\
  %build-and-deploy           &                   1 &             2 \\
  %code-quality               &                   1 &             2 \\
  %style                      &                   1 &             2 \\
  %stage\_push\_docker          &                   1 &             2 \\
  %stage\_deploy               &                   1 &             2 \\
  %scheduled                  &                   1 &             2 \\
  %deploy-backend             &                   1 &             2 \\
  %linux shell                &                   1 &             2 \\
  %darwin shell               &                   1 &             2 \\
  %cleanup\_success            &                   1 &             2 \\
  %cleanup\_failure            &                   1 &             2 \\
  %audit                      &                   1 &             2 \\
  %preflight                  &                   1 &             2 \\
  %release-package            &                   1 &             2 \\
  %int-test                   &                   1 &             2 \\
  %release-docker             &                   1 &             2 \\
  %flatpak                    &                   1 &             2 \\
  %dbt docs                 &                   1 &             2 \\
  %triage run                 &                   1 &             2 \\
  %vendors-security           &                   1 &             2 \\
  %post-run                   &                   1 &             2 \\
  %deps-prepare               &                   1 &             2 \\
  %pre-merge                  &                   1 &             2 \\
  %setup                      &                   1 &             2 \\
  %deploy-dev                 &                   1 &             2 \\
  %testinstaller              &                   1 &             2 \\
  %quality                    &                   1 &             2 \\
  %notification\_fail          &                   1 &             1 \\
  %notifications              &                   1 &             1 \\
  %archive                    &                   1 &             1 \\
  %sanity\_checks              &                   1 &             1 \\
  %externalpostbuild          &                   1 &             1 \\
  %stage\_build\_documentation  &                   1 &             1 \\
  %stage\_push\_documentation   &                   1 &             1 \\
  %stage\_build\_python         &                   1 &             1 \\
  %stage\_push\_python          &                   1 &             1 \\
  %snapshot                   &                   1 &             1 \\
  %paperwork                  &                   1 &             1 \\
  %export                     &                   1 &             1 \\
  %automation                 &                   1 &             1 \\
  %checkicons                 &                   1 &             1 \\
  %outdated                   &                   1 &             1 \\
  %clean\_up                   &                   1 &             1 \\
  %coverage\_report            &                   1 &             1 \\
  %upload\_report              &                   1 &             1 \\
  %provision-backend          &                   1 &             1 \\
  %build-frontend             &                   1 &             1 \\
  %documentation              &                   1 &             1 \\
  %static\_check               &                   1 &             1 \\
  %docker-image               &                   1 &             1 \\
  %vendors                    &                   1 &             1 \\
  %multi-arch                 &                   1 &             1 \\
  %validate\_deploy            &                   1 &             1 \\
  %build-\_new                 &                   1 &             1 \\
  %test-docker                &                   1 &             1 \\
  %push-docker                &                   1 &             1 \\
  %push-docker-master         &                   1 &             1 \\
  %save-to-cubi               &                   1 &             1 \\
  %cleanup-review             &                   1 &             1 \\
  %cleanup-staging            &                   1 &             1 \\
  %stage2-tarball             &                   1 &             1 \\
  %ci-build                   &                   1 &             1 \\
  %clickbuild                 &                   1 &             1 \\
  %pre\_deploy                 &                   1 &             1 \\
  %code\_standarts             &                   1 &             1 \\
  %staging-release            &                   1 &             1 \\
  %stable                     &                   1 &             1 \\
  %specs                      &                   1 &             1 \\
  %report                     &                   1 &             1 \\
  %schema                     &                   1 &             1 \\
  %api                        &                   1 &             1 \\
  %sentry\_release             &                   1 &             1 \\
  %verify                     &                   1 &             1 \\
  %scheduled\_monitoring\_tests &                   1 &             1 \\
  %smoke\_tests                &                   1 &             1 \\
  %deploy\_staging             &                   1 &             1 \\
  %build\_production           &                   1 &             1 \\
  %build-app                  &                   1 &             1 \\
  %snap                       &                   1 &             1 \\
  %prepare-publish            &                   1 &             1 \\
  %reproducible               &                   1 &             1 \\
  %snowflake stop           &                   1 &             1 \\
  %static-analysis            &                   1 &             1 \\
  %shellcheck                 &                   1 &             1 \\
  %triggers                   &                   1 &             1 \\
  %dist-linux-amd64           &                   1 &             1 \\
  %dev                        &                   1 &             1 \\
  %build-src                  &                   1 &             1 \\
  %go-test                    &                   1 &             1 \\
  %go-fmt                     &                   1 &             1 \\
  %docker-registry-master     &                   1 &             1 \\
  %docker-registry-tags       &                   1 &             1 \\
  %package-tiki               &                   1 &             1 \\
  %vendors\_update             &                   1 &             1 \\
  %venv-setup                 &                   1 &             1 \\
  %buildx                     &                   1 &             1 \\
  %build\_latest               &                   1 &             1 \\
  %pre-commit                 &                   1 &             1 \\
  %generate-release           &                   1 &             1 \\
  %chore                      &                   1 &             1 \\
  %stop                       &                   1 &             1 \\
  %testbuild                  &                   1 &             1 \\
  %buildrelease               &                   1 &             1 \\
  %ade                        &                   1 &             1 \\
  %build\_pypi                 &                   1 &             1 \\
  %vars                       &                   1 &             1 \\
  %build-containers           &                   1 &             1 \\
  %eslint                     &                   1 &             1 \\
  %bitrisebuildandroid        &                   1 &             1 \\
  %post-merge                 &                   1 &             1 \\
  %failed\_stage               &                   1 &             1 \\
  %upload                     &                   1 &             1 \\
  %admin\_test\_deploy          &                   1 &             1 \\
  %admin\_prod\_deploy          &                   1 &             1 \\
  %reports                    &                   1 &             1 \\
  %build\_report\_application   &                   1 &             1 \\
  %lint\_test\_coverage         &                   1 &             1 \\
  %siril                      &                   1 &             1 \\
  %uploadstore                &                   1 &             1 \\
  %build-dev                  &                   1 &             1 \\
  %build-staging              &                   1 &             1 \\
  %execute-change-prod        &                   1 &             1 \\
  %deploy-staging             &                   1 &             1 \\
  %create-change-prod         &                   1 &             1 \\
  %build-final                &                   1 &             1 \\
  %page\_deploy                &                   1 &             1 \\
  %demo                       &                   1 &             1 \\
  %destroy                    &                   1 &             1 \\
  %build-binary               &                   1 &             1 \\
  %results                    &                   1 &             1 \\
  %infrastructure             &                   1 &             1 \\
  %restart                    &                   1 &             1 \\
  %register                   &                   1 &             1 \\
  %linting                    &                   1 &             1 \\
  %generate\_documentation     &                   1 &             1 \\
  %build-docs                 &                   1 &             1 \\
  %zenodo                     &                   1 &             1 \\
  %build\_docker               &                   1 &             1 \\
  %build\_and\_deploy\_docker    &                   1 &             1 \\
  %build\_and\_deploy\_website   &                   1 &             1 \\
  %setup:prod                 &                   1 &             1 \\
  %migrate:prod               &                   1 &             1 \\
  %qual                       &                   1 &             1 \\
  %smoke-test                 &                   1 &             1 \\
  %build\_netbsd               &                   1 &             1 \\
  %check\_suite                &                   1 &             1 \\
  %multiver-test              &                   1 &             1 \\
  %mirror                     &                   1 &             1 \\
 \end{tabular}
\end{footnotesize}

\end{table}

\newpage

\subsection{Ejecución nº 2 (GitHub)}
%TABLA
\begin{table}[h]
  \centering
  \caption{Exp. final, ejecución 2. Contadores.}
  \label{tab:tabla_f2_1}

\begin{footnotesize}
\renewcommand{\arraystretch}{1.5} % Para cambiar la separación entre filas (1 por defecto)
\begin{tabular}{ccccccccccc}
  \hline
  {} &  Encontrados GitHub \\
  \hline
  jenkins         &                  14 \\
  travis          &                 212 \\
  circle ci       &                  78 \\
  github actions  &                 499 \\
  azure pipelines &                  14 \\
  bamboo          &                   0 \\
  gitlab ci       &                   9 \\
  codeship        &                   0 \\
  teamcity        &                   2 \\
  bazel           &                  12 \\
  semaphore ci    &                   0 \\
  appveyor        &                   0 \\
  \hline
  Totales         &                 669 \\
 \end{tabular}
\end{footnotesize}

\end{table}

\begin{figure}
  \centering
  \includegraphics[width=0.95\textwidth,clip=true]{\graphC}
  \caption{Exp. final, ejecución 2. Contadores.}
\end{figure}

%TABLA
\begin{table}
  \centering
  \caption{Exp. final, ejecución 2. Lenguajes.}
  \label{tab:tabla_f2_2a}

\begin{footnotesize}
\renewcommand{\arraystretch}{1.5} % Para cambiar la separación entre filas (1 por defecto)
\begin{tabular}{ccccccccccc}
  \hline
  {} &  Jenkins &  Travis &  Circle CI &  GitHub Actions &  Azure Pipelines &  Bamboo \\
  \hline
  javascript    &        0 &      63 &         20 &             114 &                1 &       0 \\
  python        &        0 &      18 &          9 &              55 &                5 &       0 \\
  java          &        5 &      25 &          3 &              51 &                1 &       0 \\
  go            &        3 &       6 &         13 &              51 &                1 &       0 \\
  typescript    &        2 &       6 &         10 &              46 &                1 &       0 \\
  c++           &        1 &      15 &          5 &              38 &                1 &       0 \\
  c             &        0 &      12 &          3 &              19 &                1 &       0 \\
  ...           &      ... &     ... &        ... &             ... &              ... &     ... \\
  \hline
  Totales       &       14 &     212 &         78 &             499 &               14 &       0 \\
 \end{tabular}
\end{footnotesize}

\end{table}

%TABLA
\begin{table}
  \centering
  \caption{Exp. final, ejecución 2. Lenguajes (continuación).}
  \label{tab:tabla_f2_2b}

\begin{footnotesize}
\renewcommand{\arraystretch}{1.5} % Para cambiar la separación entre filas (1 por defecto)
\begin{tabular}{ccccccccccc}
  \hline
  {} &  GitLab CI &  Codeship &  TeamCity &  Bazel &  Semaphore CI &  AppVeyor &  TOTALES \\
  \hline
  javascript    &        2 &         0 &         1 &      0 &             0 &         0 &    201.0 \\
  python        &        1 &         0 &         0 &      1 &             0 &         0 &     89.0 \\
  java          &        1 &         0 &         0 &      2 &             0 &         0 &     88.0 \\
  go            &        1 &         0 &         1 &      2 &             0 &         0 &     78.0 \\
  typescript    &        0 &         0 &         0 &      3 &             0 &         0 &     68.0 \\
  c++           &        0 &         0 &         0 &      4 &             0 &         0 &     64.0 \\
  c             &        0 &         0 &         0 &      0 &             0 &         0 &     35.0 \\
  ...           &      ... &       ... &       ... &    ... &           ... &       ... &      ... \\
  \hline
  Totales       &        9 &         0 &         2 &     12 &             0 &         0 &        - \\
 \end{tabular}
\end{footnotesize}

\end{table}

%TABLA
\begin{table}
  \centering
  \caption{Exp. final, ejecución 2. Estadísticas de CI GitHub.}
  \label{tab:tabla_f2_3}

\begin{footnotesize}
\renewcommand{\arraystretch}{1.5} % Para cambiar la separación entre filas (1 por defecto)
\begin{tabular}{ccccccccccc}
  \hline
  {} &  Total repositorios &  Total trabajos &  Min &  Max &  Media &  Mediana \\
  \hline
  travis         &        212 &         588 &    1 &   94 &   2.77 &        2 \\
  github actions &        499 &        3562 &    1 &  312 &   7.14 &        4 \\
  gitlab ci      &          9 &          44 &    1 &    9 &   4.89 &        4 \\
 \end{tabular}
\end{footnotesize}

\end{table}

%TABLA
\begin{table}
  \centering
  \caption{Exp. final, ejecución 2. Estadísticas por lenguaje GitHub.}
  \label{tab:tabla_f2_4}

\begin{footnotesize}
\renewcommand{\arraystretch}{1.5} % Para cambiar la separación entre filas (1 por defecto)
\begin{tabular}{ccccccccccc}
  \hline
  {} &  Total repositorios &  Total trabajos &  Min &  Max &  Media &  Mediana \\
  \hline
  javascript    &        176 &         774 &    1 &   42 &    4.40 &      2.0 \\
  python        &         68 &         473 &    1 &   43 &    6.96 &      4.0 \\
  java          &         66 &         405 &    1 &   94 &    6.14 &      3.0 \\
  go            &         58 &         472 &    1 &  107 &    8.14 &      6.0 \\
  typescript    &         52 &         407 &    1 &   35 &    7.83 &      4.0 \\
  c++           &         46 &         307 &    1 &   23 &    6.67 &      5.5 \\
  %none          &         29 &          60 &    1 &   14 &    2.07 &      2.0 \\
  %c             &         25 &         290 &    1 &   43 &   11.60 &      4.0 \\
  %php           &         17 &          98 &    1 &   15 &    5.76 &      4.0 \\
  %ruby          &         16 &          78 &    1 &   16 &    4.88 &      4.0 \\
  %swift         &         15 &          42 &    1 &   12 &    2.80 &      2.0 \\
  %objective-c   &         15 &          36 &    1 &   10 &    2.40 &      2.0 \\
  %shell         &         11 &          76 &    1 &   21 &    6.91 &      4.0 \\
  %kotlin        &         11 &          39 &    1 &   17 &    3.55 &      2.0 \\
  %html          &         10 &          17 &    1 &    4 &    1.70 &      2.0 \\
  %c\#            &          9 &          26 &    1 &   10 &    2.89 &      2.0 \\
  %css           &          6 &          18 &    2 &    7 &    3.00 &      2.5 \\
  %scala         &          5 &          51 &    2 &   20 &   10.20 &     11.0 \\
  %rust          &          3 &          19 &    4 &    8 &    6.33 &      7.0 \\
  %vim script    &          3 &          48 &    8 &   28 &   16.00 &     12.0 \\
  %scss          &          3 &          11 &    2 &    5 &    3.67 &      4.0 \\
  %clojure       &          2 &          36 &    1 &   35 &   18.00 &     18.0 \\
  %lua           &          2 &          15 &    2 &   13 &    7.50 &      7.5 \\
  %emacs lisp    &          2 &           7 &    3 &    4 &    3.50 &      3.5 \\
  %ocaml         &          2 &           3 &    3 &    3 &    1.50 &      1.5 \\
  %elixir        &          2 &           9 &    3 &    6 &    4.50 &      4.5 \\
  %dockerfile    &          2 &           6 &    1 &    5 &    3.00 &      3.0 \\
  %dart          &          1 &           3 &    3 &    3 &    3.00 &      3.0 \\
  %assembly      &          1 &           2 &    2 &    2 &    2.00 &      2.0 \\
  %julia         &          1 &           2 &    2 &    2 &    2.00 &      2.0 \\
  %markdown      &          1 &           4 &    4 &    4 &    4.00 &      4.0 \\
  %roff          &          1 &           8 &    8 &    8 &    8.00 &      8.0 \\
  %haskell       &          1 &          12 &   12 &   12 &   12.00 &     12.0 \\
  %batchfile     &          1 &           2 &    2 &    2 &    2.00 &      2.0 \\
  %asciidoc      &          1 &           3 &    3 &    3 &    3.00 &      3.0 \\
  %coffeescript  &          1 &           0 &    0 &    0 &    0.00 &      0.0 \\
  %tex           &          1 &           6 &    6 &    6 &    6.00 &      6.0 \\
  %makefile      &          1 &           8 &    8 &    8 &    8.00 &      8.0 \\
  %mustache      &          1 &         312 &  312 &  312 &  312.00 &    312.0 \\
  %objective-c++ &          1 &           4 &    4 &    4 &    4.00 &      4.0 \\
  %perl          &          1 &           2 &    2 &    2 &    2.00 &      2.0 \\
  %jinja         &          1 &           3 &    3 &    3 &    3.00 &      3.0 \\
 \end{tabular}
\end{footnotesize}

\end{table}

%TABLA
\begin{table}
  \centering
  \caption{Exp. final, ejecución 2. Estadísticas de escenarios CI GitHub.}
  \label{tab:tabla_f2_5}

\begin{footnotesize}
\renewcommand{\arraystretch}{1.5} % Para cambiar la separación entre filas (1 por defecto)
\begin{tabular}{ccccccccccc}
  \hline
  {} &  Total proyectos &  Total escenarios \\
  \hline
  push                        &                 447 &          2215 \\
  pull\_request                &                 435 &          2073 \\
  schedule                    &                 193 &           540 \\
  script                      &                 159 &           195 \\
  workflow\_dispatch           &                 135 &           875 \\
  install                     &                  81 &           103 \\
  %before\_install              &                  70 &            80 \\
  %before\_script               &                  66 &            72 \\
  %pull\_request\_target         &                  65 &           110 \\
  %release                     &                  53 &            98 \\
  %issues                      &                  44 &            79 \\
  %issue\_comment               &                  42 &            65 \\
  %workflow\_run                &                  17 &            33 \\
  %workflow\_call               &                  13 &            53 \\
  %repository\_dispatch         &                  11 &            31 \\
  %test                        &                   9 &            73 \\
  %create                      &                   9 &            31 \\
  %branch\_protection\_rule      &                   7 &             7 \\
  %after\_script                &                   6 &             6 \\
  %status                      &                   4 &             5 \\
  %pull\_request\_review         &                   3 &             3 \\
  %build                       &                   2 &            15 \\
  %lint                        &                   2 &             3 \\
  %deploy                      &                   2 &             2 \\
  %tests - phase 2             &                   1 &            68 \\
  %tests - phase 1             &                   1 &            10 \\
  %build-all                   &                   1 &             5 \\
  %vim74                       &                   1 &             4 \\
  %vim8                        &                   1 &             3 \\
  %build-one                   &                   1 &             3 \\
  %unit tests                  &                   1 &             3 \\
  %neovim                      &                   1 &             2 \\
  %test shell                  &                   1 &             2 \\
  %gollum                      &                   1 &             1 \\
  %build process               &                   1 &             1 \\
  %lint and unit               &                   1 &             1 \\
  %\_npm ci                     &                   1 &             1 \\
  %workflow                    &                   1 &             1 \\
  %create-vars                 &                   1 &             1 \\
  %pull\_request\_review\_comment &                   1 &             1 \\
  %label                       &                   1 &             1 \\
  %delete                      &                   1 &             1 \\
  %travis                      &                   1 &             1 \\
  %test-docs                   &                   1 &             1 \\
  %push-tx                     &                   1 &             1 \\
  %validations                 &                   1 &             1 \\
  %test-sbt-1.6.x              &                   1 &             1 \\
  %moderator                   &                   1 &             1 \\
  %cron                        &                   1 &             1 \\
 \end{tabular}
\end{footnotesize}

\end{table}

\newpage

\subsection{Ejecución nº 3 (GitHub)}
%TABLA
\begin{table}[h]
  \centering
  \caption{Exp. final, ejecución 3. Contadores.}
  \label{tab:tabla_f3_1}

\begin{footnotesize}
\renewcommand{\arraystretch}{1.5} % Para cambiar la separación entre filas (1 por defecto)
\begin{tabular}{ccccccccccc}
  \hline
  {} &  Encontrados GitHub \\
  \hline
  jenkins         &                   0 \\
  travis          &                  82 \\
  circle ci       &                   0 \\
  github actions  &                   0 \\
  azure pipelines &                   0 \\
  bamboo          &                   0 \\
  gitlab ci       &                   0 \\
  codeship        &                   0 \\
  teamcity        &                   0 \\
  bazel           &                   0 \\
  semaphore ci    &                   0 \\
  appveyor        &                   0 \\
  \hline
  Totales         &                  82 \\
 \end{tabular}
\end{footnotesize}

\end{table}

\begin{figure}
  \centering
  \includegraphics[width=0.95\textwidth,clip=true]{\graphD}
  \caption{Exp. final, ejecución 3. Contadores.}
\end{figure}

\newpage

%TABLA
\begin{table}[h]
  \centering
  \caption{Exp. final, ejecución 3. Lenguajes.}
  \label{tab:tabla_f3_2a}

\begin{footnotesize}
\renewcommand{\arraystretch}{1.5} % Para cambiar la separación entre filas (1 por defecto)
\begin{tabular}{ccccccccccc}
  \hline
  {} &  Jenkins &  Travis &  Circle CI &  GitHub Actions &  Azure Pipelines &  Bamboo \\
  \hline
  javascript   &        0 &      34 &          0 &               0 &                0 &       0 \\
  java         &        0 &      13 &          0 &               0 &                0 &       0 \\
  objective-c  &        0 &       9 &          0 &               0 &                0 &       0 \\
  ruby         &        0 &       6 &          0 &               0 &                0 &       0 \\
  ...          &      ... &     ... &        ... &             ... &              ... &     ... \\
  \hline
  Totales      &        0 &      82 &          0 &               0 &                0 &       0 \\
 \end{tabular}
\end{footnotesize}

\end{table}

%TABLA
\begin{table}[h]
  \centering
  \caption{Exp. final, ejecución 3. Lenguajes (continuación).}
  \label{tab:tabla_f3_2b}

\begin{footnotesize}
\renewcommand{\arraystretch}{1.5} % Para cambiar la separación entre filas (1 por defecto)
\begin{tabular}{ccccccccccc}
  \hline
  {} &  GitLab CI &  Codeship &  TeamCity &  Bazel &  Semaphore CI &  AppVeyor &  TOTALES \\
  \hline
  javascript   &        0 &         0 &         0 &      0 &             0 &         0 &     34.0 \\
  java         &        0 &         0 &         0 &      0 &             0 &         0 &     13.0 \\
  objective-c  &        0 &         0 &         0 &      0 &             0 &         0 &      9.0 \\
  ruby         &        0 &         0 &         0 &      0 &             0 &         0 &      6.0 \\
  ...          &      ... &       ... &       ... &    ... &           ... &        ...&      ... \\
  \hline
  Totales      &        0 &         0 &         0 &      0 &             0 &         0 &        - \\
 \end{tabular}
\end{footnotesize}

\end{table}

%TABLA
\begin{table}
  \centering
  \caption{Exp. final, ejecución 3. Estadísticas de CI GitHub.}
  \label{tab:tabla_f3_3}

\begin{footnotesize}
\renewcommand{\arraystretch}{1.5} % Para cambiar la separación entre filas (1 por defecto)
\begin{tabular}{ccccccccccc}
  \hline
  {} &  Total repositorios &  Total trabajos &  Min &  Max &  Media &  Mediana \\
  \hline
  travis         &         82 &         110 &    1 &    4 &   1.34 &        1 \\
  github actions &          0 &           0 &    0 &    0 &   0.00 &        0 \\
  gitlab ci      &          0 &           0 &    0 &    0 &   0.00 &        0 \\
 \end{tabular}
\end{footnotesize}

\end{table}

%TABLA
\begin{table}
  \centering
  \caption{Exp. final, ejecución 3. Estadísticas por lenguaje GitHub.}
  \label{tab:tabla_f3_4}

\begin{footnotesize}
\renewcommand{\arraystretch}{1.5} % Para cambiar la separación entre filas (1 por defecto)
\begin{tabular}{ccccccccccc}
  \hline
  {} &  Total repositorios &  Total trabajos &  Min &  Max &  Media &  Mediana \\
  \hline
  javascript   &         34 &          35 &    1 &    3 &   1.03 &      1.0 \\
  java         &         13 &          19 &    1 &    3 &   1.46 &      1.0 \\
  objective-c  &          9 &          12 &    1 &    2 &   1.33 &      1.0 \\
  ruby         &          6 &           9 &    2 &    3 &   1.50 &      2.0 \\
  %go           &          3 &           4 &    2 &    2 &   1.33 &      2.0 \\
  %css          &          3 &           2 &    2 &    2 &   0.67 &      0.0 \\
  %python       &          2 &           6 &    3 &    3 &   3.00 &      3.0 \\
  %c            &          2 &           4 &    1 &    3 &   2.00 &      2.0 \\
  %php          &          2 &           3 &    1 &    2 &   1.50 &      1.5 \\
  %r            &          2 &           6 &    3 &    3 &   3.00 &      3.0 \\
  %scala        &          1 &           1 &    1 &    1 &   1.00 &      1.0 \\
  %groovy       &          1 &           1 &    1 &    1 &   1.00 &      1.0 \\
  %typescript   &          1 &           4 &    4 &    4 &   4.00 &      4.0 \\
  %coffeescript &          1 &           2 &    2 &    2 &   2.00 &      2.0 \\
  %erlang       &          1 &           1 &    1 &    1 &   1.00 &      1.0 \\
  %lex          &          1 &           1 &    1 &    1 &   1.00 &      1.0 \\
 \end{tabular}
\end{footnotesize}

\end{table}

%TABLA
\begin{table}
  \centering
  \caption{Exp. final, ejecución 3. Estadísticas de escenarios CI GitHub.}
  \label{tab:tabla_f3_5}

\begin{footnotesize}
\renewcommand{\arraystretch}{1.5} % Para cambiar la separación entre filas (1 por defecto)
\begin{tabular}{ccccccccccc}
  \hline
  {} &  Total repositorios &  Total escenarios \\
  \hline
  script         &                  45 &            45 \\
  before\_install &                  28 &            28 \\
  before\_script  &                  21 &            21 \\
  install        &                  14 &            14 \\
  after\_script   &                   2 &             2 \\
 \end{tabular}
\end{footnotesize}

\end{table}

% Fin del documento
\end{document}
